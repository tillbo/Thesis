\section{Chapter overview}
As discussed in Chapter \ref{cha:intro}, diabetes has reached epidemic proportions in \acp{MIC} and is a major contributing factor to disabling poor health and early mortality. The economic impact of diabetes on individuals and healthcare systems has, however, received relatively little attention. Further, little is known about how successful healthcare systems currently are in encouraging behaviour change in those diagnosed in order to prevent the disabling complications of diabetes. This is despite the fact, that until now efforts to halt the increase in disease prevalence have been of little success. Consequently, the goal of this thesis was assessing the economic burden of diabetes in \acp{MIC}. This should help to better understand the importance of primary and secondary prevention of diabetes and identify those populations must susceptible to the adverse economic effects of diabetes.

To meet these aims, four separate studies were conducted around the following questions:
\begin{enumerate}
\item What is the current evidence on the economic costs of type 2 diabetes?
\item What is the causal effect of self-reported diabetes on labour market outcomes in a \ac{MIC}?
\item In how far can the results gained from self-reported diabetes data be used to characterize the entire diabetes population?
\item What is the effect of health information provided by a diabetes diagnosis in terms of affecting health behaviours in a \ac{MIC}?
\end{enumerate}

This concluding chapter has four parts. Firstly, it summarises the principal findings. Secondly, it contextualises the findings within the wider literature and provides implications for policies. Thirdly it reflects on the methods. Finally, there are suggestions for future research and concluding comments.

\section{Summary of principal findings}

Chapter \ref{cha:review} set out to provide an overview of and critically assess existing studies on the economic costs of type 2 diabetes globally. This not only included so called \ac{COI} studies but also studies on labour market outcomes. Systematic review methods were used and the evidence was synthesized narratively. 86 \ac{COI} studies and 23 labour market studies were identified. Of those, 24 came from \acp{LMIC}, with 23 being \ac{COI} studies.

For \ac{COI} studies, the review found a large range of estimated costs, with the largest per-capita costs being generated in the USA while costs were generally lower in \acp{LMIC}. However, it also found that the direct economic burden caused by the treatment of type 2 diabetes is much higher in poorer countries when taking into account the lower income levels, in particular for the poorest parts of the population. Treatment costs were paid almost entirely out-of-pocket by the poor due to a lack of health insurance coverage, consuming considerable parts of the annual income.  The review also found considerable differences in the used methodologies and the study quality. This made it difficult to directly compare the studies. While in many \acp{HIC} studies an incremental costing approach was used and data sources were representative for a distinct population, studies in developing countries often had to rely on non-representative, relatively small convenience samples, often lacking a control group. Many studies also lacked explicit mentioning of the used study perspective or the included costing components. 

For labour market impact studies, most found adverse effects of self-reported diabetes on employment probabilities, wages or working days. Studies were concentrated on a few \ac{HIC}, in particular the USA. More recent studies took into account potential biases due to the endogeneity of diabetes, mainly using an \ac{IV} strategy with the family history of diabetes as an instrument. However, the direction of bias was ambiguous across different studies and countries. 

The review also identified methodological and thematic areas so far only sparingly covered. So did no \ac{COI} studies take into account the possibility of biased estimates as a result of endogeneity of diabetes. Consequently, there is a lack of evidence in the literature about the potential bias in the cost estimates of diabetes \ac{COI} studies. Further, few studies used an incidence approach to investigate lifetime costs of people with diabetes, providing better information about the dynamics of cost increases after a diabetes diagnosis. 

Despite these identified limitations of the \ac{COI} literature, the review was able to provided a picture of the healthcare costs of diabetes in almost every continent. This was not the case for labour market studies, where almost no evidence was found for \acp{LMIC}. Arguably, given the less advanced healthcare systems, later diagnosis but earlier onset of diabetes, the larger informal labour market and overall different labour market structure in developing countries, the impact of diabetes could be very different compared to \acp{HIC}. Also in terms of methodology, studies had not taken advantage of panel data techniques to achieve a causal interpretation of their estimates. Especially studies on the effect on employment probabilities had relied on the same---at least debatable---identification strategy using \acp{IV}. Therefore, a study using a different identification strategy was warranted.

Importantly, no study investigating the impact of unobserved diabetes on labour market outcomes was identified by the review. Hence, an important part of the diabetes population had been mostly neglected. This left open the question in how far results for self-reported diabetes were applicable to the unaware population.

Based on the findings of the review, the three research studies that followed addressed parts of the identified gaps, in particular focusing on labour market outcomes. The aim of Chapter \ref{cha:Mex1} was to provide first evidence for the impact of diabetes on employment probabilities in a developing country, where diabetes had become a public health concern. Because little was known about the equity impacts of diabetes, a further goal was to investigate the heterogeneity of effects across formal and informal employment and for the "rich" and "poor". Due to the unavailability of an alternative identification strategy, the study applied the already established \ac{IV} approach using parental diabetes. However, using further familial background information on parental education, it improved upon earlier studies by controlling for a potential confounding pathway that could have invalidated the used instrument. It further used two methods to implement the \ac{IV} approach. The preferred estimates came from a bivariate probit model that had been shown to be better suited for our specific data, in comparison to a standard linear \ac{IV} model. We nonetheless also provided the results of the latter approach.

Chapter \ref{cha:Mex1} found evidence for an adverse effect of diabetes on employment chances, reducing them by about 10 percentage points for men and 5 percentage points for women. Further, diabetes was not found to be endogenous, suggesting the use of a simple probit model. The subgroup analysis showed that the adverse employment effects occurred mainly to those above age 44, while younger people seemed less affected. Also, being poorer increase the exposure to negative employment effects of diabetes. The same was the case for those in the informal compared to those in the formal labour market. Across all models, the point estimates were bigger for males than for females. 

While these results provided good evidence for an adverse effect of diabetes on employment chances in a developing country, several questions identified in Chapter \ref{cha:review} still remained. Further, the robustness of the findings of Chapter \ref{cha:Mex1} had to be tested using more extensive and recent data and a different identification strategy. Chapter \ref{cha:Mex2} addressed these issues taking advantage of a recent extension to the data used in Chapter \ref{cha:Mex1}. The data now spanned three waves and eight years, which allowed for the use of a longitudinal individual fixed effects model to estimate the relationship of self-reported diabetes with employment. Additionally, the investigated labour market outcomes were extended to wages and working hours. Further, it was now possible to investigate the relationship of diabetes duration with labour outcomes, in order to understand when diabetes tended to cause adverse labour market outcomes. Importantly, the additional wave also provided information on diabetes biomarkers to investigate the effects of diabetes for the entire diabetes population as well as those unaware of diabetes separately.

The analysis carried out in Chapter \ref{cha:Mex2} confirmed the adverse relationship of self-reported diabetes with employment, finding a five percentage point reduction for males and females alike. Given the relatively low female employment rate, this translated into a 16\% decrease in employment probabilities women compared to 6\% for men. Compared to the cross-sectional results of Chapter \ref{cha:Mex1}, the estimated effects of the \ac{FE} model are about half the size for men, but are similar and of stronger statistical significance for women. This is likely due to the additional data used in Chapter \ref{cha:Mex2}, but could also partly be the result of the different estimation technique. For wages and working hours no adverse effects of self-reported diabetes were found.

Further analysis showed that the most adverse effects were concentrated among self-employed and independent agricultural workers, potentially due to lower job security and access to healthcare in these often informal jobs. Further, Chapter \ref{cha:Mex2} revealed that the adverse effect of diabetes on employment appeared shortly after diagnosis, then levelled off for some time until it appeared again. This pattern was observed for both males and females, albeit only statistically significant for the former. Interestingly it was found that when the employment effects levelled off, wages started to fall, again for both genders. This suggested that during this period reductions in productivity mainly reduced wages, protecting against job loss.

Finally, the results of the biomarker analysis presented in Chapter \ref{cha:Mex2} showed that relying on self-reported diabetes information can lead to measurement bias in the coefficient of diabetes. Using the biomarker data to identify people with diabetes, compared to self-reported diabetes smaller effects especially on employment probabilities were found. This was caused by the non-existent associations between undiagnosed diabetes and employment chances. It was further found, that part of the difference in effects between self-reported and undiagnosed diabetes could be explained by differences in subjective health status, with those self-reporting diabetes also reporting a worse health status. Interestingly, differences in \ac{HbA1c} levels did not drive the stronger effects for those self-reporting. %Add ttest to Chapter 4 showing difference in subjective health  

Chapters \ref{cha:Mex1} and \ref{cha:Mex2} produced evidence of the adverse effect of self-reported diabetes on labour market outcomes in Mexico. Chapter \ref{cha:China} continued the investigation of the impact of self-reported diabetes on employment probabilities, but this time on China. It further investigated how a diabetes diagnosis affected diabetes relevant health behaviours in a developing country. Because the relationships may be biased due to unobservables and selection, the study used two different econometric strategies in \acp{MSM} and \ac{FE}. Each controlled for a different source of confounding, improving the robustness of the identified effects. The used dataset consisted of six waves of the \ac{CHNS}, covering a period from 1997 to 2011.

The results provided further evidence of a deterioration of employment probabilities after a diabetes diagnosis, though this time only for women. They experienced a reduction in employment chances between 11 to 12 percentage points. For men, the \ac{FE} and \ac{MSM} showed insignificant relationships. The results for health behaviours also suggested different effects for men and women. For health behaviours, all results indicated that men were able to reduce alcohol consumption, \ac{BMI} levels, waist circumference and their daily calorie consumption. These reductions also translated into lower probabilities of being obese, potentially leading to important reductions in a variety of risk factors \parencite{Wilding2014}. For women, no strong evidence for similar reductions was found. A similar picture remained when investigating the effects over time using linear and non-linear specifications. They suggested maintained reductions in female employment probabilities over time. Men were able to reduce their \ac{BMI} and waist circumference consistently in the years following diagnosis.

These results point to very different effects of a diabetes diagnosis for men and women. On the one hand women were unable to reduce their risk factors, but men were. On the other hand, women also had to bear the main adverse economic effects. 

\section{The context of the findings and their implications}

The findings of this thesis indicate an important global economic burden of diabetes and have added further evidence on the effect of self-reported diabetes on labour market outcomes in \acp{MIC}. The thesis also showed that diabetes---at least in the case of labour market outcomes---did not similarly affect the unaware diabetes population as it did those aware. Additionally it showed, that a diabetes diagnosis can elicit positive changes in health behaviours. Further, several potential equity issues are brought ot light, were the burden of diabetes appears to be distributed unequally, disproportionally affecting the poor, those in the informal labour market and women.

These findings may lead to several implications to reduce the economic burden of diabetes in \acp{MIC}. 

\subsection*{Inequities in the economic burden of diabetes}

An important implication of this thesis are the found economic inequities in the burden of diabetes. In Chapter \ref{cha:review} the review found a high \ac{OOP} burden in \acp{LMIC}, especially for those with no insurance coverage. Chapter \ref{cha:Mex1} showed that the adverse employment effects were concentrated among those in the informal labour market and with fewer resources. This was further supported by findings from Chapter \ref{cha:Mex2} that indicated a greater reduction in employment probabilities to work in the agricultural or self-employed sector, while for those working in a non-independent wage job---that often entails greater contractual job security and better access to health insurance---diabetes did not appear to elicit negative effects. Chapter \ref{cha:China} found bigger adverse employment effects and less positive behavioural changes in women compared to men after they had received a diabetes diagnosis. These gender inequities are also supported by the results for Mexico, in particular by \ref{cha:Mex2}, where, taking into account the lower overall employment rates for women in Mexico, the relative reduction in employment chances was much greater for females than for males. Accordingly, three, likely overlapping populations, being disproportionally affected by diabetes can be identified from this thesis: the poor, those in the informal labour market and women.

There may be several potential strategies how to reduce these inequalities and improve access to care. Several of these will be presented here with a focus on the identified populations in this thesis.

\subsubsection*{The identification of people with diabetes}

Adverse labour market outcomes were only observed for the self-reporting diabetes population, suggesting that the adverse impact manifest only after some time of living with the disease and mostly after diagnosis. This is not surprising given the gradual increase in blood glucose as diabetes progresses and with this a relatively slow deterioration of health \parencite{Bertram2010}. A first important step to reduce the economic burden of diabetes could therefore be the earlier diagnosis of diabetes. The large undiagnosed population found in Mexico in Chapter \ref{cha:Mex2} as well as for other \acp{LMIC} in a recent study by \textcite{Beagley2014}, suggests that in \acp{LMIC} many people with diabetes remain undiagnosed for an extended period of time. Even though this would increase healthcare demands and costs in the short term, such effects may be set off by increases in productivity and productive life years in the working age population with diabetes, as well as lower inpatient expenditures due to reduced rates of severe, cost-intensive complications such as dialysis  \parencite{Engelgau2012}. Evidence on the cost-effectiveness of a population-based diabetes screening program provided a recent study from Brazil, where over 22 million people over the age of 40 were screened for diabetes \parencite{Toscano2015}, being the first evaluating an actual real-life population-based diabetes screening program in a developing country. It was unclear if the program could be considered good value for the healthcare system, as the cost-effectiveness of the findings depended strongly on the used assumptions about how effective treatment would be in preventing coronary heart disease and stroke. Given the results from this thesis, cost-effectiveness may be greater from a societal perspective if an earlier diagnosis would prevent or decrease losses in productivity and productive lifespan. Of course, early diagnosis may only be reasonable if the healthcare system is sufficiently developed to allow all diagnosed cases access to appropriate treatment options \parencite{Toscano2015,Engelgau2012}. 

Apart from the likely worse health in the population aware of its diabetes, another policy relevant reason for the difference in the observed effects could be the psychological effect of a diabetes diagnosis. Reductions in productivity may be the result of increasing anxiety and depression as a result of becoming aware of the disease and its potential consequences. Further, difficulties in adapting to the treatment regime may cause additional stress. As discussed in \ref{cha:Mex2}, there is some evidence that becoming aware of the disease leads to reductions in labour income likely due to its psychological effects \parencite{Liu2014}. If this is confirmed by other studies, then strategies to provide better guidance and support at diagnosis and thereafter to reduce the psychological burden of the disease could be worthwhile.



\subsubsection{Diabetes treatment in resource constraint settings}

The adverse labour market effects found for those with self-reported diabetes and the increase in effect size over time after diagnosis, suggests that they are not able to prevent these adverse economic outcomes from happening. This may have several reasons. One may be that diagnosis happens so late that complications are already present at diagnosis and it becomes increasingly difficult to prevent further complications. Another reason could be the sub-optimal treatment of the disease, in particular in the most adversely affected---likely socioeconomically disadvantaged---groups identified in this thesis.
Therefore, an important step to improve outcomes would be the provision of high quality diabetes treatment. The existing evidence on treatment models applicable in very resource constrained settings has recently been reviewed by \textcite{Esterson2014}. While the evidence is still limited, the study provided information on interventions that have had some success in improving diabetes treatment for the poor. Further, it identified common characteristics of these successful interventions: collaboration, education, standardization of guidelines and algorithms, technological innovations, and resource optimization. The authors recommended that initiatives to provide care to underserved populations should be build on collaborations between academic institutions, hospitals, the private sector and other organizations such as local governments. This should help to achieve goals that would otherwise be difficult to reach for one stakeholder alone. Further, programs should aim at providing appropriate education to doctors to increase their ability to successfully treat people with diabetes. For very remote communities \textcite{Esterson2014} suggested the use of peer-support programs, so that few well educated community members or nurses can help their peers with the challenges of diabetes management. Further, a need for standardized guidelines and treatment algorithms was identified as a help for healthcare professionals to improve and maintain their standards of care. Given that mobile phones have already reached even very remote areas and are common in the developing world, interventions based on existent technologies could improve care and diabetes outcomes. They could facilitate communication between doctors and their patients as well as tracking and controlling diabetes management and outcome measures. Finally, resource optimization to use available and constrained resources more effectively, e.g., by transferring certain responsibilities from doctors to nurses or from healthcare professionals to peers could be an option in very resource constrained settings \parencite{Esterson2014}. Together, the presented strategies could help in reaching and treating poorer parts of the population.

Further, in \acp{MIC}, the provision of universal health care has been advocated for as a means to reduce health inequities by providing everyone with the ability to access healthcare \parencite{Marmot2008}. This could enable those in the informal economy to access affordable treatments, narrowing inequities.  Mexico has been one of the countries where the goal of universal health care has been almost accomplished through the introduction of "Seguro Popular", which provides those without prior health insurance coverage with social security and access to diabetes treatment options \parencite{Knaul2012,Rivera-Hernandez2016}. However, evidence on the impact of diabetes treatment and outcomes has shown that the availability of this program has only led to very modest improvements, only finding a positive effect on the use of pharmacological therapy. No effects were found on the monitoring of blood glucose or adherence to exercise plans by people with diabetes \parencite{Rivera-Hernandez2016}. A potential mechanism mentioned by the authors was that many clinics were not prepared to provide specialized diabetes care and medications, suggesting that barriers to accessing appropriate diabetes care and education still existed. Hence, while public health care provision for those previously uninsured can reduce inequities, such programs need to ensure that their efforts are not sabotaged by the low quality of the offered services.

\subsubsection*{Diabetes prevention in resource constrained settings}

Apart from improving the quality of care for people with diabetes with lesser resources, prevention of diabetes may also reduce the observed inequities and the individual economic burden of diabetes. Given the inequities found in this thesis, such efforts may be particularly worthwhile if they focus on those disproportionally affected by the adverse economic effects of diabetes.

One option are population level interventions. There is already some real life evidence of such interventions with the goal of reducing obesity in developing countries. In Mexico, a 10\% tax on purchases of sugar-sweetened beverages and "junk food" has been introduced in 2014. First results suggest a reduction in purchases of these goods, with a steeper decline for those with lower income levels \parencite{Colchero2016,Batis2016}. If these changes in consumption actually lead to a healthier diet and are large enough to cause reductions in obesity and diabetes prevalence has not been evaluated yet and remains to be seen. Other efforts to prevent diabetes in \acp{LMIC} included increasing the awareness of diabetes and how to prevent it via population level campaigns, and efforts to increase physical activity by providing easy access to sport courses and fitness equipment \textcite{Cefalu2016}. 

Another option is the identification of at risk groups and providing them with interventions to increase physical activity and dietary changes. These have shown promising results across the globe, including in developing countries such as India and China where interventions have caused long term reductions in the risk of developing diabetes \parencite{Cefalu2016}. The evidence for pharmacological interventions mainly using metformin also indicates a reduction in the risk of diabetes. However, \textcite{Cefalu2016} also mention the potentially large heterogeneity in the benefit of pharmacological interventions across ethnicities. More research to this respect will be needed to find out if successful pharmacological interventions in one ethnicity can be translated to other ethnicities. Further, all these interventions were tested in randomized controlled trials, and translation into real-world settings has been less successful, even in high-income countries \parencite{Wareham2016}. There are also questions about the cost-effectiveness of these interventions if scaled to a population level and the problem of finding sufficiently educated personal to implement these lifestyle interventions at the local level. \textcite{Cefalu2016} argue that preventive metformin treatment in individuals with a high risk of progressing to diabetes may be the best approach in countries with few economic resources. Low-cost generic versions of metformin exist, are considered essential diabetes medications in almost all \acp{LMIC} \parencite{Bazargani2014}, are effective in preventing or delaying the onset of diabetes, and are save \parencite{Gomes2013}. They therefore may present a relatively cost-effective intervention that could be applied using existent healthcare infrastructure and pharmacies. It could be especially effective in \acp{MIC}, where the healthcare system infrastructure is much more developed than in \acp{LIC}.

Obviously, the identification of high-risk individuals that could be targeted with the mentioned interventions may pose an additional hurdle to successfully preventing diabetes. Population level screening could be a way to identify people at risk. Screening could also be carried out at the workplace or the community and existing medical records could be used to identify people at an increased risk and there may be possibilities to raise self-assess the risk using online resources after being made aware of it by advertising and social media \parencite{Cefalu2016}. However, scientific evidence of the cost-effectiveness and feasibility of screening for high-risk individuals in \acp{LMIC} is non-existent, and if it where to happen may overwhelm health care systems, and may also increase health inequities if the lower income populations were less likely to attend screening efforts \parencite{Wareham2016}.


\subsubsection{The need to account for gender differences}
Finally, and one of the main results of this thesis, is the need to specifically address the needs of women. Gender differences in disease burden of diabetes have only come to the forefront recently \parencite{Peters2015}, but may hold one of the keys to reducing the economic burden of diabetes. The reduction in inequalities by the strategies discussed above may already lead to a reduction in the observed gender differences. If women have fewer economic resources then men, are more likely to work in the informal labour market and less likely to be insured \parencite{Galli2008} and therefore are more adversely affected by diabetes, then interventions targeting the poor and uninsured should also help women. However, it appears that biological differences between men and women may make it necessary to specifically target women, as diabetes likely affects them to a greater extend \parencite{Peters2015,Peters2014a,Peters2014,Bertram2010}, which could be driving the observed differences in the economic effects. Efforts to reduce the burden for females would include increasing awareness among doctors about the higher risks for women of diabetes complications as well as screening for cardiovascular risk factors in women at or before a diabetes diagnosis. This would present an opportunity to prevent a further escalation of the cardiovascular risk profile before severe complications develop \parencite{Peters2015}. For women, weight reduction thereby seems to be the single most important step to reduce the risk of diabetes and ensuing complications \parencite{Peters2015}. As this thesis has shown, women in China were not able to achieve weight reduction to the extend men did and therefore may need to be treated differently. Strategies will need to be developed that can foster this in \acp{LMIC}. 

Overall it seems that for \acp{LMIC}, population level interventions to prevent diabetes are currently the best option to halt the escalation of the economic impact of diabetes and to reduce inequities. The results of this thesis suggest that it should be a priority to design interventions that address the existent inequities by preventing diabetes in those populations that experience the worse economic consequences. Further, targeting those with little access to healthcare in screening programs for both undiagnosed diabetes and those at high-risk for diabetes, and then following up with offers for preventive pharmacological treatment and potentially also lifestyle interventions could be of value. Further, strategies to improve treatment of diabetes will need to take into account the specific circumstances of the respective target group and should be developed in cooperative efforts in order to make them work.



\section{Reflections on the methods used in the thesis}

Apart from Chapter \ref{cha:review}, the thesis used exclusively quantitative methods in an attempt to establish causal relationships between diabetes and the outcomes of interest. The goal of using econometrics in this thesis was to investigate the relationship of diabetes with labour market outcomes and health behaviours in the absence of experimental data. Given the good quality of available data and the dearth of previous quantitative research on the economics of diabetes in \ac{MIC}, using econometric methods seemed to be the most appropriate way to answer the posed research questions and to provide evidence for different geographical regions.

One of the challenges was the choice of the most appropriate method to establish a causal relationship. The main concern was that unobserved variables, measurement error as well as reverse causality may introduce bias into the estimates. A variety of methods were used that each had advantages and disadvantages in terms of the underlying assumptions and the ability to account for potential sources of bias. Nonetheless, regardless of the method used, results consistently showed an adverse relationship of self-reported diabetes with employment probabilities, suggesting a relatively robust and likely causal effect.

\section{Strengths and limitations}

The strengths and limitations of each study and the methodological approach used have been evaluated within each chapter. Additionally, the thesis overall has strengths and limitations.

A strength of this thesis is the provision of a a comprehensive overview and assessment of the state of economic research on the impact of diabetes, providing other researchers guidance by identifying areas for future research and the use of the most appropriate methods to do so. Further, the thesis itself fills some of the identified gaps by investigating the impact of diabetes on labour market outcomes in \acp{MIC}. A strength of these analyses is the use of rigorous econometric approaches taking advantage of available and previously underexplored household data, allowing to explore a variety of topics. The used methods also improved on previous approaches, providing more robust evidence.  A further strength is the provision of evidence on the potential of diabetes to widen the economic inequities in developing countries, identifying the groups that were disproportionally affected by the disease. Further, it has also advanced the understanding of diabetes as a multifaceted condition by exploring effects over time and for those with observed and unobserved diabetes. 

The thesis has several limitations. Whilst the intention was to provide evidence on the economics of diabetes in \acp{MIC}, the thesis mostly investigates the economic impact of diabetes. While this provides important information for researchers and policy makers, the thesis did not investigate how to curb this economic diabetes burden. Information about the best and most costs-effective interventions that could be applied in \acp{MIC} to lower the burden of diabetes is urgently needed as information about who is affected most will not suffice to effectively reduce the burden. Research on how to implement interventions feasible in non-\ac{HIC} settings is therefore urgently needed and has not been provided in this thesis.

This leads to the next limitation. The thesis does not investigate in how far healthcare systems in \acp{MIC} need to change in order to better provide care. Because they often lack financial resources, do not efficiently use the available resources, are designed to treat acute infectious diseases rather then affecting the outcomes of long-lasting non-acute \acp{NCD}, and often provide unequal access to their health services due to financial constraints of those seeking care, research into how to strengthen these systems is urgently needed \parencite{Mills2014,Guzman2010}.

A further limitation is the geographical concentration of the thesis in its empirical investigation. While Mexico and China are among the ten countries with most people with diabetes in the world, there are other large and small \acp{MIC} currently facing similar challenges \parencite{Risk2016}.  It cannot be assumed that the evidence provided in this thesis is representative of other \acp{MIC}. It therefore will be important to investigate the economic burden and potential solutions in other countries, given their own specific context in terms of culture, the political system, economic development and equity and equality issues.

Finally, while the thesis intended to provide a picture of the potential inequities in the economic impact of diabetes for socioeconomic subgroups, it did not investigate in detail why these inequities exist and could only speculate on the reasons. A better understanding of the underlying reasons will be paramount to design adequate strategies to address these inequities. Further, whilst the thesis has touched upon the potential reasons for the differences in employment effects between those self-reporting diabetes and those unaware, it has not provided an in depth analysis of this phenomenon. A better identification of the underlying reasons will be needed to design interventions to prevent the adverse economic effects of diabetes. 



\section{Suggestions for future research}

This thesis has shown the global economic impact of diabetes and its adverse effect on labour market outcomes in Mexico and China. It identified the poor, those in the informal economy and women as being most adversely affected by the disease. It further found that, at least in China, it is men that appear to profit most from a diabetes diagnosis in terms of positively changing their health behaviours. Finally, it provided some indication that while self-reported diabetes is related to adverse labour market effects, undiagnosed diabetes is not. Without a greater understanding of the underlying reasons for the found differences, it will be difficult to design policies that can help prevent the burden of diabetes in \ac{MIC} and reduce inequalities.

Several reasons for the observed gender differences in the impact of diabetes have been discussed in this thesis, including biological differences that increase the risk of complications \textcite{Peters2014,Peters2015,Arnetz2014,Roche2013,Policardo2014,Catalan2015,Engelmann2016,Seghieri2015} and may also impair the ability of women to lose weight \parencite{Penno2013}, as well as differences in the access to appropriate healthcare \parencite{Penno2013}. One strategy to further investigate the reasons would be the use of biomarker data in combination with information on healthcare utilization as well as socioeconomic outcomes. This could be used to investigate potential heterogeneities in the relationship between diabetes and overall metabolic health with labour market outcomes. Further, information on healthcare usage could be used to investigate if differences in healthcare access mediate the economic impact of diabetes. A potentially rich source of information would be two Chinese household surveys, the \acf{CHNS} and the \acf{CHARLS}. Both provide an extensive list of measured biomarkers and socioeconomic variables that could help to investigate differences in metabolic risk between men and women. This information may also be used to further investigate differences in metabolic risk between people aware and unaware of their diabetes.

Second, researchers should try to confirm the results of the found inequities using different data from other countries in order to establish the relationship between diabetes related labour market outcomes and poverty. If this relationship is confirmed the underlying drivers of these inequities need to be explored to design adequate policies. This could be done by identifying countries where this relationship may not have been found to isolate the causal determinants. Further, strategies implemented currently or in the future in \acp{MIC} that aim to reduce these inequities, such as the implementation of universal health insurance schemes such as Seguro Popular in Mexico, need to be evaluated in how far they are actually achieving this goal in terms of diabetes. The same is true for population level interventions such as taxes certain foods or nutrients, as these theoretically should reduce consumption in particular for those with lower levels of income \parencite{Mytton2012c}. This could then lead to a reduction in diabetes incidence in these groups. However, depending on the price elasticities of the taxed products as well as substitution effects with equally untaxed products, such taxes may only reduce disposable income and could reduce other food purchases or could lead to a shift in consumption towards other equally unhealthy untaxed products \parencite{Mytton2012c}.


Third, the diabetes population in all countries, but especially in \acp{LMIC} is only partially observed. In other words many people with diabetes are not aware that they have the disease. This thesis has provided an investigation of the differences between those aware and unaware. It, however, still remains unclear to what extend different factors such as health information and actual health status are causing this heterogeneity in the economic impact in both groups. Because increasingly household surveys are providing biomarker data in combination with socioeconomic information, they should be used together with quasi-experimental econometric techniques to investigate this topic. A regression-discontinuity design may be used in a similar vain as in \textcite{Zhao2013a}, who use cut-off values for hypertension to identify those newly diagnosed and the subsequent effect of this diagnosis on health behaviours. A similar approach could be used to explore the effects of a diabetes diagnosis and the entailed health information on labour market outcomes, health behaviours and other economic outcomes. Importantly, researches should assess the heterogeneity of effects across income groups, rural versus urban, education levels and between males and females. This would provide important information for designing interventions to reduce the physiological and economic burden of diabetes and preventing a widening of inequities.

Fourth and finally, there is a need to explore further economic downstream effects of the economic impact of diabetes. If diabetes causes reductions in employment and potentially also income, it is likely that these will cause not only problems for the individual directly affected but for the people closely related such as family members. In \acp{MIC}, where social security is less extensive and comprehensive, adverse health shocks due to diabetes could have consequences for the children, spouses or other family members living in affected households \parencite{Alam2014}. The loss in income needs to be compensated either by increasing labour supply of other household members or by reducing expenditures for other goods. Both could affect children directly, for example by reducing their time for education due to reduced household resources to pay for tuition fees and also by substituting time for education with labour time. Similarly spouses may be forced to increase their labour supply, reducing the time they can care for their children. These effects have remained unexplored for diabetes but given the scale of the diabetes epidemic may not be trivial.

Overall, researchers should address these questions using the data that is increasingly coming out to increase the understanding of the effects of diabetes on the physiology and psychology of people especially, though not exclusively, in \acp{MIC}.


\section{Concluding remarks}

Diabetes presents a major challenge for \acp{MIC}, but evidence on its economics has been scarce. This thesis has found that diabetes has an adverse economic impact on individuals and puts a burden on healthcare systems. Because evidence on the impact of diabetes on labour market outcomes was lacking in developing countries, the thesis had a special focus on this topic. Thereby it not only provided evidence of the adverse impact of diabetes on employment, but also improved upon previously used econometric methods by using novel strategies to identify a causal relationship. The thesis also identified potential inequities in the impact of diabetes, pointing to larger adverse effects for the poor, those in the informal labour market and women. The thesis also found evidence for differences in the ability to change health behaviours for men and women, with men being more successful then women. 

These findings suggest that there is a need to reduce the economic impact of diabetes in \acp{MIC}. Considering the increasingly earlier onset of diabetes and the ongoing increase in incidence in many countries, the non-trivial adverse economic effects could otherwise hinder economic development and present a substantial poverty risk. Strategies to combat the adverse diabetes effects need to be tailored to the available resources within countries, target the most affected groups to narrow inequities, also having in mind potential gender differences. Finally, there is a large undiagnosed diabetes population in \acp{MIC} that is likely to experience severe diabetes complications if identified very late. Hence, ways to diagnose this population earlier in order to prevent further deterioration of health may go a long way in preventing and delaying the most catastrophic economic and health outcomes.

In conclusion, it is hoped that this thesis, and the publications born out of it, contribute to the knowledge on the economics of diabetes and help to identify cost-effective strategies to lower the health and economic consequences of diabetes. It has demonstrated the economic burden currently caused by diabetes, in particular in Mexico and China, and has identified groups that are particularly vulnerable to the negative consequences of the disease and should be at the centre of efforts to prevent the burden of diabetes.