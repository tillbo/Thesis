\section{Chapter overview}
Diabetes has reached epidemic proportions in \acp{MIC} and is a major contributing factor to poor health and early mortality, as also discussed in Chapter \ref{cha:intro}. The economic impact of diabetes on individuals and healthcare systems has, however, received limited attention. In particular, we have a limited understanding of the effect of diabetes on individual labour market outcomes. Moreover, little is known about how people with diabetes currently achieve positive change in behaviour risk factors to prevent the disabling complications of diabetes, and whether this plays a role in the effect if the disease on labour market outcomes. The main goal of this thesis has been to assess the economic burden of diabetes in \acp{MIC}, focusing on two predominant and large countries with an increasing diabetes disease burden. This should help to better understand the importance of primary and secondary prevention of diabetes and to identify those populations most susceptible to the adverse economic effects of diabetes.

Four separate studies were conducted that intended to answer the research questions posed in Chapter \ref{cha:intro}. This concluding chapter has four parts. Firstly, it summarises the principal findings. Secondly, it contextualises the findings within the wider literature and provides implications for policies. Thirdly it reflects on the methods. Finally, there are suggestions for future research and concluding comments.

\section{Summary of principal findings}



\paragraph{Chapter \ref{cha:review}} set out to provide an overview of and critically assess existing studies on the economic costs of type 2 diabetes globally. This not only included \ac{COI} studies but also studies on labour market outcomes. Systematic review methods were used and the evidence was synthesized narratively. 86 \ac{COI} studies and 23 labour market studies were identified. Of those, 24 came from \acp{LMIC}, with 23 being \ac{COI} studies.

For \ac{COI} studies, the review found a large range of estimated costs, with the largest per-capita costs being observed in the USA while costs were generally lower in \acp{LMIC}. However, in \acp{LMIC} treatment costs were paid almost entirely out-of-pocket by the poor due to a lack of health insurance coverage, consuming considerable parts of their annual income.  The review also found considerable differences in the methodologies used and in the study quality. This made it difficult to directly compare the results across studies. While in many \acp{HIC} studies an incremental costing approach was used and data sources were representative for a distinct population, studies in developing countries often had to rely on non-representative, relatively small convenience samples, often lacking a control group. Many studies also lacked explicit mentioning of the study perspective or of the costing components that were included. 

For labour market impact studies, most found adverse effects of self-reported diabetes on employment probabilities, wages or working days. Studies were concentrated on a few \ac{HIC}, in particular the USA. More recent studies took into account potential biases due to the endogeneity of diabetes, mainly using an \ac{IV} strategy with the family history of diabetes as an instrument. However, the direction of bias was ambiguous across different studies and countries. 

The review also identified methodological and thematic areas that previous research had only covered sparingly. No \ac{COI} studies took into account the possibility of biased estimates as a result of endogeneity of diabetes. Consequently, there is a lack of evidence in the literature about the potential bias in the cost estimates of diabetes \ac{COI} studies. Further, few studies used an incidence approach to investigate lifetime costs of people with diabetes, which could provide better information about the dynamics of cost increases post diabetes diagnosis. 

Despite these identified limitations of the \ac{COI} literature, the review provided a picture of the healthcare costs of diabetes in almost every continent. This was not the case for labour market studies, where almost no evidence was found for \acp{LMIC}. There is reason to expect the labour market impact of diabetes to be very different in \acp{LMIC} compared to \acp{HIC}, given the \acp{LMIC}' less advanced healthcare systems, later diagnosis but---in some populations---earlier onset of diabetes and greater susceptibility to develop it, the larger informal labour market and overall different labour market structure in \acp{LMIC}. Also, in terms of methodology, studies had not taken advantage of panel data techniques to get closer to a causal interpretation of their estimates. Especially studies on the effect on employment probabilities had instead relied on the same---at least debatable---identification strategy using \acp{IV}. Therefore, a study using a different identification strategy was warranted.

Importantly, no study investigating the impact of undiagnosed diabetes on labour market outcomes was identified by the review. Hence, an important part of the population with diabetes had been mostly neglected. This left open the question in how far results for self-reported diabetes were applicable to the part of the population that was unaware of its diabetes status.

Based on the findings of the review, the three research studies that followed addressed parts of the identified gaps, in particular focusing on labour market outcomes. 

\paragraph{Chapter \ref{cha:Mex1}} provided the first evidence for the impact of diabetes on employment probabilities in a developing country, where diabetes had become a public health concern. Because little was known about the equity impacts of diabetes, a further goal was to investigate the heterogeneity of effects across formal and informal employment and for the 'rich' and 'poor'. Due to the unavailability of an alternative identification strategy, the study applied the already established \ac{IV} approach using parental diabetes as the instrument. Using further background information on parental education, it improved upon earlier studies by controlling for a potential confounding pathway that could have invalidated the specific instrument. It further used two methods to implement the \ac{IV} approach. The main analysis was based on a bivariate probit model that had been shown to be better suited for our specific data, in comparison to a standard linear \ac{IV} model. We nonetheless also provided the results of the latter approach. Both models found no indication of diabetes being exogenous in this context so that a simple univariate probit model was used for inference. The results showed an adverse effect of diabetes on employment probabilities in Mexico, reducing them by about 10 percentage points for men and 5 percentage points for women. The subgroup analysis suggested that the adverse employment effects occurred mainly to those above age 44, while younger people seemed less affected. Also, being poorer increases the exposure to negative employment effects of diabetes. The same was the case for those in the informal compared to those in the formal labour market. Across all models, the point estimates were bigger for males than for females. 

\paragraph{Chapter \ref{cha:Mex2}}  went on to address several questions identified in Chapter \ref{cha:review} that had not been investigated in the first Mexico study. Further, the robustness of the findings of Chapter \ref{cha:Mex1} had to be tested using more extensive and recent data and a different identification strategy. Chapter \ref{cha:Mex2} thereby took advantage of a recent extension to the data used in Chapter \ref{cha:Mex1}. The data now spanned three waves and eight years, which allowed for the use of a longitudinal individual fixed effects model to estimate the relationship of self-reported diabetes with employment. Additionally, the investigated labour market outcomes were extended to wages and working hours. In addition, it was now possible to investigate the relationship of diabetes duration with labour market outcomes, in order to better understand the timing of any diabetes impact on labour market outcomes. Importantly, the additional wave also provided information on diabetes biomarkers to explore the effects of diabetes for the entire population with diabetes as well as those unaware of diabetes separately.

The analysis carried out in Chapter \ref{cha:Mex2} confirmed the adverse relationship of self-reported diabetes with employment, finding a five percentage point reduction for males and females alike. Given the relatively low female employment rate, this translated into a 14\% relative decrease in employment probabilities for women compared to 6\% for men. Compared to the cross-sectional results of Chapter \ref{cha:Mex1}, the estimated effects of the \ac{FE} model are about half the size for men, but are similar and of stronger statistical significance for women. This is likely due to the additional data used in Chapter \ref{cha:Mex2}, but could also partly be the result of the different estimation technique. For wages and working hours no adverse effects of self-reported diabetes were found.

Further analysis showed that the most adverse effects were concentrated among self-employed and independent agricultural workers, potentially due to lower job security and access to healthcare in these often informal jobs. Further, Chapter \ref{cha:Mex2} revealed that the adverse effect of diabetes on employment appeared shortly after diagnosis, then levelled off for some time until it appeared again. This pattern was observed for both males and females, albeit only statistically significant for the former. Interestingly it was found that when the employment effects levelled off, wages started to fall, again for both genders. This suggested that during this period diabetes, plausibly through reductions in productivity, mainly reduced wages, without affecting job loss.

Finally, the results of the biomarker analysis presented in Chapter \ref{cha:Mex2} showed that relying on self-reported diabetes information can lead to measurement bias in the coefficient of diabetes. Using the biomarker data to identify people with diabetes, compared to self-reported diabetes smaller effects especially on employment probabilities were found. This was caused by the non-existent associations between undiagnosed diabetes and employment probabilities. It was further found, that part of the difference in effects between self-reported and undiagnosed diabetes could be explained by differences in subjective health status, with those self-reporting diabetes also reporting a worse health status. Interestingly, differences in \ac{HbA1c} levels did not drive the stronger effects for those self-reporting. %Add ttest to Chapter 4 showing difference in subjective health  


\paragraph{Chapter \ref{cha:China}} continued the investigation of the impact of self-reported diabetes on employment probabilities, but this time in China. It further investigated how a diabetes diagnosis affected diabetes-relevant health behaviours in a developing country. Because the relationships may be biased due to confounders not previously taken into account, the study used two different econometric strategies: \acp{MSM} and \ac{FE}. Each controlled for a different source of confounding, improving the robustness of the identified effects. The dataset used consisted of six waves of the \ac{CHNS}, covering a period from 1997 to 2011.

The results from Chapter \ref{cha:China} provided further evidence of a deterioration of employment probabilities after a diabetes diagnosis, though this time only for women. They experienced a reduction in employment probabilities between 11 to 12 percentage points. For men, the \ac{MSM} and \ac{FE} models showed insignificant relationships. These reductions for women were similar to those found in Mexico (16--17\% in China and 14\% in Mexico) when the different female employment rates were taken into account. The results for behavioural risk factors also suggested different effects for men and women. According to the results, men were able to reduce alcohol consumption, \ac{BMI} levels, waist circumference and their daily calorie consumption, potentially reducing the risk for diabetes complications \parencite{Wilding2014}. For women, no strong evidence for similar reductions was found. A similar picture remained when investigating the effects over time using linear and non-linear specifications. On the one hand they suggested maintained reductions in female employment probabilities over time but no strong changes in risk factors. On the other hand, men were able to consistently reduce behavioural risk factors in the years following diagnosis while not experiencing any labour market penalties. Overall, the findings suggest a potential relationship of changes in risk factors with changes in labour market outcomes.

\section{Implications for policy making}

The findings of this thesis indicate an important global economic burden of diabetes and have added first evidence on the effect of diabetes on labour market outcomes in \acp{MIC}. The thesis also showed that diabetes---at least as far as labour market outcomes are concerned---did not similarly affect the population unaware of their diabetes diagnosis as it did those who were aware. Additionally it showed, that a diabetes diagnosis can elicit positive changes in behavioural risk factors, though to different degrees for men and women. Further, the distributional analysis brought to light that the burden of diabetes appears to be distributed unequally, disproportionally affecting the poor, those in the informal labour market and women.

These findings may lead to several implications to reduce the economic burden of diabetes in \acp{MIC}. 

\subsection{Inequities in the economic burden of diabetes}

An important finding of this thesis are the economic inequities in the burden of diabetes. In Chapter \ref{cha:review} the review found a high \ac{OOP} burden in \acp{LMIC}, especially for those with no insurance coverage. Chapter \ref{cha:Mex1} showed that the adverse employment effects were concentrated among those in the informal labour market and with fewer resources. This was further supported by findings from Chapter \ref{cha:Mex2} that indicated a greater reduction in employment probabilities to work in the agricultural or self-employed sector, while for those working in a non-independent wage job---that often entails greater contractual job security and better access to health insurance---diabetes did not appear to elicit negative effects. Chapter \ref{cha:China} found bigger adverse employment effects and more modestly positive behavioural changes in women compared to men after they had received a diabetes diagnosis. These gender inequities are also supported by the results for Mexico, in particular by Chapter \ref{cha:Mex2}, where, taking into account the lower overall employment rates for women in Mexico, the relative reduction in employment probabilities was much greater for females than for males.

There may be several potential strategies how to reduce these inequalities. In particular these include tackling the observed differences by gender, better prevention of diabetes, earlier diagnosis and better treatment of those diagnosed.

\subsubsection{Gender}

One of the main results of this thesis, is the identification of women with diabetes as a specific target group. Gender differences in the disease burden of diabetes have come to the forefront only recently \parencite{Peters2015}, but may hold one of the keys to reducing the economic burden of diabetes. However, it appears that biological differences between men and women may make it necessary to specifically target women who are likely more affected by diabetes than men \parencite{Peters2015,Peters2014a,Peters2014,Bertram2010} which could be driving the observed differences in the economic effects. Efforts to reduce the burden for females would include increasing awareness among doctors about the higher risks for women to develop diabetes complications, as well as screening for cardiovascular risk factors in women at or before a diabetes diagnosis. This would present an opportunity to prevent a further escalation of the cardiovascular risk profile \parencite{Peters2015}. For women, weight reduction thereby seems to be the single most important step to reduce the risk of diabetes and ensuing complications \parencite{Peters2015}. As this thesis has shown, women in China were not able to achieve weight reduction to the extent men did and therefore may need to be treated differently. Future work on \acp{LMIC} can provide important contributions to help develop effective strategies to obtain this type of improvements in women's health outcomes.

Moreover, reductions in socioeconomic inequities identified in this thesis may also contribute to a reduction in the observed gender differences. If women have fewer economic resources than men, are more likely to work in the informal labour market and less likely to be insured \parencite{Galli2008} and therefore are more adversely affected by diabetes, then interventions targeting the poor and uninsured should specifically help women. Some of these interventions will be discussed below.

\subsubsection{Prevention}

Greater prevention of diabetes could help to reduce the observed inequities and the individual economic burden of diabetes. Given the inequities found in this thesis, such efforts may be particularly worthwhile if they focus on those disproportionally affected by the adverse economic effects of diabetes.

One option is the introduction of national policies to affect food consumption. There is already some real life evidence of such interventions with the goal of reducing obesity in developing countries. In Mexico, a 10\% tax on purchases of sugar-sweetened beverages and 'junk food' has been introduced in 2014. First results suggested a reduction in purchases of these goods after the introduction of the tax, with a steeper decline for those with lower income levels \parencite{Colchero2016,Batis2016}. If these changes in consumption actually lead to a healthier diet and are large enough to cause reductions in obesity and diabetes prevalence has not been evaluated yet and remains to be seen. Other efforts to prevent diabetes in \acp{LMIC} include increasing the awareness of diabetes and how to prevent it via population level campaigns, and  increasing the accessibility to sport courses and fitness equipment to increase physical activity \parencite{Cefalu2016}. 

Another option is the identification of at risk groups and targeting them with interventions to increase physical activity and dietary changes. These have shown promising results across the globe, including in developing countries such as India and China, where interventions have caused long term reductions in the risk of developing diabetes \parencite{Cefalu2016}. For example, for China a randomized controlled trial provided long term lifestyle interventions to reduce the incidence of diabetes and cardiovascular disease as well as to reduce mortality in people at risk of developing type 2 diabetes. Over the active trial period of six years, the diet and exercise intervention reduced the relative risk for diabetes incidence by  over 50\% \parencite{Pan1997}. A more recent evaluation of the long-term impact of the interventions showed that over 20 years after the intervention had ended, the incidence of diabetes was still over 40\% lower in the intervention group. Further, people that had received the intervention spend 3.6 years less with diabetes than those in the control group \parencite{Li2008}. However, the translation of these interventions to real-world settings has been less successful, even in high-income countries \parencite{Wareham2016, Kahn2014}. For example, weight loss has only been a small fraction of the reductions achieved in trials, often likely too little to prevent diabetes. \textcite{Kahn2014} argue that weight loss is notoriously difficult to maintain over a longer period of time, with trials often only capturing initial weight loss, but not the return to previous weight levels over time.  Therefore, prevention efforts based on lifestyle interventions or aiming at weight loss may not yet be translatable into real life, as too little is known about their cost-effectiveness and long-term effects to justify the use of limited resources \parencite{Kahn2014}. There are also questions about the cost-effectiveness of these interventions if scaled to a population level and the problem of finding qualified staff to implement lifestyle interventions at the local level. 

The evidence for pharmacological interventions mainly using metformin also indicates a reduction in the risk of diabetes. However, \textcite{Cefalu2016} mention the potentially large heterogeneity in the benefit of pharmacological interventions across ethnicities. More research on this subject will be needed to find out if successful pharmacological interventions in one ethnicity can be translated to other ethnicities. Nonetheless, \textcite{Cefalu2016} argue that preventive metformin treatment---which has been shown to reduce diabetes incidence in a number of randomized controlled trials---in individuals with a high risk of progressing to diabetes may be the best approach in countries with few economic resources. Low-cost generic versions of metformin exist, are considered essential diabetes medications in almost all \acp{LMIC} \parencite{Bazargani2014}, are effective in preventing or delaying the onset of diabetes, and are safe \parencite{Gomes2013}. They therefore may present a relatively cost-effective intervention that could be applied using existent healthcare infrastructure and pharmacies. It could be especially effective in \acp{MIC}, where the healthcare system infrastructure is much more developed than in \acp{LIC}. Nonetheless, specific targeting of populations most likely to benefit from pharmacological preventative treatment will be needed, as effects of metformin appear to be heterogeneous across age a diabetes risk. Further, pharmacological treatments may also exhibit different effects across populations and ethnicities \parencite{Cefalu2016}. 

The identification of high-risk individuals that could be targeted with the mentioned interventions may pose an additional hurdle to successfully preventing diabetes. Population level screening could be a way to identify people at risk. Screening could also be carried out at the workplace or the community and existing medical records could be used to identify people at an increased risk. Further, there may be possibilities to promote risk self-assessments using online resources through advertising and social media \parencite{Cefalu2016}. However, scientific evidence of the cost-effectiveness and feasibility of screening for high-risk individuals in \acp{LMIC} is non-existent, and if it were to happen may overwhelm health care systems. It also carries the risk of further widening health inequities if the lower income populations were less likely to attend screening efforts \parencite{Wareham2016}.


\subsubsection{Diagnosis}

If prevention is not successful and people have developed diabetes, the earlier diagnosis of diabetes to prevent further complications could be a viable option to reduce the economic burden of diabetes. In Chapter \ref{cha:Mex2}, adverse labour market outcomes were only observed for the self-reporting population with diabetes, suggesting that the adverse impact manifested only after some time of living with the disease and mainly after diagnosis. This is not surprising given the gradual increase in blood glucose as diabetes progresses and the concomitant relatively slow deterioration of health \parencite{Bertram2010}. While earlier detection of diabetes via screening did not yield important improvements in disease outcomes in the Addition-Trial in European \acp{HIC}, this might be markedly different in \acp{MIC}. The large undiagnosed population found in Mexico in Chapter \ref{cha:Mex2} as well as for other \acp{LMIC} in a recent study by \textcite{Beagley2014}, suggests that, compared to \acp{HIC}, in \acp{MIC} more people with diabetes remain undiagnosed for an extended period of time. Therefore, earlier detection may have a greater beneficial effect  \parencite{Choukem2013}, in particular if it can prevent complications from appearing within a person's productive lifespan. 

The results of Chapter \ref{cha:China} indicate that a diagnosis can introduce positive changes in behavioural risk-factors that may be directly related to a reduced economic burden of diabetes, suggesting that diagnosing those currently unaware could have positive effects. Nonetheless, earlier detection would also increase healthcare demands and costs, at least in the short term. Therefore evidence is needed that explores the trade-off between the costs generated by longer treatment periods and a greater number of patients due to an earlier diagnosis and potential reductions in healthcare expenditures and productivity losses as a result of lower complication rates at later stages \parencite{Engelgau2012}. Evidence on the cost-effectiveness of a population-based diabetes screening program was provided by a recent study from Brazil, where over 22 million people over the age of 40 were screened for diabetes, being the first evaluating an actual real-life population-based diabetes screening program in a developing country \parencite{Toscano2015}. It was unclear if the program could be considered good value for the healthcare system, as the cost-effectiveness of the findings depended strongly on the assumptions used about how effective treatment would be in preventing coronary heart disease and stroke. Given the results from this thesis, cost-effectiveness might be greater from a societal perspective if an earlier diagnosis would prevent or decrease losses in productivity and productive lifespan. Of course, early diagnosis may only be reasonable if the healthcare system is sufficiently developed to allow all diagnosed cases access to appropriate treatment options \parencite{Toscano2015,Engelgau2012}. 

Apart from worse health in the population aware of its diabetes, another policy relevant reason for the difference in the observed effects could be the psychological effect of a diabetes diagnosis. Reductions in productivity may be the result of increasing anxiety and depression as a result of becoming aware of the disease and its potential consequences. Further, difficulties in adapting to the treatment regime may cause additional stress. As discussed in Chapter \ref{cha:Mex2}, there is some evidence that becoming aware of the disease leads to reductions in labour income likely due to its psychological effects \parencite{Liu2014}. If this is confirmed by other studies, then strategies to provide better guidance and support at diagnosis and thereafter to reduce the psychological burden of the disease could be worthwhile.



\subsubsection{Treatment}

Earlier diagnosis, however, will only be worthwhile if those diagnosed are able to receive effective diabetes treatment. The adverse labour market effects found for those with self-reported diabetes and the increase in effect size over time after diagnosis, suggest that currently this may not be the case and adverse health and ensuing economic events often cannot be prevented. This may have several reasons. The diagnosis could happen too late to prevent first complications from having developed, making it increasingly difficult to prevent further complications. Another reason could be the sub-optimal treatment of the disease, in particular in the most adversely affected---likely socioeconomically disadvantaged---groups identified in this thesis.

Therefore, an important step to improve outcomes would be the provision of better quality in diabetes treatment, targeting the identified groups and tailoring interventions according to their socioeconomic, physical and personal characteristics \parencite{Cefalu2016}. The existing evidence on diabetes treatment models applicable in very resource constrained settings has recently been reviewed by \textcite{Esterson2014}. While the evidence is still limited, the study provided information on interventions that have had some success in improving diabetes treatment for the poor. Further, it identified common characteristics of these successful interventions: collaboration, education, standardization of guidelines and algorithms, technological innovations, and resource optimization. The authors recommended that initiatives to provide care to underserved populations should be built on collaborations between academic institutions, hospitals, the private sector and other organizations such as local governments. This should help to achieve goals that would otherwise be difficult to reach for one stakeholder alone. Further, programs should aim at providing appropriate education to doctors to increase their ability to successfully treat people with diabetes. For very remote communities \textcite{Esterson2014} suggested the use of peer-support programs, so that few well educated community members or nurses could help their peers with the challenges of diabetes management. Further, a need for standardized guidelines and treatment algorithms was identified as a means for healthcare professionals to improve and maintain their standards of care. Given that mobile phones have already reached even very remote areas and are common in the developing world, interventions based on existent technologies could also improve care and diabetes outcomes. They could facilitate communication between doctors and their patients as well as tracking and controlling diabetes management and outcome measures. Finally, resource optimization to use available and constrained resources more effectively, e.g. by transferring certain responsibilities from doctors to nurses or from healthcare professionals to peers could be an option in very resource constrained settings \parencite{Esterson2014}. Together, the presented strategies could help in reaching and treating poorer parts of the population.

A number of interventions have been implemented in \acp{LMIC} to improve care for people with diabetes. Focusing on China, Mexico and other \acp{MIC}, some of these will be mentioned here. Most of these interventions apply at least one of the recommendations mentioned in the previous paragraph. For Mexico, a recent randomized controlled trial tested the effects of providing better diabetes training to physicians as well as supporting them with nurses trained in diabetes care and peer-support groups \parencite{Contreras2016}. Further, the additional monitoring and support of patients via the use of mobile phone technology was tested in a second intervention group, given the common use of mobile phones in Mexico. First results indicated a significant reduction in \ac{HbA1c} and better diabetes knowledge in both intervention groups compared to standard care, with better, but not statistically significant outcomes, for the group also using mobile phone technology. Other studies investigating the use of mobile phone technology have also shown promising results \parencite{Singh2016}. Two randomized controlled trials investigated ways to improve diabetes outcomes in Costa Rica and China, respectively \parencite{Goldhaber-Fiebert2003a,Sun2008}. In Costa Rica, the application of a community-based nutrition and exercise program led to reductions on weight, fasting glucose and \ac{HbA1c} levels. In Shanghai, China, more extensive diabetes education and the provision of meal plans led to improvements in blood glucose, \ac{HbA1c} levels, blood pressure and waist-to-hip ratios compared to the group receiving standard diabetes eduction.  Unfortunately, so far information about the ultimate value of these interventions in terms of their cost-effectiveness and long term effects is scarce, partly because the investigation is still under way \parencite{Contreras2016} or has not (yet) been evaluated \parencite{Singh2016}.

Further, in \acp{MIC}, the provision of universal health care has been advocated as a means to reduce health inequities by providing everyone with the ability to access healthcare \parencite{Marmot2008}.  Mexico has been one of the countries where the goal of universal health care has been almost accomplished through the introduction of ``Seguro Popular'', which provides those without prior health insurance coverage with social security and access to diabetes treatment options \parencite{Knaul2012,Rivera-Hernandez2016}. However, evidence on the impact of diabetes treatment and outcomes has shown that the availability of this program has only led to very modest improvements, only finding a positive effect on the use of pharmacological therapy. No effects were found on the monitoring of blood glucose or adherence to exercise plans by people with diabetes \parencite{Rivera-Hernandez2016}. A likely reason for this brought up by the authors was that many clinics were not prepared to provide specialized diabetes care and medications, suggesting that barriers to accessing appropriate diabetes care and education still existed. Hence, while public health care provision for those previously uninsured can reduce inequities, such programs need to ensure that their efforts are not sabotaged by the low quality of the offered services.


\subsubsection{The overall disease burden and structural constraints}

The mentioned strategies may be able to reduce the diabetes, however, they mostly focus on diabetes only and do not take into account potential possibilities for the integration of treatment with other diseases common among the poor, nor do these interventions address overall structural problems responsible for the inequities in the burden of diabetes. They therefore tend to represent temporary solutions aiming to address specific needs of people at risk of or living with diabetes under current circumstances, but may not help to substantially reduce the burden of diabetes in the long term if structural constraints existent in most \acp{MIC} are not taken into account. 

One constraint to the successful implementation of above mentioned interventions is the wider disease burden, which may inhibit the healthcare system  from providing effective treatment for diabetes and other chronic diseases. However, integrating diabetes care with the healthcare for other diseases may also present a viable opportunity for healthcare systems in \acp{MIC}.

Health systems in developing countries have been slow to adopt technologies to reduce the burden of communicable diseases, maternal and perinatal conditions as well as nutritional deficiencies \parencite{Gutierrez-delgado2009}. The main reasons for this slow adoption are social and political instability limiting long-term planning, a lack of resources to finance the introduction of health technologies, and a dearth of qualified personnel in the public sector due to a lack of training and the greater attractiveness of the private sector and developed countries \parencite{Gutierrez-delgado2009}. Therefore many \acp{MIC} face a double disease burden with high rates of communicable and non-communicable diseases at the same time \parencite{Gutierrez-delgado2009}. The treatment of \acp{NCD} places additional pressure on health systems that did mainly develop to provide acute care of infectious diseases based on single-visit treatments and are lacking the infrastructure, resources and experience for the treatment of chronic diseases such as diabetes \parencite{Nulu2016}. Policy makers in \acp{MIC} therefore are forced to make decisions about the prioritization of treatments in an effort to use the available resources in a cost-effective as well as equitable manner \parencite{Gutierrez-delgado2009}, potentially limiting a systems ability to provide effective diabetes care.

To improve treatment for diabetes under these circumstances, a greater integration of health services and control efforts for diabetes with the treatment of communicable diseases and other \acp{NCD} could help to exploit synergies and interactions between diseases. One such example presents the known relationship of diabetes with tuberculosis, where diabetes patients have a two- to threefold higher risk to develop tuberculosis. Further, tuberculosis may also complicate glucose management in people with diabetes \parencite{Dooley2009}. Therefore, instead of competing for resources, the detection and treatment of both diseases may be integrated to reduce costs and improve health outcomes \parencite{Marais2013,Remais2013}. Because tuberculosis and other communicable diseases are more common in groups of lower socioeconomic status with less access to high quality care, the double burden with diabetes and the interplay between the diseases has the potential to even further increase the already existing health and social inequities \parencite{Marais2013}.  Similarly, diabetes is often accompanied by other \acp{NCD} that share risk factors with diabetes and are further worsened by high blood glucose levels \parencite{Cheung2012}. In particular hypertension is very prevalent in \acp{MIC}, one of the major causes of mortality \parencite{Mills2016} and often appears together with diabetes \parencite{Cheung2012,Barquera2013}, offering another avenue for treatment integration to improve health outcomes by better using existing resources. Therefore, focusing on ways to take advantage of the synergies presenting themselves in the treatment of communicable and non-communicable diseases could provide a way to reduce the overall disease burden, in particular of more marginalized populations, which could also reduce the existing inequities while limiting the strain on healthcare budgets.

Additionally, studies have consistently shown a relationship of early life health with later life health outcomes, suggesting that bad health and nutritional status early in life could increase the risk to develop diabetes and other diseases later  \parencite{Currie2013,Hanson2012}. Therefore, efforts to improve maternal and early life health outcomes of children will not only have short-term effects but likely help to prevent adverse health outcomes later in life \parencite{Marais2013,Bygbjerg2012}. As a result, investing in the treatment of infectious diseases, nutritional deficiencies and maternal health could help to reduce the overall disease burden now and in the future. Further, because again it is the poor that are likely most exposed to the risk of adverse early life events, such efforts would likely help to reduce the economic inequities found in this thesis.  

However, while a grater integration of diabetes care with the care of other diseases may be a viable way forward, these changes in the formal health-care sectors will not be sufficient. Because of the feedback loops between poverty and bad health, i.e. poor people are more likely to be sick which then further worsens their economic situation, socioeconomic inequities themselves are drivers of the disease burden \parencite{DiCesare2013}. Consequently, structural problems such as an unequal distribution of power, financial resources, education, the environment, housing as well as access to high quality health care,  need to be addressed. Only this will help to achieve lasting reductions in inequalities and consequently also the disease burden due to both communicable and non-communicable diseases \parencite{DiCesare2013}.

\subsubsection{Discrimination of people with diabetes}

Despite the proposed efforts to reduce inequities in the burden of diabetes, people with diabetes may still face discrimination. The thesis has found considerable adverse effects of diabetes on employment chances which may not only be explained by its health impact, but also by employers discriminating against people with the disease. Once employers are aware of the employee's diabetes, they may decide to replace the employee with a healthy person as they suspect reductions in productivity due to health problems or disease management at the workplace. Little information exists regarding the importance of discrimination of employers against people with diabetes in \acp{LMIC}. For the USA, studies show that people with diabetes were more likely to experience discharge, constructive discharge or suspensions affecting their ability to retain their job \parencite{McMahon2005}. Further, working for smaller employers, being older and the ethnic background affected the risk of experiencing discrimination due to diabetes in the workplace. Similarly, a study for Switzerland found that people with diabetes were less likely to be hired and diabetes related events---such as hypoglycemia---made it more likely to experience job loss \parencite{Nebiker-Pedrotti2009}. Even though we have no information about the importance of discrimination for the employment effects found in this thesis, given the evidence from \acp{HIC} it is likely that it plays a considerable role. The adverse effects for the poor and informally employed found in this thesis suggest that discrimination may play a more important role in manual occupations that value physical health to a greater extent than more brain based jobs in the formal sector. Additionally, informal jobs are not affected by job security legislation \parencite{Ulyssea2010,Loayza2011}, reducing the costs of hiring and training a new employee, making it easier to replace a 'unhealthy' with a 'healthy' employee, further incentivising discrimination against people with diabetes.

Unfortunately, simple remedies for this type of discrimination are difficult in \acp{MIC}. Because informal labour markets are a substantial part of transition economies, legislative measures to reduce the incentives of discriminating against people with diabetes may fall short---at least partly---as they would not be enforceable in the informal sector. Further, stricter protection legislation may have counterproductive effects in middle-income countries if they lead to reduced hirings of people with diabetes or those at a higher risk to develop diabetes, such as overweight or obese candidates \parencite{Muravyev2014}. Companies may be inclined to demand health check-ups prior to hiring to prevent the employment of personal with a higher risk of adverse health outcomes. Therefore measures to reduce discriminatory behaviour in employers in \acp{MIC} should also aim at reducing prejudices about people with diabetes, increase the knowledge about the treatment of the condition and the potential to prevent its adverse health consequences.


Overall it seems that for \acp{MIC}, national policies to change food consumption behaviours to prevent diabetes could currently be the best option to halt the escalation of the economic impact of diabetes and to reduce inequities. The results of this thesis suggest that it should be a priority to design interventions that address the existent inequities by preventing diabetes in those populations that experience the worst economic consequences, i.e. the poor and more marginalised groups of a country. One way to reduce the existing inequities using the existing health care system would be the integration of the treatment of diabetes with already existing strategies to treat related communicable diseases, common among underserved populations. This would also reduce competition for resources to treat different diseases, a problem facing many decision makers in very resource constrained healthcare systems. The evidence base for the effectiveness of screening programs, preventative pharmacological treatment and lifestyle interventions is less conclusive, potentially due to the social and economic structural constraints existent in many \acp{MIC}, preventing their successful implementation. Therefore, the structural problems underlying the already existing social, economic as well as health inequities will need to be addressed to achieve long term reductions in the burden of diabetes. This also pertains to issues of discrimination of people with diabetes at the workplace, currently being mostly unprotected from such behaviour due to the large informal labour markets in \acp{MIC}.



\section{Strengths and limitations}

The strengths and limitations of each study and the methodological approach used have been evaluated within each chapter. Additionally, the thesis overall has strengths and limitations.


A strength of this thesis is the provision of a comprehensive overview and assessment of the state of economic research on the impact of diabetes. It provides other researchers guidance by identifying areas for future research and suggestions on which methods to use. Further, the thesis itself fills some of the identified gaps by investigating the impact of diabetes on labour market outcomes in \acp{MIC}. A strength of these analyses is the use of rigorous econometric approaches taking advantage of available and previously underexplored, high quality, household data, allowing to investigate a variety of topics in the absence of experimental data. One of the challenges was the choice of the most appropriate method to establish a causal relationship. The main concern was that unobserved variables, measurement error as well as reverse causality may introduce bias into the estimates. A variety of methods were used that each had advantages and disadvantages in terms of the underlying assumptions and the ability to account for potential sources of bias. Their choice was mainly guided by the available data and the best way of achieving a causal interpretation under the given circumstances. Nonetheless, regardless of the method used, results consistently showed an adverse relationship of self-reported diabetes with employment probabilities, suggesting a relatively robust and likely causal effect. The methods used also improved upon previous approaches, providing more robust evidence and also incorporated methods predominantly known in epidemiology. 

A further strength is the provision of evidence on the potential of diabetes to widen the economic inequities in developing countries, identifying the groups that were disproportionally affected by the disease. Further, it has also advanced the understanding of diabetes as a multifaceted condition by exploring effects over time and for those who are aware and those who are unaware of their diabetes. Finally, it provides evidence from different data sources and contexts and also investigates the value of becoming aware of the disease through a diagnosis and its ability to influence health behaviours.

The thesis has several limitations. Whilst the intention was to provide evidence on the economics of diabetes in \acp{MIC}, the thesis mostly investigates the economic impact of diabetes. While this provided important information for researchers and policy makers, the thesis did not investigate how to curb this economic diabetes burden. Information about the best and most costs-effective interventions that could be applied in \acp{MIC} to lower the burden of diabetes is urgently needed as information about who is affected most will not suffice to effectively reduce the burden. Research on how to implement interventions feasible in non-\ac{HIC} settings is therefore of paramount importance, but was beyond the scope of this thesis.

This leads to the next limitation. The thesis does not investigate in how far healthcare systems in \acp{MIC} need to change in order to better provide care. Because they often lack financial resources, do not efficiently use the available resources, are designed to treat acute infectious diseases rather then affecting the outcomes of long-lasting non-acute \acp{NCD}, and often provide unequal access to their health services due to financial constraints of those seeking care, research into how to better equip healthcare systems to confront the challenges of treating \acp{NCD} is urgently needed \parencite{Mills2014,Guzman2010}.

A further limitation is the geographical concentration of the thesis, as far as the empirical, analytical chapters are concerned. While Mexico and China are among the ten countries worldwide regarding the absolute size of their population with diabetes, there are other large and small \acp{MIC} currently facing similar challenges \parencite{Risk2016}.  It cannot be assumed that the evidence provided in this thesis is perfectly representative of all other \acp{MIC}. There is hence a need to investigate the economic burden and potential solutions in other countries, given their own specific context in terms of culture, the political system, economic development and existing inequities.

Finally, while the thesis intended to provide a picture of the potential inequities in the economic impact of diabetes for socioeconomic subgroups, it did not investigate in detail why these inequities exist and could only speculate on the reasons. A better understanding of the underlying reasons will be essential for designing adequate strategies to address these inequities. Further, whilst the thesis has touched upon the potential reasons for the differences in employment effects between those self-reporting diabetes and those unaware, it has not provided an in depth analysis of this phenomenon. A better identification of the underlying reasons will be required to design interventions that can prevent the adverse economic effects of diabetes. 



\section{Suggestions for future research}

This thesis has shown the global economic impact of diabetes and its adverse effect on labour market outcomes in Mexico and China. It identified the poor, those in the informal economy and women as being most adversely affected by the disease. It further found that, at least in China, it is men that appear to make the most from a diabetes diagnosis in terms of positively changing their health behaviours. Finally, it provided some indication that while self-reported diabetes is related to adverse labour market effects, undiagnosed diabetes is not. Without a greater understanding of the underlying reasons for the differences found, it will be difficult to design policies that can help prevent the burden of diabetes in \ac{MIC} and reduce inequalities.

Several reasons for the observed gender differences in the impact of diabetes have been discussed in this thesis, including biological reasons that increase the risk of complications in women \parencite{Peters2014,Peters2015,Arnetz2014,Roche2013,Policardo2014,Catalan2015,Engelmann2016,Seghieri2015} and may also impair the ability of women to lose weight \parencite{Penno2013}, as well as differences in the access to appropriate healthcare \parencite{Penno2013}. One strategy to further investigate these differences would be the use of biomarker data in combination with information on healthcare utilization as well as socioeconomic outcomes. This could then be used to investigate potential heterogeneities in the relationship between diabetes and overall metabolic health with labour market outcomes. Further, information on healthcare usage could be used to investigate if differences in healthcare access mediate the economic impact of diabetes. A potentially rich source of information is provided by two Chinese household surveys, the \acf{CHNS} and the \acf{CHARLS}. Both contain an extensive list of measured biomarkers and socioeconomic variables that could help to investigate gender differences in metabolic risk. Because biomarker data were only available for one wave in both the \ac{CHNS} and \ac{CHARLS}, in the present studies they could not be used longitudinally to predict future effects of diabetes. However, they will be able to be used for this purpose in future waves. This information may also be used to further explore differences in metabolic risk between people aware and unaware of their diabetes. Also, studies measuring potential mediating variables---such as knowledge, motivation, treatment, diabetes control and complications---would help clarify the causal mechanisms through which diabetes affects economic and other outcomes. Structural equation and mediation models could be useful with such data. 

Researchers should also try to confirm the results regarding the inequities found, using different data and countries. Whether these relationships can be confirmed or not, the underlying drivers of these inequities need to be explored to design adequate policies. This could be done by identifying countries where these inequities may not have been found, to isolate the causal determinants. Further, strategies implemented currently or in the future in \acp{MIC} that aim at reducing these inequities, such as the implementation of universal health insurance schemes need to be evaluated in how far they are actually achieving this goal in terms of diabetes. The same is true for population level interventions such as taxes on foods or nutrients, as these are probably regressive and theoretically should reduce consumption in particular for those with lower levels of income \parencite{Mytton2012c}. This could then lead to a reduction in diabetes incidence in these groups. However, depending on the price elasticities of the taxed products, such taxes may only reduce the disposable income of the poor, leading to reductions in the consumption of other, potentially healthier foods. They therefore may be seen as taxes on the poor, raising political and ethical dilemmas. Further, substitution effects with equally untaxed products may only cause a shift in consumption towards other equally unhealthy, but untaxed products \parencite{Mytton2012c}.

The population with diabetes in all countries, but especially in \acp{LMIC} is only partially observed. In other words many people with diabetes are not aware that they have the disease. This thesis has provided an investigation of the differences between those who are aware and unaware in Chapter \ref{cha:Mex2}. It, however, still remains unclear to what extent different factors such as health information and actual health status are causing the observed heterogeneity in the economic impact. Because increasingly household surveys are providing biomarker data in combination with socioeconomic information, they should be used together with quasi-experimental econometric techniques to investigate this topic. A regression-discontinuity design may be used in a similar vein as in \textcite{Zhao2013a}, who use cut-off values for hypertension to identify those newly diagnosed and the subsequent effect of this diagnosis on health behaviours. A similar approach could be used to explore the effects of a diabetes diagnosis and the entailed health information on labour market outcomes, health behaviours and other economic outcomes. Importantly, research should assess the heterogeneity of effects across income groups, rural versus urban, education levels and between males and females. This would provide important information for designing interventions to reduce the physiological and economic burden of diabetes while preventing a widening of inequities.

Finally, there is a need to explore further economic downstream effects of the economic impact of diabetes. If diabetes causes reductions in employment and potentially also income, it is likely that these will cause not only problems for the individual directly affected, but for the entire household as well. In \acp{MIC}, where social security is less extensive and comprehensive, adverse health shocks due to diabetes could have consequences for the children, spouses or other family members living in affected households \parencite{Alam2014}. The loss in labour income due to diabetes needs to be compensated either by increasing the labour supply of other household members or by reducing expenditures for other consumption goods. Both could affect children directly, for example by reducing the time for or quality of education when tuition fees cannot be paid any more and also by having to substitute time for education with labour time. Similarly spouses may be forced to increase their labour supply, reducing the time they can care for their children. These effects have remained unexplored for diabetes but given the scale of the diabetes epidemic may not be trivial.



\section{Conclusion}

Diabetes presents a major challenge for \acp{MIC}, but evidence on its economic effects has been scarce. This thesis has found that diabetes has an adverse economic impact on individuals and puts a burden on healthcare systems. Because evidence on the impact of diabetes on labour market outcomes was lacking in developing countries, the thesis did focus particularly on this topic. Thereby it not only provided evidence of the adverse impact of diabetes on employment, but also improved upon previously used econometric methods by using novel strategies to identify a causal relationship. The thesis also identified potential inequities in the impact of diabetes, pointing to larger adverse effects for the poor, those in the informal labour market and women. But the thesis did not only focus on the economic impact of diabetes, but also investigated the effects of a diabetes diagnosis on health behaviours, unravelling evidence for differences in the ability to change health behaviours between men and women.

These findings suggest that there is a need to reduce the economic impact of diabetes in \acp{MIC}. Considering the increasingly earlier onset of diabetes and the ongoing increase in incidence in many countries, the non-trivial adverse economic effects could otherwise hinder economic development and present a substantial poverty risk. Strategies to combat the adverse diabetes effects need to be tailored to the available resources within countries, target the most affected groups to narrow inequities, also having in mind potential gender differences. Finally, there is a large undiagnosed population with diabetes in \acp{MIC} that is likely to experience severe diabetes complications if identified very late. Hence, ways to diagnose this population earlier in order to prevent further deterioration of health may go a long way in preventing and delaying the most catastrophic economic and health outcomes.

In conclusion, it is hoped that the research presented in this thesis contributes to the knowledge on the economics of diabetes and help to identify cost-effective strategies to lower the health and economic consequences of diabetes. It has demonstrated the economic burden currently caused by diabetes, in particular in Mexico and China, and has identified groups that are particularly vulnerable to the negative consequences of the disease and should be at the centre of efforts to prevent the burden of diabetes.