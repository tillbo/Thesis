\section{Chapter overview}
As discussed in Chapter \ref{cha:intro}, diabetes has reached epidemic proportions in \acp{MIC} and is a major contributing factor to disabling poor health and early mortality. The economic impact of diabetes on individuals and healthcare systems has, however, received relatively little attention. Further, little is known about how successful healthcare systems currently are in encouraging behaviour change in those diagnosed to prevent the disabling complications of diabetes. This is despite the fact, that until now efforts to halt the increase in disease prevalence have been of little success. Consequently, the goal of this thesis was assessing the economic burden of diabetes in \acp{MIC}, which by now are home to the majority of people with diabetes worldwide. This should to better understand the importance of primary and secondary prevention of diabetes and identify those populations must susceptible to the adverse economic effects of diabetes.

To meet these aims, four separate studies were conducted around the following questions:
\begin{enumerate}
\item What is the current evidence on the economic costs of type 2 diabetes?
\item What is the causal effect of self-reported diabetes on labour market outcomes?
\item In how far can results gained from self-reported diabetes data be used to characterize the entire diabetes population?
\item What is the current value of health information via a diabetes diagnosis in terms of affecting health behaviours in a \ac{MIC}?
\end{enumerate}

This concluding chapter has four parts. Firstly it summarises the principal findings. Secondly, it contextualises the findings within the wider literature with implications for practice. Thirdly it reflects on the methods. Finally, there are suggestions for future research and concluding comments.

\section{Summary of principal findings}

Chapter \ref{cha:review} set out to provide an overview of and critically assess existing studies on the economic costs of type 2 diabetes globally. This not only included so called \ac{COI} studies but also studies on labour market outcomes. Systematic review methods were used and the evidence was synthesized narratively. 86 \ac{COI} studies and 23 labour market studies were identified. Of those, 24 came from \acp{LMIC}, with 23 being \ac{COI} studies.

For \ac{COI} studies, the review found a large range of estimated costs, with the largest per-capita costs being generated in the USA while costs were generally lower in \acp{LMIC}. However, it also found that the direct relative economic burden caused by the treatment of type 2 diabetes is much higher in poorer countries, in particular for the poorest parts of the population. To pay for treatment, the poor have to pay almost entirely out-of-pocket due to a lack of insurance, with considerable parts of the annual income being spent on these payments. The difference in costing approaches used across studies and the varying quality of data sources made it difficult to directly compare the studies. While in many \acp{HIC} studies some sort if incremental costing approach was used and data sources were representative for a distinct population, studies in developing countries often had to rely on data without a control group and collected non-randomly. Also, studies from low-income countries in particular had much less observations. Many studies also still lacked explicit mentioning of the used study perspective or the included costing components. 

For labour market impact studies, most found adverse effects on employment probabilities, wages or working days, suggesting an adverse effect of diabetes on these outcomes. Studies were concentrated to a few \ac{HIC}, in particular the USA. More recent studies took into account potential biases in the case of endogeneity of diabetes, mainly using an \ac{IV} strategy with the family history of diabetes as an instrument. If a bias was found, its direction was ambiguous across different studies and countries. 

The review also identified areas for which little to no studies had been found. For \ac{COI} studies, none of the studies took into account the possible of biased estimates as a result of endogeneity of diabetes. This may have led to over- or underestimations in the studies reviewed here. One potential source of bias could be accidents that have led to diabetes by restricting physical activity and at the same time also caused to higher healthcare expenditures. Further, few studies used an incidence approach to investigate lifetime costs of people with diabetes, providing better information about the dynamics of cost increases after a diabetes diagnosis. 

Despite these identified limitations of the \ac{COI} literature, they at least provides a picture of the healthcare costs of diabetes in almost every continent. This was not the case for labour market studies, where almost no evidence was found for \acp{LMIC}. Arguably, given the less advanced healthcare systems, later diagnosis but earlier onset of diabetes, the larger informal labour market and overall different labour market structure in developing countries, the impact of diabetes may be very different compared to \acp{HIC}. Also in terms of methodology, studies had not taken advantage of panel data techniques to achieve a causal interpretation of their estimates. Especially studies on the effect on employment probabilities had relied on the same identification strategy using \acp{IV} where it is at least debatable if the underlying assumptions are valid. This reliance may expose the field to wrong inference if the \ac{IV} were invalid. Only one study used panel data, but did not specifically account for the panel structure and only used pooled regression techniques. Therefore, a study using a different identification strategy is warranted taking advantage of the available panel data.

Importantly, also information about the impact of diabetes on the undiagnosed population---comprised of people with diabetes that have remained unaware of their disease---was not identified by the review. This neglected an important part of the diabetes population, especially in \acp{LMIC}.  Only one study identified by the review used biomarker information but did not specifically investigate the undiagnosed population, warranting further research.

Based on these findings, the three research studies following the review addressed the identified gaps for labour market studies. The aim of Chapter \ref{cha:Mex1} was to provide first evidence for the impact of diabetes on employment probabilities in a developing country where diabetes has become a recognized public health problem. Because little was known about the equity impacts of diabetes, a further goal was to investigate the heterogeneity of adverse effects for those in formal and informal employment and for the "rich" and "poor". Due to the lack of other identification strategies, this study also used parental diabetes as an instrument. However, using further familial background information on parental education, it improved upon earlier studies by controlling for a potential confounding pathway that could invalidate the used instrument. It further used two methods to implement the \ac{IV} approach. The preferred estimates came from a bivariate probit model that had been shown to be better suited for our specific data in comparison to a standard linear \ac{IV} model. We nonetheless also showed the results of the latter approach.

Chapter \ref{cha:Mex1} found evidence for an adverse effect of diabetes on employment chances, reducing them by about 10 percentage points for men and 5 percentage points for women. Further, no indication of the endogeneity of diabetes was found, suggesting the use of a simple probit model. The subgroup analysis suggested that the adverse employment effects occurred mainly to those above age 44, while younger people seemed less affected. Also, being poorer appeared to increase exposure to negative employment effects of diabetes. Similar was the case for those in the informal labour market. Across all models, the effects were more pronounced for males than for females. 

While these results provided good evidence for an adverse effect of diabetes on employment chances in a developing country, several questions still remained. Further, the robustness of the findings of Chapter \ref{cha:Mex1} had to be tested using more extensive and recent data and a different identification strategy. Chapter \ref{cha:Mex2} addressed these issues using a newly released addition to the data used in Chapter \ref{cha:Mex1}. The data now spanned three waves and eight years which allowed for the use of a longitudinal individual fixed effects model to estimate the relationship of self-reported diabetes with not only employment. Additionally the outcomes of interest were extended to wages and working hours. Further it was now possible to investigate the relationship of diabetes duration with labour outcomes, adding to the understanding of when people with diabetes experience adverse labour market outcomes. Importantly, the additional wave also provided information on diabetes biomarkers in order to investigate the effects of diabetes for the entire diabetes population and those unaware in particular.

The analysis carried out in Chapter \ref{cha:Mex2} confirmed the adverse relationship of self-reported diabetes with employment, finding a five percentage point reduction for males and females. Given the relatively low female employment rate for females, this translates into a 16\% decrease in female employment probabilities compared to 6\% for men, taken the respective average employment rates as the baseline. Compared to the cross-sectional results of Chapter \ref{cha:Mex1}, the estimated effects of the \ac{FE} model are about half the size for men, but are marginally bigger for women. This is likely due to the additional data used in Chapter \ref{cha:Mex2}, but could also partly be the result of the different estimation technique. For wages and working hours the results did not show an adverse effect of self-reported diabetes, suggesting that having a diabetes diagnosis does not lead to important reductions in productivity, but rather to a sudden inability to continue working. This could be caused by the appearance of very debilitating complications. Further analysis showed that in professions mainly in the informal sector, such as being self-employed or a farmer, self-reported diabetes had the greatest adverse impact. Another reason for the found effects may also have been selection into certain professions of people with diabetes. Further, Chapter \ref{cha:Mex2} revealed that the adverse effect of diabetes on employment appeared shortly after diagnosis, then levelled off for some time until it appeared again. This pattern was observed for both males and females, albeit only statistically significant for the former. Interestingly, contrary to the standard analysis using the binary diabetes indicator an adverse effect on wages was found for males and females. Further, this effect was mostly limited to 5--11 years after diagnosis, exactly the time where no employment effects were found. This downward adjustment in wages may therefore be tentatively interpreted as a result of lower productivity due to diabetes. The reduction in productivity, however, is not so strong as to justify job loss. However, due to the quite imprecise measurement of these wage effects, such an interpretation remains highly speculative.

Finally, the results of the biomarker analysis presented in Chapter \ref{cha:Mex2} provided evidence that relying on self-reported diabetes information leads to measurement bias if the effects are interpreted as representative for the entire diabetes population. Using the biomarker data to analysis found much smaller effects especially on employment probabilities, trending towards zero for males. This was caused by the non-existent associations between undiagnosed diabetes and employment chances. It was further found, that part of the difference in effects between self-reported and undiagnosed diabetes can be explained by differences in subjective health status, with those self-reporting diabetes reporting a significantly worse health status. Interestingly, differences in \ac{HbA1c} levels did not appear to be driving the stronger effects for those self-reporting. These two results leads to the following tentative explanation for the differences in results. Those how self-report diabetes are aware of the disease because they have been diagnosed. Often a diagnosis only happens after many years of having diabetes and may only be a result of first diabetes related complications manifesting. Therefore, there is considerable selection of people with deteriorating health, as a result of diabetes, into the self-reporting population. This then also, at least partly, explains the reduction in employment probabilities in this group. The unaware population, however, has remained unaware of the disease because it is still largely asymptomatic and hence not prompting a diagnosis nor reductions in productivity.  %Add ttest to Chapter 4 showing difference in subjective health  

Chapters \ref{cha:Mex1} and \ref{cha:Mex2} provided of the adverse effect of self-reported diabetes on labour market outcomes in Mexico. Chapter \ref{cha:China} continued the investigation of the impact of self-reported diabetes on employment probabilities, but this time on China. It further extended its scope to investigate how a diabetes diagnosis affects diabetes relevant health behaviours in a developing country. These health behaviours were smoking, alcohol consumption, anthropometrically measured \ac{BMI} and waist circumference, and daily calorie consumption. Because identification of a causal relationship may be confounded, the study used two different econometric strategies in \aclp{MSM} and \ac{FE}. Each controlled for a different source of confounding, improving the robustness of the identified effects. The used dataset consisted of six waves from the \acl{CHNS}, covering a period from 1997 to 2011.

The results provided further evidence of a deterioration of employment probabilities after a diabetes diagnosis, though only for women. They experienced a reduction in employment chances between 8 to 11 percentage points. For men, the \ac{FE} and \ac{MSM} showed insignificant relationships. The results of for health behaviours also suggested different effects for men and women. First of all, the descriptive results showed that smoking and alcohol consumption is much more prevalent in men than it is in females. For the latter, these risk factors were almost non-existent. All results indicated that men did not reduce smoking but alcohol consumption after a diabetes diagnosis. They further reduced their \ac{BMI} and waist circumference and calorie consumption. These reductions were small in size but might be important at a population level, given the number of people with diabetes in China. For women, no strong evidence for similar reductions was found. A similar picture remained when investigating the effects over time using linear and non-linear specifications. They suggested maintained reductions in female employment probabilities over time. Men were able to reduce their \ac{BMI} and waist circumference consistently in the years following diagnosis. No strong evidence for a rebound effect, where weight measurements would go up after an initial reduction, was found. 

These results suggest very different effects of a diabetes diagnosis for men and women. On the one hand women were unable to reduce their risk factors, but men were. On the other hand, women were those that had to bear stronger adverse economic effects. Several issues may be important to explaining this difference. First, they may be a result of a different access to healthcare resources between men and women due to difference in income, with women not receiving appropriate treatment that would have allowed them to prevent complications. Second, women may work in less protected jobs were they are easily replaceable, making it difficult to take time off for medical check ups or appropriate treatment out of fear of job loss. Unfortunately only little research explores gender differences in healthcare access in China. One study by \textcite{Fan2013} finds that female migrant workers in particular face barriers to access healthcare. Further, there relatively lower education levels compared to men may limited their ability to efficiently put the information received at diagnosis into practice, though quasi-experimental evidence that would support a causal effect of education on health behaviours in China is yet inconclusive \parencite{Xie2014a}. Whatever the true reason for this difference may be, it appears that women in China are much worse affected by diabetes than men.

\section{The context of the findings and their implications}

The findings of this thesis indicate an important economic burden of diagnosed diabetes as measured by diabetes self-reports in the \acp{MIC} of Mexico and China. Chapter \ref{cha:review} further found many studies that suggested a large burden in terms of healthcare costs in both, low- and middle-income countries. The thesis has also provided evidence that diabetes---at least in the case of labour market outcomes---does not similarly affect the unaware diabetes population. This differences is likely explained by two main factors: 

\begin{enumerate}
\item Worse health in the diagnosed population compared to the undiagnosed population.
\item Differences in health information as a result of a diabetes diagnosis by a doctor.
\end{enumerate}

Both factors are likely captured by self-reported health, which not only depends on the actual physical health status but also expresses a belief about the own health status that is influenced by the awareness of the own disease status \parencite{Jylha2009}, here awareness of diabetes. Because self-reported health is worse in those diagnosed, it is likely that they are in an actual worse physical health state. Time since onset in the self-reporting diabetes population is almost surely longer than in the undiagnosed group as onset is often several years prior to diagnosis. Because severe diabetes complications take several years to develop after the onset of diabetes, they are more likely to be prevalent in the self-reporting population. Also because the occurrence of a  complication may had been the reason for a doctor visit and the diabetes diagnosis in the first place. The additional health information as a result of the diagnosis could affect labour market outcomes in two ways. The results of Chapter \ref{cha:China} suggest, that it could lead to improved health behaviours, potentially reducing the health burden of diabetes. However, it could also worsen health and consequently adversely affect labour market outcomes by increasing anxiety or depression. So did a study on China find a reduction in labour income as a result of the additional health information received at diagnosis \parencite{Liu2014}. 

These findings may lead to several implications in order to reducing the economic burden of diabetes in \acp{MIC}. First, the findings from China suggest that a diagnosis of diabetes can be positive as it has the potential to lead to a reduction on risk behaviours. While the found effects were small in size, at least for the weight related measures, on a population level they may lead to significant reductions in the risk of complications.

Further, the finding that adverse labour market outcomes were only observed for the diagnosed population suggest that these only occur after some time of living with the disease. They further indicate that many people with a diagnosis are not able to prevent debilitating complications to occur during their productive lifespan. One reason may be that diagnosis happens to late to prevent complications for an extended period of time. This is also what is suggested by the large undiagnosed population found for Mexico in Chapter \ref{cha:Mex2} as well as for other \acp{LMIC} in a recent study by \textcite{Beagley2014}. Therefore, efforts to achieve earlier diagnoses of diabetes in countries with a large undiagnosed diabetes population may well be worthwhile \parencite{Engelgau2012}. Even though this will increase healthcare demands and costs in the short term, such effects may be set off by increases in productivity and productive years in the working population with diabetes, as well smaller inpatient expenditures due to reduced rates of severe, cost-intensive complications. Evidence on the cost-effectiveness of a population-based diabetes screening program provided a recent study from Brazil, where over 22 million people over the age of 40 were screened for diabetes \parencite{Toscano2015}. This study is the first study providing cost-effectiveness estimates based on an actual population-based diabetes screening program. Using a Markov model they investigate the long term cost-effectiveness of this program from a public healthcare system perspective. The findings are inconclusive as to whether this intervention could be cost-effective at conventional thresholds. It depends strongly on the used assumptions about the ability of first line treatments to prevent coronary heart disease and stroke. Further, the societal perspective was not considered in this study, from which the cost-effectiveness of screening may be greater if an earlier diagnosis leads to increases in productivity and a longer productive lifespan. Of course, early diagnosis may only be reasonable if the healthcare system is sufficiently developed to allow all diagnosed cases access to appropriate treatment options. 

A further implication of this thesis are the found inequities. They manifested in particular in the economic burden of diabetes being disproportionately large for the poor and generally less protected against negative health shocks. In Chapter \ref{cha:review} the studies reviewed suggested a high \ac{OOP} burden in \acp{LMIC}, especially for those with no insurance coverage. Further, the results of Chapter \ref{cha:Mex1} showed that the adverse employment effects were concentrated among those in the informal labour market and with fewer resources. This was further supported by findings from Chapter \ref{cha:Mex2} that indicated a greater reduction in employment probabilities to work in the agricultural or self-employed sector, while for those working in a non-independent wage job---that often entail greater contractual job security and better access to health insurance---diabetes did not appear to elicit negative effects. Finally, the results for China from Chapter \ref{cha:China} showed a much stronger adverse employment effect for females than for males. Also in Mexico the relative reduction in employment chances was much greater for females than for males when the generally lower employment rates for females are taken into account. All this suggest that women are disproportionately affected by diabetes. Finally the results from China also suggested that women are less likely to achieve positive and sustained changes in health behaviours.

These findings can be placed in the context of a larger literature on inequalities in \acp{NCD}, suggesting less access to care in \acp{LMIC} and, in the case of diabetes, especially for women in \acp{LMIC} \parencite{DiCesare2013}. There are several proposed strategies how to reduce these inequalities and improve access to care \parencite{Jacobs2012}. Several of these will be presented here with a focus on the identified populations in this thesis. First, there is a need to reduce access to care in particular for women. One of the components likely hindering women to appropriately  access and use healthcare are their lower educational levels \parencite{Jacobs2012}. These may ultimately limit their ability to effectively use the information received at diagnosis, but also reduces their income levels making it more difficult to access appropriate treatment in the first place. Accordingly, improving female access to education may already go a long way in improving their later life health outcomes. Further, particularly in China, women had been exposed to discrimination in early childhood that may have reduced their education attainment. Even today there still appears to be some gender inequality in education in China, even though it is narrowing \parencite{Zeng2014}...


see Jacobs, B., Ir, P., Bigdeli, M., Annear, P.L., Van Damme, W., 2012. Addressing access barriers to health services: An analytical framework for selectingappropriate interventions in low-income Asian countries. Health Policy Plan. 27, 288–300. doi:10.1093/heapol/czr038

Ali, M.K., Narayan, K.M.V., 2016. Screening for Dysglycemia: Connecting Supply and Demand to Slow Growth in Diabetes Incidence. PLOS Med. 13, e1002084. doi:10.1371/journal.pmed.1002084

M. Silink, J. Tuomilehto, J. C. Mbanya, K. M. Venkat Narayan, Judy Fradkin, G.R., 2010. Research priorities: Prevention and Control of Diabetes with A Focus on Low and Middle Income Countries. WHO Meet. Dev. A Prioritized Res. Agenda Dev. Prev. Control Noncommunicable Dis. 6.

 While  In particular, in China, discrimination against women The literature on the access to healthcare in \acp{LMIC} provides various indications on how to improve equity in health and healthcare access.  However, a universal approach does not exist that is guaranteed to work everywhere.  Potential policy implications from these finding could be a greater provision of insurance coverage to the informal sector in developing countries to allow earlier diagnosis. Specifically targeting the poor in efforts to inform about diabetes to achieve earlier diagnosis and improve healthcare access for them. In China, it has been shown that women have been systematically discriminated against and may have experienced worse early life provision of nutrients etc, today predisposing them to develop diabetes that is more severe. CITE STUDIES SHOWING EFFECT OF FEMALE DISCRIMINATION ON LATER LIFE HEALTH OUTCOMES. 

Show greater literature context findings are placed in...Health and women and equity issue for poor.

\section{Suggestions for future research}

*further exploration of difference for males and females
*




 Further  equity reasons to serve better poor populations. all studies show that they are those most affected (women in china, poor in Mexico). earlier diagnosis, universal health 









