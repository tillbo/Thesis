\section{Chapter overview}
As discussed in Chapter \ref{cha:intro}, diabetes has reached epidemic proportions in \acp{MIC} and is a major contributing factor to disabling poor health and early mortality. The economic impact of diabetes on individuals and healthcare systems has, however, received relatively little attention. Further, little is known about how successful healthcare systems currently are in encouraging behaviour change in those diagnosed to prevent the disabling complications of diabetes. This is despite the fact, that until now efforts to halt the increase in disease prevalence have been of little success. Consequently, the goal of this thesis was assessing the economic burden of diabetes in \acp{MIC}, which by now are home to the majority of people with diabetes worldwide. This should to better understand the importance of primary and secondary prevention of diabetes and identify those populations must susceptible to the adverse economic effects of diabetes.

To meet these aims, four separate studies were conducted around the following questions:
\begin{enumerate}
\item What is the current evidence on the economic costs of type 2 diabetes?
\item What is the causal effect of self-reported diabetes on labour market outcomes?
\item In how far can results gained from self-reported diabetes data be used to characterize the entire diabetes population?
\item What is the current value of health information via a diabetes diagnosis in terms of affecting health behaviours in a \ac{MIC}?
\end{enumerate}

This concluding chapter has four parts. Firstly it summarises the principal findings. Secondly, it contextualises the findings within the wider literature with implications for practice. Thirdly it reflects on the methods. Finally, there are suggestions for future research and concluding comments.

\section{Summary of principal findings}

Chapter \ref{cha:review} set out to provide an overview of and critically assess existing studies on the economic costs of type 2 diabetes globally. This not only included so called \ac{COI} studies but also studies on labour market outcomes. Systematic review methods were used and the evidence was synthesized narratively. 86 \ac{COI} studies and 23 labour market studies were identified. Of those, 24 came from \acp{LMIC}, with 23 being \ac{COI} studies.

For \ac{COI} studies, the review found a large range of estimated costs, with the largest per-capita costs being generated in the USA while costs were generally lower in \acp{LMIC}. However, it also found that the direct relative economic burden caused by the treatment of type 2 diabetes is much higher in poorer countries, in particular for the poorest parts of the population. To pay for treatment, the poor have to pay almost entirely out-of-pocket due to a lack of insurance, with considerable parts of the annual income being spent on these payments. The difference in costing approaches used across studies and the varying quality of data sources made it difficult to directly compare the studies. While in many \acp{HIC} studies some sort if incremental costing approach was used and data sources were representative for a distinct population, studies in developing countries often had to rely on data without a control group and collected non-randomly. Also, studies from low-income countries in particular had much less observations. Many studies also still lacked explicit mentioning of the used study perspective or the included costing components. 

For labour market impact studies, most found adverse effects on employment probabilities, wages or working days, suggesting an adverse effect of diabetes on these outcomes. Studies were concentrated to a few \ac{HIC}, in particular the USA. More recent studies took into account potential biases in the case of endogeneity of diabetes, mainly using an \ac{IV} strategy with the family history of diabetes as an instrument. If a bias was found, its direction was ambiguous across different studies and countries. 

The review also identified areas for which little to no studies had been found. For \ac{COI} studies, none of the studies took into account the possible of biased estimates as a result of endogeneity of diabetes. This may have led to over- or underestimations in the studies reviewed here. One potential source of bias could be accidents that have led to diabetes by restricting physical activity and at the same time also caused to higher healthcare expenditures. Further, few studies used an incidence approach to investigate lifetime costs of people with diabetes, providing better information about the dynamics of cost increases after a diabetes diagnosis. 

Despite these identified limitations of the \ac{COI} literature, they at least provides a picture of the healthcare costs of diabetes in almost every continent. This was not the case for labour market studies, where almost no evidence was found for \acp{LMIC}. Arguably, given the less advanced healthcare systems, later diagnosis but earlier onset of diabetes, the larger informal labour market and overall different labour market structure in developing countries, the impact of diabetes may be very different compared to \acp{HIC}. Also in terms of methodology, studies had not taken advantage of panel data techniques to achieve a causal interpretation of their estimates. Especially studies on the effect on employment probabilities had relied on the same identification strategy using \acp{IV} where it is at least debatable if the underlying assumptions are valid. This reliance may expose the field to wrong inference if the \ac{IV} were invalid. Only one study used panel data, but did not specifically account for the panel structure and only used pooled regression techniques. Therefore, a study using a different identification strategy is warranted taking advantage of the available panel data.

Importantly, also information about the impact of diabetes on the undiagnosed population---comprised of people with diabetes that have remained unaware of their disease---was not identified by the review. This neglected an important part of the diabetes population, especially in \acp{LMIC}.  Only one study identified by the review used biomarker information but did not specifically investigate the undiagnosed population, warranting further research.

Based on these findings, the three research studies following the review addressed the identified gaps for labour market studies. The aim of Chapter \ref{cha:Mex1} was to provide first evidence for the impact of diabetes on employment probabilities in a developing country where diabetes has become a recognized public health problem. Because little was known about the equity impacts of diabetes, a further goal was to investigate the heterogeneity of adverse effects for those in formal and informal employment and for the "rich" and "poor". Due to the lack of other identification strategies, this study also used parental diabetes as an instrument. However, using further familial background information on parental education, it improved upon earlier studies by controlling for a potential confounding pathway that could invalidate the used instrument. It further used two methods to implement the \ac{IV} approach. The preferred estimates came from a bivariate probit model that had been shown to be better suited for our specific data in comparison to a standard linear \ac{IV} model. We nonetheless also showed the results of the latter approach.

Chapter \ref{cha:Mex1} found evidence for an adverse effect of diabetes on employment chances, reducing them by about 10 percentage points for men and 5 percentage points for women. Further, no indication of the endogeneity of diabetes was found, suggesting the use of a simple probit model. The subgroup analysis suggested that the adverse employment effects occurred mainly to those above age 44, while younger people seemed less affected. Also, being poorer appeared to increase exposure to negative employment effects of diabetes. Similar was the case for those in the informal labour market. Across all models, the effects were more pronounced for males than for females. 

While these results provided good evidence for an adverse effect of diabetes on employment chances in a developing country, several questions still remained. Further, the robustness of the findings of Chapter \ref{cha:Mex1} had to be tested using more extensive and recent data and a different identification strategy. Chapter \ref{cha:Mex2} addressed these issues using a newly released addition to the data used in Chapter \ref{cha:Mex1}. The data now spanned three waves and eight years which allowed for the use of a longitudinal individual fixed effects model to estimate the relationship of self-reported diabetes with not only employment. Additionally the outcomes of interest were extended to wages and working hours. Further it was now possible to investigate the relationship of diabetes duration with labour outcomes, adding to the understanding of when people with diabetes experience adverse labour market outcomes. Importantly, the additional wave also provided information on diabetes biomarkers in order to investigate the effects of diabetes for the entire diabetes population and those unaware in particular.

The analysis carried out in Chapter \ref{cha:Mex2} confirmed the adverse relationship of self-reported diabetes with employment, finding a five percentage point reduction for males and females. Given the relatively low female employment rate for females, this translates into a 16\% decrease in female employment probabilities compared to 6\% for men, taken the respective average employment rates as the baseline. Compared to the cross-sectional results of Chapter \ref{cha:Mex1}, the estimated effects of the \ac{FE} model are about half the size for men, but are marginally bigger for women. This is likely due to the additional data used in Chapter \ref{cha:Mex2}, but could also partly be the result of the different estimation technique. For wages and working hours the results did not show an adverse effect of self-reported diabetes, suggesting that having a diabetes diagnosis does not lead to important reductions in productivity, but rather to a sudden inability to continue working. This could be caused by the appearance of very debilitating complications. Further analysis showed that in professions mainly in the informal sector, such as being self-employed or a farmer, self-reported diabetes had the greatest adverse impact. Another reason for the found effects may also have been selection into certain professions of people with diabetes. Further, Chapter \ref{cha:Mex2} revealed that the adverse effect of diabetes on employment appeared shortly after diagnosis, then levelled off for some time until it appeared again. This pattern was observed for both males and females, albeit only statistically significant for the former. Interestingly, contrary to the standard analysis using the binary diabetes indicator an adverse effect on wages was found for males and females. Further, this effect was mostly limited to 5--11 years after diagnosis, exactly the time where no employment effects were found. This downward adjustment in wages may therefore be tentatively interpreted as a result of lower productivity due to diabetes. The reduction in productivity, however, is not so strong as to justify job loss. However, due to the quite imprecise measurement of these wage effects, such an interpretation remains highly speculative.

Finally, the results of the biomarker analysis presented in Chapter \ref{cha:Mex2} provided evidence that relying on self-reported diabetes information leads to measurement bias if the effects are interpreted as representative for the entire diabetes population. Using the biomarker data to analysis found much smaller effects especially on employment probabilities, trending towards zero for males. This was caused by the non-existent associations between undiagnosed diabetes and employment chances. It was further found, that part of the difference in effects between self-reported and undiagnosed diabetes can be explained by differences in subjective health status, with those self-reporting diabetes reporting a significantly worse health status. Interestingly, differences in \ac{HbA1c} levels did not appear to be driving the stronger effects for those self-reporting. These two results leads to the following tentative explanation for the differences in results. Those how self-report diabetes are aware of the disease because they have been diagnosed. Often a diagnosis only happens after many years of having diabetes and may only be a result of first diabetes related complications manifesting. Therefore, there is considerable selection of people with deteriorating health, as a result of diabetes, into the self-reporting population. This then also, at least partly, explains the reduction in employment probabilities in this group. The unaware population, however, has remained unaware of the disease because it is still largely asymptomatic and hence not prompting a diagnosis nor reductions in productivity.  %Add ttest to Chapter 4 showing difference in subjective health  

Chapters \ref{cha:Mex1} and \ref{cha:Mex2} provided of the adverse effect of self-reported diabetes on labour market outcomes in Mexico. Chapter \ref{cha:China} continued the investigation of the impact of self-reported diabetes on employment probabilities, but this time on China. It further extended its scope to investigate how a diabetes diagnosis affects diabetes relevant health behaviours in a developing country. These health behaviours were smoking, alcohol consumption, anthropometrically measured \ac{BMI} and waist circumference, and daily calorie consumption. Because identification of a causal relationship may be confounded, the study used two different econometric strategies in \aclp{MSM} and \ac{FE}. Each controlled for a different source of confounding, improving the robustness of the identified effects. The used dataset consisted of six waves from the \acl{CHNS}, covering a period from 1997 to 2011.

The results provided further evidence of a deterioration of employment probabilities after a diabetes diagnosis, though only for women. They experienced a reduction in employment chances between 8 to 11 percentage points. For men, the \ac{FE} and \ac{MSM} showed insignificant relationships. The results of for health behaviours also suggested different effects for men and women. First of all, the descriptive results showed that smoking and alcohol consumption is much more prevalent in men than it is in females. For the latter, these risk factors were almost non-existent. All results indicated that men did not reduce smoking but alcohol consumption after a diabetes diagnosis. They further reduced their \ac{BMI} and waist circumference and calorie consumption. These reductions were small in size but might be important at a population level, given the number of people with diabetes in China. For women, no strong evidence for similar reductions was found. A similar picture remained when investigating the effects over time using linear and non-linear specifications. They suggested maintained reductions in female employment probabilities over time. Men were able to reduce their \ac{BMI} and waist circumference consistently in the years following diagnosis. No strong evidence for a rebound effect, where weight measurements would go up after an initial reduction, was found. 

These results suggest very different effects of a diabetes diagnosis for men and women. On the one hand women were unable to reduce their risk factors, but men were. On the other hand, women were those that had to bear stronger adverse economic effects. Several issues may be important to explaining this difference. First, they may be a result of a different access to healthcare resources between men and women due to difference in income, with women not receiving appropriate treatment that would have allowed them to prevent complications. Second, women may work in less protected jobs were they are easily replaceable, making it difficult to take time off for medical check ups or appropriate treatment out of fear of job loss. Unfortunately only little research explores gender differences in healthcare access in China. One study by \textcite{Fan2013} finds that female migrant workers in particular face barriers to access healthcare. Further, there relatively lower education levels compared to men may limited their ability to efficiently put the information received at diagnosis into practice, though quasi-experimental evidence that would support a causal effect of education on health behaviours in China is yet inconclusive \parencite{Xie2014a}. Whatever the true reason for this difference may be, it appears that women in China are much worse affected by diabetes than men.

\section{The context of the findings and their implications}

The findings of this thesis indicate an important economic burden of diagnosed diabetes as measured by diabetes self-reports in the \acp{MIC} of Mexico and China. Chapter \ref{cha:review} further found many studies that suggested a large burden in terms of healthcare costs in both, low- and middle-income countries. The thesis has also provided evidence that diabetes---at least in the case of labour market outcomes---does not similarly affect the unaware diabetes population. This differences is likely explained by two main factors: 

\begin{enumerate}
\item Worse health in the diagnosed population compared to the undiagnosed population.
\item Differences in health information as a result of a diabetes diagnosis by a doctor.
\end{enumerate}

Both factors are likely captured by self-reported health, which not only depends on the actual physical health status but also expresses a belief about the own health status that is influenced by the awareness of the own disease status \parencite{Jylha2009}, here awareness of diabetes. Because self-reported health is worse in those diagnosed, it is likely that they are in an actual worse physical health state. Time since onset in the self-reporting diabetes population is almost surely longer than in the undiagnosed group as onset is often several years prior to diagnosis. Because severe diabetes complications take several years to develop after the onset of diabetes, they are more likely to be prevalent in the self-reporting population. Also because the occurrence of a  complication may had been the reason for a doctor visit and the diabetes diagnosis in the first place. The additional health information as a result of the diagnosis could affect labour market outcomes in two ways. The results of Chapter \ref{cha:China} suggest, that it could lead to improved health behaviours, potentially reducing the health burden of diabetes. However, it could also worsen health and consequently adversely affect labour market outcomes by increasing anxiety or depression. So did a study on China find a reduction in labour income as a result of the additional health information received at diagnosis \parencite{Liu2014}. 

These findings may lead to several implications in order to reducing the economic burden of diabetes in \acp{MIC}. First, the findings from China suggest that a diagnosis of diabetes can be positive as it has the potential to lead to a reduction of risk behaviours. While the found effects were small in size, at least for the weight related measures, on a population level they may lead to significant reductions in the risk of complications.

Further, the finding that adverse labour market outcomes were only observed for the diagnosed population suggest that these only occur after some time of living with the disease. They further indicate that many people with a diagnosis are not able to prevent debilitating complications to occur during their productive lifespan. One reason may be that diagnosis happens to late to prevent complications for an extended period of time. This is also what is suggested by the large undiagnosed population found for Mexico in Chapter \ref{cha:Mex2} as well as for other \acp{LMIC} in a recent study by \textcite{Beagley2014}. Therefore, efforts to achieve earlier diagnoses of diabetes in countries with a large undiagnosed diabetes population may well be worthwhile \parencite{Engelgau2012}. Even though this will increase healthcare demands and costs in the short term, such effects may be set off by increases in productivity and productive years in the working population with diabetes, as well lower inpatient expenditures due to reduced rates of severe, cost-intensive complications such as dialysis. Evidence on the cost-effectiveness of a population-based diabetes screening program provided a recent study from Brazil, where over 22 million people over the age of 40 were screened for diabetes \parencite{Toscano2015}. This study is the first study providing cost-effectiveness estimates based on an actual population-based diabetes screening program. Using a Markov model they investigate the long term cost-effectiveness of this program from a public healthcare system perspective. The findings are inconclusive as to whether this intervention could be cost-effective at conventional thresholds. It depends strongly on the used assumptions about the ability of first line treatments to prevent coronary heart disease and stroke. Further, the societal perspective was not considered in this study, from which the cost-effectiveness of screening may be greater if an earlier diagnosis leads to increases in productivity and a longer productive lifespan. Of course, early diagnosis may only be reasonable if the healthcare system is sufficiently developed to allow all diagnosed cases access to appropriate treatment options. 

A further implication of this thesis are the found inequities. They manifested in particular in the economic burden of diabetes being disproportionately large for the poor and generally less protected against negative health shocks. In Chapter \ref{cha:review} the studies reviewed suggested a high \ac{OOP} burden in \acp{LMIC}, especially for those with no insurance coverage. Further, the results of Chapter \ref{cha:Mex1} showed that the adverse employment effects were concentrated among those in the informal labour market and with fewer resources. This was further supported by findings from Chapter \ref{cha:Mex2} that indicated a greater reduction in employment probabilities to work in the agricultural or self-employed sector, while for those working in a non-independent wage job---that often entails greater contractual job security and better access to health insurance---diabetes did not appear to elicit negative effects. Finally, the results for China from Chapter \ref{cha:China} showed a much stronger adverse employment effect for females than for males. Also in Mexico the relative reduction in employment chances was much greater for females than for males when the generally lower employment rates for females are taken into account. All this suggest that women are disproportionately affected by diabetes. Finally the results from China also suggested that women are less likely to achieve positive and sustained changes in health behaviours.

These findings can be placed in the context of a larger literature on inequalities in \acp{NCD} showing less access to care for people living in \acp{LMIC} and, especially, for women with diabetes \parencite{DiCesare2013}. There are several proposed strategies how to reduce these inequalities and improve access to care \parencite{Jacobs2012}. Several of these will be presented here with a focus on the identified populations in this thesis. A potentially worthwhile goal could be to improve access to care for women. One of the components likely hindering women to appropriately access and use healthcare are their lower educational levels \parencite{Jacobs2012}. These likely reduces their ability to effectively use the information received at diagnosis and also reduces their income levels, making it more difficult to access appropriate treatment in the first place. Accordingly, improving female access to education may already go a long way in improving their later life health outcomes. Further, particularly in China, women had been exposed to discrimination in early childhood that may have reduced their education attainment \parencite{Zeng2014}. Even today there still appears to be some gender inequality in education in China, even though it is narrowing \parencite{Zeng2014}. Also female migrant workers have been shown to have less access to healthcare compared to male migrants and locals, so targeting this group to improve their access may be an important measure to reduce the economic disease burden of diabetes for Chinese women \parencite{Fan2013}. 

Finally these differences between men and women in the economic burden may also be the result of actual biological differences in the risk of complications. Two recent systematic reviews found a significantly higher risk of coronary heart disease and stroke, respectively, in women with diabetes compared to men with diabetes \parencite{Peters2014a,Peters2014}. The results were robust across different populations, including Asian populations. While there are likely also differences in the management of diabetes between men and women, with women achieving fewer treatment goals, the difference in healthcare access and usage likely does not completely explain the higher risk of women with diabetes to have a stroke or coronary heart disease \parencite{Peters2015,Peters2014a,Peters2014}. Potentially, part of the difference is explained by the preferred ways of fat storage, with women storing fat mainly subcutaneously while men have more visceral fat. While this protects against an earlier onset of diabetes, women may be living in a hazardous metabolic state for a longer time, until they have accumulated sufficient visceral fat to reach a diabetic state. It therefore appears that women spend a longer time in a pre-diabetes state then men \parencite{Bertram2010}. So once diabetes appears it may be more problematic and lead to greater complication rates as women with diabetes tend to be in a worse metabolic state compared to men with diabetes \parencite{Peters2015}. 

Proposed strategies to reduce the risk for women are, first of all, to ensure equal access to healthcare and equally aggressive diabetes treatment for men and women. Further, awareness among doctors about the higher risks for women has to be increased and screening for cardiovascular risk factors in women with an increased risk of diabetes could be sensible. It would present an opportunity to prevent a further escalation of the cardiovascular risk profile before a diabetes diagnosis and afterwards. Innovative ways have to be found to then provide persons at risk with lifestyle programs that can also be accessed by people living in low-income settings. Weight reduction thereby seems to be the single most important step to reduce the risk of diabetes and ensuing complications. The long duration of pre-diabetes could provide a window of opportunity to identify high-risk populations and provide them with measures to reduce their risk through lifestyle changes \parencite{Peters2015}.

The existing evidence on treatment models applicable in very resource constrained settings  has recently been reviewed by \textcite{Esterson2014}. While research on this topic is still limited, the study provides information on interventions that have had some success in improving diabetes treatment for the poor. Further, it identified common characteristics of these interventions that contributed to their success: collaboration, education, standardization of guidelines and algorithms, technological innovations, and resource optimization. Accordingly, initiatives to provide care to underserved populations should, if possible, be build on collaborations between academic institutions, hospitals, the private sector and other organizations such as local governments, in order to achieve goals that would otherwise be difficult to reach for one stakeholder alone. Further, programs should aim at providing appropriate education to doctors, so that they have the knowledge to successfully treat people with diabetes. Further, using peer-support programs may be a viable option in remote communities, so that few well educated community members or nurses can help their peers with the challenges of diabetes management. Standardized guidelines and treatment algorithms can help doctors and nurses to improve their standards of care and maintain these standards. Given that mobile phones have already reached even very remote areas interventions using technological innovations based on already existent technologies, could improve care and diabetes outcomes by improving communication between doctors and their patients as well as by making it possible to better track and control diabetes management and outcome measures. Finally, resource optimization to use available and constrained resources more effectively, e.g., by transferring certain responsibilities from doctors to nurses or from healthcare professionals to peers while providing them with the needed education to carry out these new tasks. This may be especially important in cases where the needed healthcare professionals are not readily available or financial limitations prevent new hirings \parencite{Esterson2014}.

Finally, diabetes prevention has to play a large role in preventing a further increase in the diabetes burden in the future. Given the results of this thesis, these interventions should have the particular goal of preventing diabetes across the entire population, and in particular among the poor in \acp{LMIC}. A number of population level interventions have already been introduced, with a 10 percent tax on purchases of sugar-sweetened beverages and "junk food" in Mexico being a prominent case. First results suggest a reduction in purchases of these goods, with a steeper decline for those with lower income levels \parencite{Colchero2016,Batis2016}. If these changes in consumption actually lead to a healthier diet and are large enough to cause reductions in obesity and diabetes prevalence remains to be seen, however \parencite{Singh2016}. There is some evidence that population based awareness campaigns of diabetes and campaigns to promote changes in health behaviours could also introduce positive changes in health. Further, there were efforts to increase physical activity by providing easy access to sport courses and fitness equipment. Overall, it appears that interventions aiming to improve lifestyle in terms of physical activity and dietary changes have shown promising results across the globe, including developing countries such as India and China, reaching long term reductions in the risk of developing diabetes \parencite{Cefalu2016}. The evidence for pharmacological interventions mainly using metformin also indicates a reduction in the risk of diabetes. However, \textcite{Cefalu2016} also mention the potentially large heterogeneity in the benefit of pharmacological interventions across ethnicities. More research to this respect will be needed to find out if successful pharmacological interventions in one ethnicity can be translated to other ethnicities. Further, these interventions were tested in randomized controlled trials, and translation into real-world settings has been less successful. There are also questions about the cost-effectiveness of these interventions if scaled to a population level and the problem of finding sufficiently educated personal to implement these lifestyle interventions at the local level. Therefore, \textcite{Cefalu2016} also argue for considering preventive metformin treatment in individuals with a high risk of progressing to diabetes. Given that low-cost generic versions of metformin exist and considered essential diabetes medications in almost all \acp{LMIC} \parencite{Bazargani2014} and its relatively high effectiveness in preventing or delaying the onset of diabetes and its low risk of side-effects \parencite{Gomes2013}, this may also be a relatively cost-effective intervention more conveniently applied using existent healthcare infrastructure and pharmacies. It could be especially effective in \acp{MIC}, where the healthcare system infrastructure is much more developed than in \acp{LIC}.

Obviously, the identification of high-risk individuals that could be targeted with the mentioned interventions may pose an additional hurdle to successfully preventing diabetes. The already mentioned population level screening could be a way to identify people at risk. Screening could also be carried out at the workplace or the community and existing medical records could be used to identify people at an increased risk Further, risk scores available on the internet could help people self-screen for their risk after having been made aware of this possibility by advertising and social media \parencite{Cefalu2016}. Unfortunately, scientific evidence of the cost-effectiveness and feasibility of screening for high-risk individuals in \acp{LMIC} is non-existent. 

The results of this thesis suggest that targeting those with little access to healthcare in screening programs for both undiagnosed diabetes and those at high-risk for diabetes and following up with offers for preventive pharmacological treatment and potentially also lifestyle interventions could be of value. Identifying the poor at high-risk could reduce the risk for catastrophic future health expenditures by preventing or delaying complications of diabetes and may also prevent the adverse labour market effects found in this thesis.


\section{Reflections on the methods used in the thesis}

Apart from \ref{cha:review}, the thesis used exclusively quantitative methods in an attempt to establish causal relationships between diabetes and the outcomes of interest. Given the good quality of available data and the dearth of previous quantitative research on the economics of diabetes in \ac{MIC} using quantitative, and in particular econometric methods seemed to be the most appropriate way to answer the posed research questions. The goal of using econometrics in this thesis was to investigate the relationship of diabetes with labour market outcomes and health behaviours in the absence of experimental data. 

One of the challenges is the choice of the most appropriate method to establish a causal relationship. For the relationship of diabetes with labour market outcomes, one main concern was that unobserved variables, measurement error as well as reverse causality may introduce bias into the estimates. A variety of methods were used that each had advantages and disadvantages in terms of the underlying assumptions and the ability to account for potential sources of bias. Nonetheless, regardless of the method used, results consistently showed am adverse relationship of self-reported diabetes with employment probabilities, suggesting a relatively robust and likely causal effect.

\section{Strengths and limitations}

The strengths and limitations of each study and the methodological approach used has been evaluated within each chapter. Additionally, the thesis overall has strengths and limitations.

The strength of this thesis is the contribution it makes to the evidence base on the economics of diabetes in \acp{MIC}. A further strength is its contribution to the evidence on the potential of diabetes to widen the economic inequities in developing countries. Additionally, it gives an overview of the current state of research on the economic costs of diabetes, providing other researchers guidance in identifying areas for future research and the use of the most appropriate methods to do so. Further, it has also advanced the understanding of diabetes as a multifaceted condition which, in order to successfully prevent its economic burden, has to be taken into account accordingly. Overall, it provides a first comprehensive overview of how diabetes affects people in \acp{MIC}.

The thesis has several limitations. Whilst the intention was to provide evidence on the economics of diabetes in \acp{MIC}, the thesis mostly provides evidence on the economic impact of diabetes. This is only one part of the relationship between diabetes and economics. So did the thesis not investigate potentially costs-effective interventions that could be applied in \acp{MIC} to lower the economic burden of diabetes. This is important as resources are especially scarce in these countries and therefore highly cost-effective interventions are needed. Further, simply transferring interventions that have been proven to be highly cost-effective in \acp{HIC} may not work due to a different cultural context, priorities and especially healthcare systems in \acp{MIC} \parencite{Mills2014}.

This leads to the next limitation. The thesis does not investigate in how far healthcare systems in \acp{MIC} need to change in order to better provide care. Given that they often lack financial resources and the efficient use of the available resources, are designed to treat acute infectious diseases rather then affecting the outcomes of long-lasting non-acute \acp{NCD}, and often provide unequal access to their health services due to financial constraints of those seeking care, research into how to strengthen these systems is urgently needed \parencite{Mills2014,Guzman2010}.

A further limitation is the geographical concentration of the thesis in the empirical investigation. While Mexico and China are two of the ten countries with most people with diabetes in the world, there are other large and small \acp{MIC} that are currently or will likely be facing similar challenges due to an important part of the population being at risk for of already having diabetes \parencite{Risk2016}. While the challenges might be similar, each country tends to have its own specific context in terms of culture, the political system, economic development and equity and equality issues. Therefore, it cannot be assumed that the results obtained in this thesis are likely to be found as well in other \acp{MIC}. They can only provide some indication of the potential size of the economic burden of diabetes and its effects on health behaviours in a \ac{MIC}.

Finally, while the thesis intended to provide a picture of the potential inequities in the economic impact of diabetes for socioeconomic subgroups, it does not investigate in detail why these inequities exist, but can only speculate on the reasons. This is important as a better understanding of the underlying reasons could help to design adequate strategies to address these inequities. Further, whilst it has touched on the potential reasons for the differences in employment effects between those self-reporting diabetes and those unaware, it has not provided an in depth analysis of this phenomenon. These would be important as it would allow for a better identification of the underlying reasons and of interventions aiming to prevent these adverse economic effects. 



\section{Suggestions for future research}

Preventing the economic burden of diabetes in \acp{MIC} while having in mind the inequities in the economic impact of diabetes will form the basis for the following recommendations.

This thesis has shown the global economic impact of diabetes and its adverse effect on labour market outcomes in Mexico and China. It has further demonstrated that it is the poor and potentially also women that are most adversely affected by it. It further found that, at least in China, it is men that appear to profit most from a diabetes diagnosis in terms of positively changing their health behaviours. Finally, it provided some indication that while self-reported diabetes is related to adverse labour market effects, undiagnosed diabetes is not. Accordingly, it has identified three groups that are particularly affected by diabetes: the poor, women and those self-reporting diabetes. Without a greater understanding of the underlying reasons for the found differences it will be difficult to design policies that can help prevent the burden of diabetes in \ac{MIC}.

First, the investigation of the observed differences between men and women. It remains unclear what the reasons are that appear to cause women to experience worse labour market outcomes and also to be less able to achieve a positive behaviour change. Two potential reasons come to mind. Firstly, a biological explanation where diabetes in women could be more severe due to a worse metabolic health profile as suggested by \textcite{Peters2014,Peters2015} and therefore leads to a more rapid development of debilitating complications, in particular due to a greater relative cardiovascular disease risk \parencite{Arnetz2014,Roche2013,Policardo2014,Catalan2015,Engelmann2016,Seghieri2015}. Biology in terms of a different hormonal profile of women may also be the reason for their lower ability to lose weight \parencite{Penno2013}.  Secondly, particularly in \acp{MIC} women may face a reduced access to treatment, causing a later diagnosis and less intensive treatment, explaining part of the observed differences. While evidence from Italy suggests that it is not differences in treatment leading to worse outcomes but rather other factors \parencite{Penno2013}, in poorer countries with a different social context, women may still additionally face reduced access to treatment. The use of biomarker data in combination with information on healthcare utilization may help to disentangle the relationship. A potentially rich source of information could be provided by two Chinese household surveys, the \acf{CHNS} and the \acf{CHARLS}, that both provide an extensive list of measured biomarkers and socioeconomic variables that could help to investigate differences in metabolic risk between men and women and those self-reporting diabetes and those unaware, and relate this to economic outcomes.

Second, researchers should also try to confirm this pattern using different data and in other countries to establish if the relationship between diabetes related labour market outcomes and poverty is robust. If this is the case then the underlying drivers of these inequities need to be explored to design adequate policies. This could either be done by identifying countries where this relationship may not have been found to isolate the causal determinants. Further, strategies implemented currently or in the future in \acp{MIC} that aim to reduce these inequities, such as the implementation of universal health insurance schemes such as Seguro Popular in Mexico, need to be evaluated in how far they are actually achieving this goal in terms of diabetes. The same is true for population level interventions such as taxes on sugar or fat, as these theoretically should increasingly reduce consumption for those with lower levels of income. This could then lead to a reduction in the diabetes incidence in these groups. However, depending on the price elasticities of the taxed products as well as substitution effects with equally untaxed products, such taxes may only reduce disposable income for other food purchases or to a shift in consumption towards other equally unhealthy products. REFERENCE NEEDED


Third, the diabetes population in all countries, but especially in \acp{LMIC} is only partially observed, in other words many people with diabetes are not aware that they have the disease. This thesis has done a first investigation of the differences between those aware and unaware. It, however, still remains unclear in how far the sheer knowledge difference compared to differences in other factors, such as health status, are behind the heterogeneity in the economic impact in both groups. Because more datasets are providing biomarker data in combination with socioeconomic information, they should be used together with quasi-experimental econometric techniques to investigate this topic. So could regression-discontinuity design be used in a similar vain as in \textcite{Zhao2013a}, who use cut-off values for hypertension identify those newly diagnosed and the subsequent effect on health behaviours in China. A similar approach could be used to explore the effects of a diabetes diagnosis and the entailed health information on labour market outcomes, health behaviours and other economic outcomes. Importantly, researches should assess the heterogeneity of effects across socioeconomic statuses, i.e. income groups, rural versus urban, by education levels and between males and females. All this will provide important information for designing interventions to reduce the physiological and economic burden of diabetes and preventing a widening of inequities.

Fourth and finally, there is a need to explore further economic downstream effects of the economic impact of diabetes. If diabetes causes reductions in employment and potentially also income, it is likely that these will cause not only problems for the individual directly affected but for the people closely related such as family members. In \acp{MIC}, where social security is less extensive and comprehensive, adverse health shocks due to diabetes likely also have consequences for the children living in that family and spouses ore even grandparents \parencite{Alam2014}. The loss in income needs to be compensated either by increasing labour supply of other household members or by reducing expenditures for other goods. Both could affect children directly, for example by reducing their time for education due to reduced household resources to pay for tuition fees and also by substituting time for education with labour time. Similarly spouses may be forced to increase their labour supply, reducing the time they can care for their children. These effects have remained unexplored for diabetes but given the scale of the diabetes epidemic may be not trivial.

Overall, researchers should address these questions using the novel data that has come and is coming out to increase the understanding of the effects of diabetes on the physiology and psychology of people especially, though not exclusively, in \acp{MIC}.


\section{Concluding remarks}

Diabetes presents a major challenge in \acp{MIC}, but evidence on its economics is scarce. This thesis has found that diabetes has an adverse economic impact on individuals affected by the disease and puts a burden on healthcare systems. Because evidence on the impact of diabetes on labour market outcomes was lacking in developing countries, the thesis had a special focus on this topic. Thereby it not only provided evidence of the adverse impact of diabetes on employment, but also improved upon previously used econometric methods by using novel strategies to identify a causal relationship. The thesis also identified potential inequities in the impact of diabetes, pointing to larger adverse effects on the poor, those in the informal labour market and on women. The thesis also found evidence for differences in the ability to change health behaviours for men and women, with men being more successful then women. 

These findings suggest that there is a need to reduce the economic impact of diabetes in \acp{MIC}. Considering the increasingly earlier onset of diabetes and the ongoing incidence rise in many countries, the non-trivial adverse economic effects will otherwise hinder economic development and pose a substantial poverty risk. Strategies to prevent this from happening a urgently needed. These need to be tailored towards the respective population having in mind potential differences in disease severity between Whites, Asians, Afro-Americans or Mexicans, and men and women. Further, these policies need to aim at reducing inequities in the impact of diabetes on those at the lower socioeconomic spectrum. Finally, there is a large undiagnosed diabetes population in \acp{MIC} that is likely to experience severe diabetes complications if not treated. Hence, ways to diagnose this population earlier in order to prevent further deterioration of health may go a long way in preventing and delaying the most catastrophic economic and health outcomes.

In conclusion, it is hoped that this thesis, and the publications born out of it, contribute to the knowledge on the economics of diabetes, in particular in \acp{MIC} and helps to identify cost-effective strategies to lower the health and economic consequences of diabetes. It has demonstrated the economic burden currently caused by diabetes, in particular in Mexico and China, and has identified groups that are particularly vulnerable to the negative consequences of the disease and should be at the centre of efforts to prevent the burden of diabetes. 






*methodologically use quasi experimental methods and biomarker data to explore these effects 
*further exploration of difference for males and females
*what is driving differences for women
*in China, how is diabetes among migrant women?
*innovative ways need to be found to reduce risk in high rsk groups in low income settings / commercial providers may be able to reduce weight more successfully. needs to be tailored to respective setting. may work in mexico but not in china














