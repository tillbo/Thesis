To do


CONTINUE WITH UPDATING TABLES. FINISHED MAIN TABLES, BUT ROBUSTNESS CHECKS AND DESCRIPTIVES ARE MISSING
LOOK AT COVARIANCE ADJUSTED ESTIMATES AGAIN

*Introduction:

Need to decide if I want to number research questions or not. Probably better to not do it.

*Chapter 5:  REFERENCE NEEDED:Further, its main conclusions are based on the dynamic panel data models where the lagged dependent variable is used as one of the predictors. This method, however, might suffer from biased estimates.

* Chapter 5: Continue with comments of Max in results section
*maybe use truncated instead of untruncated results as main results
*maybe add migration status to set of covariates as ample literature shows that migrants are worse off
*threeparttable in chapter 3 to have nicer looking notes below tables
*what is age at diagnosis for those in informal labour market in Mexico. . The older they are the more this would suggest that due to lack of insurance they get diagnosed really late and therefore bear larger burden. Maybe have exploratory analysis of this in chapter 4?
*correct statement in Cha 3 that women are less affected. this is not true if we look at relative changes instead of absolute

TALK ABOUT MORTALITY DUE TO DIABETES AND HEALTH BURDEN ESPECIALLY IN MICs. ALSO NEED TO TALK ABOUT INEQUALITIES IN DIABETES ANYWHERE FURTHER DOWN


**FUTURE RESEARCH
*Are rural women more affected by diabetes (or migrating women)?




see Jacobs, B., Ir, P., Bigdeli, M., Annear, P.L., Van Damme, W., 2012. Addressing access barriers to health services: An analytical framework for selecting appropriate interventions in low-income Asian countries. Health Policy Plan. 27, 288–300. doi:10.1093/heapol/czr038

Ali, M.K., Narayan, K.M.V., 2016. Screening for Dysglycemia: Connecting Supply and Demand to Slow Growth in Diabetes Incidence. PLOS Med. 13, e1002084. doi:10.1371/journal.pmed.1002084

M. Silink, J. Tuomilehto, J. C. Mbanya, K. M. Venkat Narayan, Judy Fradkin, G.R., 2010. Research priorities: Prevention and Control of Diabetes with A Focus on Low and Middle Income Countries. WHO Meet. Dev. A Prioritized Res. Agenda Dev. Prev. Control Noncommunicable Dis. 6.