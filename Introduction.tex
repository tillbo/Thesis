
\section{Background to the thesis}

Diabetes, and especially type 2 diabetes, has seen an unprecedented rise in prevalence globally and especially in \acp{LMIC}, where rates reached and often surpassed those of \acp{HIC} such as the USA, UK or Germany \parencite{Risk2016,Hu2011}. Today, two-thirds of the over 400 million people with diabetes live in \acp{LMIC} \parencite{InternationalDiabetesFederation2013} and in particular in China, India, Brazil, Indonesia, Pakistan, Russia, Egypt and Mexico \parencite{Risk2016}. In 2015 diabetes has been responsible for over 5 million death and people with diabetes are estimated to die 6 years earlier due to the disease and increasingly before the age of 60 \parencite{InternationalDiabetesFederation2015,Seshasai2011}. This increase is due to a shift in age structure towards older populations and is further spurred by rapid changes in levels of physical activity, nutrition and other lifestyle related factors \parencite{Risk2016,Hu2011}.

In \acp{LMIC} the rise of \acp{NCD} has in many cases led to a double disease burden, where health systems have to deal with high rates of infectious as well as non-communicable diseases \parencite{Jamison2013}. Given the scarce resources in these countries \parencite{Mills2014}, the increasing number of people with diabetes and at risk of the disease are putting an additional burden on these systems \parencite{Wareham2016,Chan2016}. However, despite the epidemic levels diabetes has reached in \acp{LMIC}, research on its economic consequences has remained sparse for these countries and mostly limited to \acp{HIC}. More research is needed to identify how diabetes is affecting individuals in \acp{LMIC} and what are the groups most adversely affected. This could help raise awareness of policy makers of the size and of the potential inequities of the disease burden, and help to design strategies to reduce them.

Currently healthcare systems in \acp{LMIC} are likely further increasing inequities by providing better care and coverage for those in formal employment and economically better off \parencite{Mills2014,DiCesare2013}. For Peru, a recent study identified several barriers to care for people with diabetes, that are likely highly relevant for other \acp{MIC} as well. They included a generally low political commitment to improve access to and the quality of diabetes care, little qualified personal to treat diabetes at the primary care level, high out-of-pocket expenditures partly related to the seeking of specialized diabetes care in the private sector, and few resources in the healthcare budget being allocated to non-communicable-diseases treatment despite its high mortality burden \parencite{Cardenas2016}. Further, it appears that diabetes diagnosis happens often too late to prevent first complications, with a first diagnosis often being made after a patient had been admitted to a hospital emergency department due to diabetes related complications \parencite{Cardenas2016}. Similar observations have been made for other \acp{LMIC} \parencite{Beran2015,WHO2014}.

 
\subsection{Types of diabetes}

Diabetes is a term used to describe various conditions characterised by elevated blood glucose levels. These either occur because the pancreas is not able to produce sufficient insulin, or due to insulin resistance, where the body is not able to use the produced insulin effectively \parencite{WorldHealthOrganization2016}. The different conditions themselves have distinct origins, especially the two most common types called type 1 diabetes and type 2 diabetes. 

\begin{itemize}
\item \textbf{Type 1 diabetes} is an autoimmune disease with an important genetic component and whose triggers still remain largely elusive. It emerges when the insulin producing cells on the pancreas are attacked and destroyed by the immune system and insulin has to be provided exogenously. About 10\% of all global diabetes cases are type 1 diabetes and it is particularly prevalent in Northern European countries such as Finland, though generally exhibits much geographic variation. Its onset is mainly during the first 30 years of life. Symptoms tend to appear rather quickly and can be quite severe leading to a relatively rapid diagnosis or death, if insulin is not given or available. People with type 1 diabetes will need to inject insulin to control their blood glucose levels for their entire life following diagnosis \parencite{Tuomilehto2013}. 
\item \textbf{Type 2 diabetes} results from the body's ineffective use of insulin and accounts for about 90\% of all diabetes cases \parencite{WorldHealthOrganization2016}. Albeit there is a considerable genetic component to the development of type 2 diabetes, there are many known risk factors that favour the development of type 2 diabetes, such as overweight and obesity, unhealthy diet, physical inactivity and smoking, among others \parencite{WorldHealthOrganization2016, AmericanDiabetesAssociation2014}. Interestingly, the risk to develop type 2 diabetes varies also by ethnicity, with South-East Asian populations developing diabetes at lower \ac{BMI} levels than populations of European decent \parencite{Ramachandran2010}. Type 2 diabetes often remains undetected for several years due to its more gradual development compared with type 1 diabetes \parencite{AmericanDiabetesAssociation2014}. Therefore, even in \acp{HIC} and especially in \acp{LMIC}, a proportion of at least 1/4 of the type 2 diabetes population is unaware of the condition \parencite{Beagley2014}. 
\end{itemize}

The onset of type 2 diabetes also appears to be increasingly earlier in life. This has been observed mainly in ethnic minorities in \acs{HIC}, such as Mexicans and Asians, while data is limited for \ac{LMIC} \parencite{FazeliFarsani2013}. Also the increasing numbers of obesity in child- and early adulthood are leading to the earlier onset of type 2 diabetes \parencite{Chen2012}. Hence, type 2 diabetes increasingly affects people in the middle of their productive lifespan, extending the time they have to live with the disease and the probability of developing debilitating complications.

\subsection{Diabetes complications}

The most common complication for all types of diabetes, and often already present at diagnosis, is retinopathy (35\% at diagnosis), being responsible for 2.6\% of blindness globally. Further, up to 50\% of cases of end stage renal disease are a direct result of diabetes, especially in countries where access to dialysis is restricted. People with diabetes also have a 2--3 times higher risk to experience cardiovascular disease compared to people without diabetes. A further complication is amputation of lower limps due to impaired wound healing, being 10--20 higher for people with diabetes. In addition to these microvascular complications, diabetes has its greatest health impact as a risk factor for cardiovascular disease and stroke \parencite{WorldHealthOrganization2016}. There is also a growing literature suggesting a---potentially bidirectional---relationship between diabetes and depression \parencite{VanDooren2013,Nouwen2010,Roy2012}. In addition, there seems to be a link between diabetes and the development of certain types of cancer, \parencite{Tsilidis2015,Nead2015}, as well as an array of other of other infectious diseases, intentional self-harm and degenerative disorders diseases \parencite{Seshasai2011}.


\subsection{Diabetes prevention}

Diabetes complications are a result of consistently elevated blood glucose levels, and  are aggravated if blood pressure is high as well, as is often the case. Hence many complications could be prevented if recommended treatment goals were achieved. However, limited resources and access to healthcare make it difficult to properly treat type 2 diabetes in \acp{LMIC} \parencite{Villalpando2010}, and even in \acp{HIC} a large part of the diabetes population does not achieve treatment goals to prevent complications \parencite{DiabetesUK2012}. 

Primary prevention of diabetes or at least a delayed onset are further major goals of diabetes research and could be achieved by reducing the prevalence of the known risk factors such as obesity, an unhealthy diet and sedentary behaviour \parencite{WorldHealthOrganization2016}. However, so far most approaches to prevent type 2 diabetes have not had the desired effect and may not always be realistic in very resource constrained settings \parencite{White2016}. In particular efforts to reduce the biggest type 2 diabetes risk factors of obesity and overweight have been unsuccessful \parencite{Roberto2015}.

\subsection{The need for further economic research on diabetes}

To design effective interventions and make qualified decisions about the use of primary and secondary prevention strategies of diabetes, researchers and policy makers need information about the current burden of diabetes, both in terms of health and economically. Information on all aspects of economic costs and the quality of the estimates has to be available optimally. In particular, in \acp{LMIC} equity issues are likely to be of importance if the burden of diabetes varies by socioeconomic groups, ethnicity or sex, potentially widening existing socioeconomic inequities. However, at the start of this thesis, little was known about the economic impact of diabetes in developing countries. There had, to my knowledge, not been a comprehensive systematic review of studies assessing the costs related to diabetes, both in terms of direct and indirect costs. One (non-systematic) review existed \parencite{Ettaro2004}, including \ac{COI} studies published until the year 2001. Completely absent in that review were studies from \acp{LMIC}. Further, considerable time had passed since that review and the methodological quality of research published since then needed to be assessed and areas of future research had to be identified. Also missing was a comprehensive overview of studies using quantitative methods to estimate the impact of diabetes on labour market outcomes, such as employment and wages.



\section{Objectives of the thesis}

The thesis focuses on three main research questions related to the economics of diabetes in \acp{MIC}. 

\begin{enumerate}
\item What is the global economic burden of type 2 diabetes, both in terms of \ac{COI} and the labour market effects of diabetes? 

\item What is the impact of diabetes on labour market outcomes in \acp{MIC}?

\item How does a diabetes diagnosis affect behavioural risk factors?

\end{enumerate}

These three research questions are answered in Chapters \ref{cha:review}, \ref{cha:Mex1}, \ref{cha:Mex2} and \ref{cha:China}. Thereby several sub-themes are explored, including the potential inequities of the economic burden of diabetes, time trends in the impact of diabetes on labour market outcomes and behavioural risk factors, the robustness of the found results to different estimation techniques and settings, and heterogeneities in the impact of diabetes between those aware and those unaware of the condition.


\subsection{The global economic burden of diabetes}

Chapter \ref{cha:review}: \textit{The Economic Costs of Type 2 Diabetes: A Global Systematic Review} provides a first comprehensive global picture of the economic burden of type 2 diabetes, including both \ac{COI} studies and studies on the labour market effects of diabetes from both \acp{HIC} and \ac{LMIC}. Together, the aim was to provide information on the economic costs of diabetes for as many countries as possible. Another goal was the identification of research areas, both in terms of methodology and topic, where evidence was lacking and/or current methodologies could be improved upon. This was intended to guide the subsequent chapters of this thesis as well as other researchers interested in the economics of diabetes. Chapter \ref{cha:review} thereby answers research question one.

\subsection{The labour market impact of type 2 diabetes}

The review identified the labour market impact of diabetes in \acp{LMIC} as a topic that had not received much attention. Apart from the lack of evidence from developing countries, there was also scope for methodological improvements compared to the existing \ac{HIC} evidence. Further, information on the effects on sub-populations, i.e. comparisons between rich and poor and the formal and informal labour market were non-existent.

However, in order to carry out such an analysis, appropriate data needed to be identified. To this end a search for suitable household data from \acp{LMIC} was carried out, using generalas well as specialized search engines such as the World Bank Central Microdata Catalog  \url{http://microdata.worldbank.org/}, the Demographic and Health Survey Database \url{http://dhsprogram.com/data/},the Global Health Data Exchange Database \url{http://ghdx.healthdata.org/}, and the International Household Survey Network Catalog \url{http://catalog.ihsn.org/index.php/catalog}. The aim was to identify datasets containing information on self-reported or measured diabetes. Specialized websites providing an overview on household survey data in developing countries were also scoped to identify relevant data (such as \url{http://ipl.econ.duke.edu/dthomas/dev_data/index.html} and \url{https://sites.google.com/site/medevecon/development-economics/devecondata/micro} for household survey from developing countries, and an overview on data sets containing biomarker information provided by The Biomarker Network at \url{http://gero.usc.edu/CBPH/network/resources/studies/}). An overview of the identified surveys is provided in Table \ref{tab:datasets} in the appendix.

Given the availability of data and the extent of diabetes in \acp{MIC} compared to \acp{LIC}, a decision was made to focus on \acp{MIC} for the remainder of the thesis. In particular, Mexico and China were chosen to be investigated. The main reason was the availability of suitable  data provided by the \ac{MxFLS} and \ac{CHNS}. First, the \ac{MxFLS} was used to investigate the impact of diabetes on labour market outcomes in Mexico as the data provided information on important covariates, including parental diabetes, not available in other surveys. Further, Mexico is a country with particularly high obesity and diabetes rates making it an interesting case to study. Chapter \ref{cha:Mex1} therefore investigates the causal effect of diabetes on employment probabilities in Mexico, providing first answers to research question two.

\subsubsection{Identification of the causal effect of diabetes on labour market outcomes}

As is eluded to in Chapter \ref{cha:Mex1}, identifying a causal relationship of diabetes with labour market outcomes is being complicated by the possibility of unobserved time-variant and -invariant heterogeneity. In Chapter \ref{cha:Mex1}, an \ac{IV} approach was used as a first step of analysis, to address this research question. However, as is often the case with \acp{IV}, the identification strategy is imperfect and it remains open to debate whether the instrument used fully satisfies the exclusion restriction, even if formal econometric testing suggests it does, leaving the possibility of biased estimates. Several other strategies potentially exist to identify the true effect of diabetes on labour market outcomes using quasi-experimental econometric approaches \parencite{Antonakis2012}. For example, a natural experiment---that would affect people's diabetes risk while at the same time have no direct effect on labour market outcomes such as employment probabilities or wages---may be used. However, a setting with exogenously introduced variation is notoriously difficult to find (moreover, it  may provide information only for a very---often geographically or economically---specific population that has been exposed to this natural experiment). Another strategy to improve inference is the use of panel data and in particular the \ac{FE} estimation, which does not depend on exogenously introduced variation. Relying only on within-individual variation the strategy allows to fully account for time-invariant factors that may affect diabetes and labour market outcomes simultaneously. This is likely of importance in the case of diabetes and economic outcomes, where the use of \ac{IV}s has been motivated by the possibility that unobserved character trades---generally thought to be stable over time---such as motivation as well as early life experiences may be confounding the relationships \parencite{Seuring2015}.

Therefore, part one of Chapter \ref{cha:Mex2}, takes advantage of a recent addition of data to the \ac{MxFLS} to apply a \ac{FE} estimation approach, testing if the effects of diabetes on employment probabilities found in Chapter \ref{cha:Mex1} using this alternative identification strategy. Further, it extends the number of investigated outcomes to three, adding wages and working hours.

\subsubsection{Do the effects of diabetes change over time?}

Diabetes is a lifelong disease whose debilitating complications generally appear after several years of elevated blood glucose levels \parencite{WorldHealthOrganization2016}. So far, little is known about the exact time after diagnosis diabetes starts exhibiting potential adverse effects on labour market outcomes. However, in order to design strategies to mitigate the economic impact of diabetes this would be important to know as it would help in finding the most efficient point in time to intervene. If effects occur immediately after diagnosis, it may be because severe complications are already present at the point of diagnosis, leaving little possibilities to prevent the economic burden. This would suggest that much could be prevented by an earlier diagnosis and appropriate treatment and lifestyle changes. It could further indicate a potential effect of the diagnosis itself, for example on psychological health, causing reductions in employment probabilities or wages. However, if effects appear only years after the diagnosis, severe diabetes complications that have developed due to sub-optimal blood glucose management may be causing the reductions in productivity. This could hint at a possibility to mitigate the negative economic consequences of diabetes by secondary prevention through better diabetes management, even without an earlier diagnoses. The systematic review in Chapter \ref{cha:review} showed a lack of evidence in this area. Only one study by \textcite{Minor2013} investigated the long term consequences of diabetes, finding non-linear effects in a USA population. Apart from the need for additional evidence, also several possibilities for methodological improvements exist. Part two of Chapter \ref{cha:Mex2} therefore assesses the impact of the time since diagnosis on labour market outcomes, using both linear and non-linear specifications in a \ac{FE} framework. 

\subsubsection{Measurement of diabetes in household surveys}

There are two possibilities of measuring diabetes in household surveys: (1) asking participants about their diabetes status or (2) identifying people with diabetes using biometric tests, such as fasting blood glucose or \ac{HbA1c} levels. Using self-reported information likely leads to the exclusion of a considerable part of the diabetes population that has not yet received a diagnosis by a health care professional \parencite{Beagley2014}. Using biomarker information, also previously undiagnosed cases can be identified. Blood glucose measurements provide information on glucose levels at the time of the blood draw but it is not possible to infer on glucose levels over time. They are also sensitive to food consumption and may lead to false positives if taken in a non-fasted state. \ac{HbA1c} levels provide an indication of the average blood glucose levels over the preceding three months and are not sensitive to the glucose level at the time of the blood draw \parencite{WorldHealthOrganization2011}. They are, however, sensitive to an array of disorders such as haemoglobinopathies, anaemias, and disorders associated with accelerated red cell turnover \parencite{WorldHealthOrganization2011}. The cut-off points for diabetes detection for blood glucose measurement and \ac{HbA1c} measurement are 126 mg/dl and 6.5\%, respectively \parencite{WorldHealthOrganization2006,WorldHealthOrganization2011}.

Unfortunately, and largely due to data limitations, previous research had to rely mainly on self-reported diabetes information. It has therefore remained unclear if the found effects also extended to the diabetes population unaware of its condition. Part 3 of Chapter \ref{cha:Mex2} uses a relatively large sample of biomarker data with \ac{HbA1c} measurements, made available in wave 3 of the \ac{MxFLS} that was released in 2015, to investigate the extent of the undiagnosed population in Mexico and the association of diabetes with labour market outcomes for the entire and undiagnosed diabetes population. This part also addresses the question if current disease severity, as proxied by \ac{HbA1c} levels, is related to labour market outcomes. 

Overall, the three parts of Chapter \ref{cha:Mex2} provide extensive additional evidence to answer research question two, by providing evidence of the effect of diabetes on employment probabilities using an alternative estimation strategy compared to Chapter \ref{cha:Mex1}, extending the investigated outcomes to wages and working hours and providing evidence on the effects of diabetes duration. Finally, it investigates heterogeneities in the effects of diabetes for the entire diabetes population, i.e., those aware as well as those unaware of their condition.

\subsection{Diabetes, behavioural risk factors and employment status}

Previous research on the impact of diabetes on employment has assumed a non-dynamic relationship between diabetes and employment probabilities, with diabetes affecting employment but not employment affecting diabetes. This, however, may be a too restrictive assumption, for example if employment status affects behavioural risk factors such as smoking, alcohol consumption or weight that can affect the likelihood of developing diabetes. However, simply accounting for these risk behaviours in a non-dynamic framework may also lead to biased estimates as it is likely that these risk factors themselves are affected by a diabetes diagnosis as people try to live healthier to prevent further diabetes complications or through the effects of medications. This also makes it impossible to account for the potential effect of obesity on labour market outcomes when trying to identify the causal effect of diabetes in such a framework.

These behavioural risk factors also themselves represent an important outcome to investigate, given that there is evidence that the adverse impact of diabetes could be at least partly prevented by changes in lifestyle and appropriate treatment \parencite{Wareham2016}. This would require a diagnosis of diabetes, in order to create awareness of the disease. As Chapter \ref{cha:Mex2} has shown for Mexico, a large part of the diabetes population is unaware of its condition, whether in \acp{HIC} or developing countries \parencite{Beagley2014}. But even once a diagnosis has been made, appropriate changes towards a healthier lifestyle and medical treatment are required to prevent complications and are only possible if the type of information about ways to achieve this is accessible to and understood by the person with diabetes. This information is typically provided by a healthcare professional at the time of diagnosis and thereafter. Relatively little is known about the extent to which people with diabetes are making such changes after a diagnosis, especially in \acp{MIC} where healthcare access and health literacy is likely more limited than in \acp{HIC} \parencite{Mills2014}.

Research study three in Chapter \ref{cha:China} investigates the effect of a diabetes diagnosis on both employment probabilities and health behaviours in China, using six waves of very detailed panel data from the \ac{CHNS}. China, like Mexico, is a country where diabetes rates have increased dramatically over the last decades, now affecting about 100 million people or close to 10\% of the adult population \parencite{Risk2016}, with many remaining unaware of having the condition \parencite{Wang2015}. In a first step to take into account the potential interrelatedness of diabetes, employment status and behavioural risk factors, the study uses \acp{MSM}, which are able to account for time-variant confounding. This strategy allows adjusting for the fact that behavioural risk factors and also employment status could be causes as well as effects of diabetes, which cannot be distinguished with traditional econometric methods such as \ac{OLS} or \ac{FE}. To further investigate the potential sources of bias and robustness of the results also a \ac{FE} and \ac{RE} approach are used. This chapter intends to answer research question three by providing evidence on the effect of a diabetes diagnosis on behavioural risk factors and by taking into account the potential relationship with employment as well. It thereby also provides further evidence to answer research question two, using a different estimation strategy and information from a different country, and also suggests that future research should try to model employment and health behaviours simultaneously to uncover the underlying pathways through which they may affect each other. 


\section{Thesis methods and structure}

This research uses systematic review and advanced quantitative methods to answer the research questions that together form this thesis.

A series of four independent research studies form this thesis. Chapters \ref{cha:review} and \ref{cha:Mex1} have already been published as journal articles and Chapter \ref{cha:Mex2} has been published as a discussion paper and has been submitted to an international peer reviewed journal the time of completion of the thesis. Chapter \ref{cha:China} will be submitted within the next months. This is outlined in more detail in the publication and statement of ownership section. Each study addresses different research questions, but has the investigation of the labour market impact of diabetes as a unifying theme. Taken together the studies progressively complement each other, providing a better understanding of the economic impact of diabetes in \acp{MIC}. Each study is presented in a separate chapter. For Chapters \ref{cha:Mex1}, \ref{cha:Mex2} and \ref{cha:China}, a pre-amble precedes the actual study to contextualize the respective findings with the preceding chapter and the entire thesis.


