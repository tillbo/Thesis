
\begin{itemize}

\item Set stage describing burden of chronic disease/diabetes in world and MICs (Mexico/China) more specifically.  (e.g. burden of disease study/high level studies).
\item Describe general goal of thesis:
\subitem Identify gaps in literature on the economic burden of diabetes in terms of evidence but also methodology, particularly in MICs, and fill some of the gaps.
\item Describe each of the chapters and the motivation behind it
\subitem 
\end{itemize}

\section{Background to the thesis}

Diabetes, and especially type 2 diabetes, has seen an unprecedented rise in prevalence in \acp{LMIC} \parencite{Risk2016}. This rise has been much greater than in \acp{[HIC} such as the USA, UK or Western Europe. This increase can only partly be explained by a shift in age structure towards older populations. Especially in \acp{LMIC} it appears to be driven by rapid changes in levels of physical activity, in nutrition and other lifestyle related factors. 

Diabetes is a term used to describe various conditions characterised by elevated blood glucose levels. These either occur because the pancreas is not able to produce sufficient insulin or due to insulin resistance, where the body is not able to use the produced insulin effectively \parencite{WorldHealthOrganization2016}. The different conditions themselves, however, have distinct origins, especially for the two most common types of type 1 diabetes and type 2 diabetes. 

\begin{itemize}
\item Type 1 diabetes is an autoimmune disease with an important genetic component whose triggers still remain largely elusive. It emerges when the insulin producing cells on the pancreas are attacked and destroyed by the own immune system, so that insulin has to be provided exogenously. About 10\% of all global diabetes cases are type 1 diabetes and it is particularly prevalent in Northern European countries such as Finland, and generally exhibits large geographic variation. Its onset is mainly in early childhood,teenage years and early adulthood. Symptoms tend to appear rather quickly and can be quite severe leading to a relatively rapid diagnosis or death. People with type 1 diabetes will need to inject insulin to control their blood glucose levels. If access to insulin is not given type 1 diabetes leads to death within a short period of time \parencite{Tuomilehto2013}. 
\item Type 2 diabetes results from the body's ineffective use of insulin and accounts for about 90\% of all diabetes cases \parencite{WorldHealthOrganization2016}. Albeit there is a considerable genetic component to the development of type 2 diabetes, there are many known risk factors that favour the development of type 2 diabetes, such as overweight and obesity, unhealthy diet, physical inactivity and smoking, among others \parencite{WorldHealthOrganization2016}. Interestingly, the risk of developing type 2 diabetes varies also by population, with South-East Asian populations developing diabetes at lower \ac{BMI} levels than populations of European decent \parencite{Ramachandran2010}. Type 2 diabetes often remains undetected for several years due to its more gradual development compared with type 1 diabetes. Therefore, even in \acp{HIC} and especially in \acp{LMIC}, a considerable proportion of at least 1/4 of the population with type 2 diabetes is unaware of the condition \parencite{Beagley2014}. Further, even after the diagnosis in many cases blood glucose levels are not successfully managed as to prevent further complications \parencite{Villalpando2010,DiabetesUK2012}.

Recently, studies also have observed an earlier onset of type 2 diabetes, especially in populations that have a higher genetic susceptibility such as Mexicans with some indigenous roots and South-East Asians. Further, the increasing numbers of obesity and overweight in childhood and early adulthood have also likely caused an earlier onset of type 2 diabetes. Therefore type 2 diabetes increasingly affects people in the middle of their productive lifespan, increasing the time they have to live with the disease and the probability of developing debilitating complications.

The most common complications for all types of diabetes are an increase in the risk of heart disease and stroke with 50 percent of the people with diabetes dying from cardiovascular disease. Reduced blood flow and neuropathy in the feet can lead to foot ulcers and may make amputation of the limps necessary. A very common complication, often already present at diagnosis is rethinopathy (find some prevalence number and reference) . Overall, 2\% of the people with diabetes are blind after 15 years of illness and about 10\% have to cope with severe visual impairments. Kidney failure is responsible for 10--20 percent of deaths in people with diabetes. Nerve damage can be observed in 50 percent of all people with diabetes, often resulting in tingling, pain, numbness, or weakness in the feet and hands. In general, the overall risk of dying is about double the risk of people without diabetes [WHO2009, InternationalDiabetesFederation2009]. Further, there is a growing literature suggesting a significant relationship between diabetes and depression with one disease affecting the other [Pan2010a]. In addition, there seems to be a link between diabetes and the development of certain types of cancer, were diabetes increases the risk of developing cancer [Giovannucci2011a, Chowdhury2010]. Accordingly, diabetes not only causes vascular diseases but is also related to a wide array of other diseases [Seshasai2011]. UPDATE REFERENCES AND TEXT

WHAT ARE CONSEQUENCES. TALK ABOUT PREVENTION POSSIBILITIES. NAME OTHER TYPES OF DIABETES SHORTLY. GO OVER TO THE ECONOMIC BURDEN ASSOCIATED WITH DIABETES (MAINLY KNOWN IN HICS AND ONLY FOR HEALTHCARE COSTS. TO PREVENT AND TREAT DIABETES IT IS IMPORTANT TO IDENTIFY MOST AFFECTED, ALSO ECONOMICALLY SET STAGE FOR REVIEW.)
\end{itemize} Therefore, elevated blood glucose levels can cause damage to organs and blood vessels long before a diagnosis.

write about increase in child obesity and type 2 diabetes
common complications of diabetes, differences in population susceptability to type 2 diabetes

