
\begin{itemize}

\item Set stage describing burden of chronic disease/diabetes in world and MICs (Mexico/China) more specifically.  (e.g. burden of disease study/high level studies).
\item Describe general goal of thesis:
\subitem Identify gaps in literature on the economic burden of diabetes in terms of evidence but also methodology, particularly in MICs, and fill some of the gaps.
\item Describe each of the chapters and the motivation behind it
\subitem 
\end{itemize}

\section{Background to the thesis}

Diabetes, and especially type 2 diabetes, has seen an unprecedented rise in prevalence in \acp{LMIC}. This rise has been much greater than in \acp{HIC} such as the USA, UK or Western Europe and can only partly be explained by a shift in age structure towards older populations. Especially in \acp{LMIC} it appears to be driven by rapid changes in levels of physical activity, in nutrition and other lifestyle related factors  \parencite{Risk2016,Hu2011}.

The transition towards \acp{NCD} in \acp{LMIC}, including diabetes, has taken place rapidly over the last three decades and has lead in many places to a double disease burden, i.e. health systems having to deal with both communicable and \acp{NCD}. So far countries have had little success in halting the increase in diabetes, so that by now the majority of people with diabetes lives in middle-income countries, in particular in China, India, Brazil, Indonesia, Pakistan, Russia, Egypt and Mexico \parencite{Risk2016}. Despite this increase in diabetes in less developed countries over the last decades, research on its economic consequences had been limited mainly to \acp{HIC}.

 
\subsection{Types of diabetes}

Diabetes is a term used to describe various conditions characterised by elevated blood glucose levels. These either occur because the pancreas is not able to produce sufficient insulin or due to insulin resistance, where the body is not able to use the produced insulin effectively \parencite{WorldHealthOrganization2016}. The different conditions themselves, however, have distinct origins, especially for the two most common types of type 1 diabetes and type 2 diabetes. 

\begin{itemize}
\item \textbf{Type 1 diabetes} is an autoimmune disease with an important genetic component and whose triggers still remain largely elusive. It emerges when the insulin producing cells on the pancreas are attacked and destroyed by the immune system, so that insulin has to be provided exogenously. About 10\% of all global diabetes cases are type 1 diabetes and it is particularly prevalent in Northern European countries such as Finland, and generally exhibits large geographic variation. Its onset is mainly in early childhood,teenage years and early adulthood. Symptoms tend to appear rather quickly and can be quite severe leading to a relatively rapid diagnosis or death. People with type 1 diabetes will need to inject insulin to control their blood glucose levels. If access to insulin is not given type 1 diabetes leads to death within a short period of time \parencite{Tuomilehto2013}. 
\item \textbf{Type 2 diabetes} results from the body's ineffective use of insulin and accounts for about 90\% of all diabetes cases \parencite{WorldHealthOrganization2016}. Albeit there is a considerable genetic component to the development of type 2 diabetes, there are many known risk factors that favour the development of type 2 diabetes, such as overweight and obesity, unhealthy diet, physical inactivity and smoking, among others \parencite{WorldHealthOrganization2016}. Interestingly, the risk of developing type 2 diabetes varies also by population, with South-East Asian populations developing diabetes at lower \ac{BMI} levels than populations of European decent \parencite{Ramachandran2010}. Type 2 diabetes often remains undetected for several years due to its more gradual development compared with type 1 diabetes. Therefore, even in \acp{HIC} and especially in \acp{LMIC}, a considerable proportion of at least 1/4 of the population with type 2 diabetes is unaware of the condition \parencite{Beagley2014}. 
\end{itemize}

Recently, an earlier onset of type 2 diabetes has been observed, especially in minorities in \acs{HIC}, such as Mexicans and Asian populations, while data is limited for \ac{LMIC} \parencite{FazeliFarsani2013}. Further, the increasing numbers of obesity and overweight in childhood and early adulthood have also likely caused an earlier onset of type 2 diabetes \parencite{Chen2012}. Hence, type 2 diabetes increasingly affects people in the middle of their productive lifespan, extending the time they have to live with the disease and probability of developing debilitating complications.

\subsection{Diabetes complications}

The most common complications for all types of diabetes and often already present at diagnosis is retinopathy being present in 35\% of people with diabetes and responsible for 2.6\% of blindness globally. Further, up to 50\% of cases of end stage renal disease are a direct result of diabetes, especially in countries where access to dialysis is restricted. People with diabetes also have a 2--3 times higher risk to experience cardiovascular disease compared to people without diabetes. A further, very debilitating, complication is amputation of lower limps due to impaired wound healing, being 10--20 higher than for people without diabetes \parencite{WorldHealthOrganization2016}. There is also a growing literature suggesting a---potentially bidirectional---relationship between diabetes and depression \parencite{VanDooren2013,Nouwen2010,Roy2012}. In addition, there seems to be a link between diabetes and the development of certain types of cancer, \parencite{Tsilidis2015,Nead2015}, as well as an array of other of other infectious diseases, intentional self-harm and degenerative disorders diseases \parencite{Seshasai2011}.


\subsection{Diabetes prevention}

While a causal relationship between type 2 diabetes, depression and cancer has not yet been established, most of the other complications are a result of consistently elevated blood glucose levels. Hence many diabetes cases could be prevented if recommended treatment goals were achieved. However, limited resources and access to healthcare make it difficult to properly treat type 2 diabetes in \acp{LMIC}, and even in \acp{HIC}, a large part of the diabetes population does not achieve treatment goals. Further, even after the diagnosis in many cases blood glucose levels are not successfully managed as to prevent further complications \parencite{Villalpando2010,DiabetesUK2012}. 

Further, there is also scope for the primary prevention of diabetes, in particular of type 2 diabetes, by reducing the prevalence of the known risk factors such as obesity, an unhealthy diet, smoking and sedentary behaviour \parencite{WorldHealthOrganization2016}. However, so far most approaches to prevent type 2 diabetes have not had the desired effect and may not always be realistic in very resource constrained settings \parencite{White2016}. In particular, efforts to reduce the biggest type 2 diabetes risk factors of obesity and overweight have mostly fallen flat \parencite{Roberto2015}.

\subsection{The need for further economic research on diabetes}

To provide good research to aid qualified decision about the use of primary and secondary prevention strategies of diabetes, researchers and policy makers need information about the current burden of disease, both in terms of health and economically, that is caused by diabetes and could be realistically prevented. Information on all aspects of economic costs and the quality of the estimates has optimally to be available. However, at the start of this thesis, little was known about the global economic impact of diabetes, and especially in developing countries. There had never been a comprehensive systematic review of studies assessing the costs related to diabetes, both in terms of direct and indirect costs. Only one (non-systematic) review existed by \textcite{Ettaro2004}, including studies on the \ac{COI} until the year 2001. They did,however, not find research from \acp{LMIC}. Further, the methodological quality of existing research had not been comprehensively assessed and areas of future research remained unidentified. Also missing was a review on studies not using a \ac{COI} approach but using quantitative methods to estimate the impact of diabetes on labour market outcomes, such as employment and wages.

These gaps in evidence form research question one and are addressed in Chapter \ref{cha:review}: \textit{The Economic Costs of Type 2 Diabetes: A Global Systematic Review}. The review had several goals. One was to provide a first comprehensive global picture of the economic burden of type 2 diabetes, not limited to traditional \ac{COI} studies but also including studies on the labour market effects of diabetes. It was also expected to find evidence on the economic costs of diabetes in developing countries. Together, the aim was to provide information on the economic costs of diabetes for as many countries as possible. Another goal was the identification of areas, both in terms of methodology and topic, where evidence was lacking and/or current methodologies could be improved upon. This should help me determining the subsequent chapters of my thesis as well as other researchers interested in researching the economics of diabetes.

\subsection{The labour market impact of type 2 diabetes}

The review identified the labour market impact of diabetes in \acp{LMIC} as a topic that had not received much attention. Apart from the lack of evidence from developing countries, there was also scope for methodological improvements compared to the existing \ac{HIC} evidence. Further, information on the effect on sub-populations, i.e. comparisons between rich and poor and the formal and informal labour market were non-existent.

However, in order to carry out such an analysis, appropriate data needed to be identified. To this end I carried out an internet search, using general search engines as well as specialized engines such as the World Bank Central Microdata Catalog  \url{http://microdata.worldbank.org/}, the Demographic and Health Survey Database \url{http://dhsprogram.com/data/},the Global Health Data Exchange Database \url{http://ghdx.healthdata.org/}, and the International Household Survey Network Catalog \url{http://catalog.ihsn.org/index.php/catalog} in particular searching for datasets containing information on self-reported or measured diabetes. Specialized websites providing an overview on household survey data in developing countries were also searched to identify relevant data (such as \url{http://ipl.econ.duke.edu/dthomas/dev_data/index.html} and \url{https://sites.google.com/site/medevecon/development-economics/devecondata/micro} for household survey from developing countries, and an overview on data sets containing biomarker information provided by The Biomarker Network at \url{http://gero.usc.edu/CBPH/network/resources/studies/}). An overview of the identified studies is provided in Table \ref{tab:datasets}.

Given the availability of data and the extend of diabetes in \acp{MIC} compared to \acp{LIC}, a decision was made to focus on \acp{MIC} for the remained of the thesis. In particular, Mexico shall be the country of interest for Chapters \ref{cha:Mex1} and \ref{cha:Mex2}. The main reason to chose Mexico is the availability of data, provided by the \ac{MxFLS}. It allows for the investigation of the impact of diabetes on labour market outcomes by providing high quality information on a rich set of important covariates, including family background and diabetes itself, not available in other surveys. Further, Mexico is a country with particularly high obesity and diabetes rates making it an interesting case to study. Chapter \ref{cha:Mex1} therefore investigates \textit{The impact of diabetes on employment in Mexico}.The goal was to provide an answer to research question two, what is the causal effect of self-reported diabetes on employment probabilities in a \ac{MIC}, here in the case of Mexico?

\subsection{Identification of the causal effect of diabetes on labour market outcomes}

As eluded to in Chapter \ref{cha:Mex1}, identifying a causal relationship of diabetes with labour market outcomes is being complicated by the possibility of unobserved time-variant and -invariant heterogeneity. In Chapter \ref{cha:Mex1}, an \ac{IV} approach is used, though as with all \ac{IV} it cannot be tested if it is truly exogeneous, leaving the possibility of biased estimates. Several other strategies potentially exist to identify the true effect of diabetes on labour market outcomes using quasi-experimental econometric approaches \parencite{Antonakis2012}. For example, a natural experiment may be used that would affect people's diabetes risk while at the same time have no direct effect on labour market outcomes such as employment probabilities or wages. However, exogeneously introduced variation may be difficult to identify and may only provide information for a very---often geographically or economically---selected population that has been exposed to this natural experiment. Another strategy to improve inference is the use of panel data and in particular the \ac{FE} model, which does not depend on some external exogeneousely introduced variation. It allows the elimination of all time-invariant factors that may affect diabetes and labour market outcomes simultaneously. This may be particularly fruitful in the case of diabetes and economic outcomes, where the use of \ac{IV}s has been motivated by the possibility that unobserved character trades---generally though to be stable over time---such as motivation as well as early life experiences may be confounding the relationships.

Therefore, part one of Chapter \ref{Mex2}, uses a recent addition of data to the \ac{MxFLS} to apply a \ac{FE} estimation approach, testing if the effects of diabetes on employment probabilities remain using this alternative, and arguably, more credible identification strategy. Further, it extends the number of investigated outcomes to 3, adding wages and working hours.

\subsection{Do the effects of diabetes change over time?}

Diabetes is a lifelong disease whose debilitating complications generally appear after several years of elevated blood glucose levels. Therefore, it may be reasonable to expect that any adverse labour market effects of diabetes appear after several years of living with the disease. In order to design strategies to mitigate the economic impact of diabetes, it is important to understand at which point after diagnosis these effects appear. If they appeared immediately after diagnosis, it may be because severe complications have already appeared at the point of diagnosis, leaving little possibilities to prevent the economic burden. This could suggest that much could be prevented by an earlier diagnosis. It could further indicate a potential effect of the diagnosis itself, for example on psychological health, decreasing employment probabilities or wages. If effects appear ears after the diagnosis, this could suggest that severe diabetes complications have developed due to sub-optimal blood glucose management, causing reductions in productivity. This would also hint to the possibility to mitigate the negative economic consequences of diabetes by secondary prevention through better diabetes management, even without earlier diagnoses. The systematic review in Chapter \ref{cha:review} showed a lack of evidence in this area. Only one study by \textcite{Minor2013} investigated the long term consequences of diabetes, finding non-linear effects in a USA population. However, apart from the need for additional evidence, several possibilities for methodological improvements exist. Part two of Chapter \ref{cha:Mex2} therefore assesses the impact of diabetes duration, or time since diagnosis, on labour market outcomes, using both linear and non-linear specifications in a \ac{FE} framework.   

\subsection{Measurement of diabetes in household surveys}

There are two possibilities of measuring diabetes in household surveys: (1) asking participants about their diabetes status or (2) trying to identify people with diabetes using biometric exams, such as fasting blood glucose levels or \ac{HbA1c} exams. Using self-reported information likely leads to the exclusion of a considerable part of the diabetes population that has not yet received a diagnosis by a health care professional. Using biomarker information, also those "undiagnosed" cases can be identified, however, it might miss cases where diabetes is present but well managed with glucose levels below the accepted diagnosis thresholds. Blood glucose measurements provide information on blood glucose levels at the time of measurement but it is not possible to infer on longer term blood glucose levels. They are also sensitive to food consumption and may lead to false positives if taken in a non-fasted state. \ac{HbA1c} measurement provide an indication of the average blood glucose levels over the preceding 3 months and are not sensitive to the blood glucose level at the time of the blood draw. They are, however, sensitive to an array of disorders such as haemoglobinopathies, anaemias, and disorders associated with accelerated red cell turnover \parencite{WorldHealthOrganization2011}. The cut-off points for diabetes detection for blood glucose measurement and \ac{HbA1c} measurement are 126 mg/dl and 6.5\%, respectively \parencite{WorldHealthOrganization2006,WorldHealthOrganization2011}.

Unfortunately, and largely due to data limitations, previous research had to rely mostly on self-reported diabetes information. It has there fore remained unclear if the found effects also extend to the diabetes population unaware of its condition. Part 3 of Chapter \ref{cha:Mex2} uses rich biomarker data with \ac{HbA1c} measurements, made available in wave 3 of the \ac{MxFLS} released in 2015, to investigate the extend of the undiagnosed population in Mexico and in how far results using self-reported diabetes extend to this population. This part also addresses the question in how far current disease severity, as proxied by \ac{HbA1c} levels, is related to labour market outcomes.

\subsection{The effect of health information provided by a diabetes diagnosis}

The adverse impact of diabetes could be prevented by changes in lifestyle and appropriate treatment. A prerequisite to this is a diagnosis of diabetes. As Chapter \ref{cha:Mex2} and has shown, a large population of all people with diabetes is unaware of their condition, likely also in other developing countries. But even once a diagnosis has been made, the person with diabetes needs to be able to make the appropriate changes towards a healthier behaviour. This is only possible if this information is accessible to and understood by the person with diabetes, i.e. it has been provided by a healthcare professional at diagnosis or thereafter and the person is capable of making the proposed changes. Relatively little is known about the extend to which people with diabetes are making such changes after a diagnosis, especially in \acp{LMIC} where healthcare access is likely more limited than in \acp{HIC}.

China, similar to Mexico, is a country where diabetes rates have increased dramatically over the last decades, now affecting about 100 million people or close to 10\% of the adult population \parencite{Risk2016}. A large part of that population is not yet diagnosed \parencite{Wang2015}. For those that are, studies on health literacy show that those that have received some diabetes information also achieve better blood glucose control  and have better knowledge of beneficial health behaviours \parencite{Guo2012}. However, less is known about the actual impact of a diagnosis on long term health behaviours and risk factor reduction such as smoking, alcohol consumption and weight management. Given the number of people with diabetes in China potentially small long term changes in these behaviours just as a result of the information gained through diagnosis and subsequent treatment could make an important contribution to prevent the burden of diabetes. So far, only a study for the USA investigated a similar question, finding mostly short lived reductions in risk behaviours \parencite{Slade2012}. 

Chapter \ref{cha:China} intents to answer research question 3 "The effect of a diabetes diagnosis on health behaviours and economic outcomes in China", using several waves of very detailed panel data. Because selection into diagnosis is likely related to socioeconomic characteristics and healthcare access as well as health behaviours, it is important to account for this by using appropriate econometric techniques. In this chapter, selection bias is accounted for using two strategies. First with a \ac{FE} approach to account for any time-invariant confounding, and second, using \acp{MSM} to prevent selection bias due to time-variant confounders. These strategies should help to get closer to a causal estimate. Additionally, further evidence for the impact of diabetes on employment probabilities is provided.


