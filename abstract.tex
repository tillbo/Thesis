This thesis researches the economics of type 2 diabetes in \acfp{MIC}. Given its rising prevalence, in-depth country specific analysis is key for understanding the economic consequences of T2D in \acp{MIC}. I analyse the economic burden of type 2 diabetes in terms of labour market consequences, taking into account the heterogeneity of the diabetes population, for both Mexico and China. For China I further investigate the effects of a diabetes diagnosis on health behaviours.

The thesis consists of four studies with the unifying theme of improving our understanding of the causal impact of diabetes on  economic outcomes. Study (1) provides an updated overview, critically assesses and identifies gaps in the current literature on the economic costs of T2D using a systematic review approach; study (2) investigates the effects of self-reported diabetes on employment probabilities in Mexico, using cross-sectional data and making use of a commonly used \acf{IV} approach; study (3) revisits and extends these results via the use of \ac{FE} panel data analysis, also considering a broader range of outcomes, including wages and working hours. Further, it makes use of cross-sectional biomarker data that allow for the investigation of measurement error in self-reported diabetes. Study (4) researches the effect of a diabetes diagnosis on employment as well as behavioural risk factors in China, using longitudinal data and applying an alternative identification strategy, \acf{MSM} estimation, while comparing these results with \ac{FE} estimation results.\footnote{Chapters \ref{cha:review} and \ref{cha:Mex1} have appeared as journal articles in \textit{PharmacoEconomics} and \textit{Economics \& Human Biology}, respectively. Chapter \ref{cha:Mex2} has appeared as a working paper and been submitted for publication. For further details see page \pageref{publication_statement}.}

The findings of the first paper document a considerable increase in studies on the economic costs of diabetes in \acp{MIC}. It also illustrates that most of the evidence is based on cost-of-illness studies and the literature on adverse labour market effects of diabetes in \acp{MIC} is scarce. The thesis fills part of this void and shows that self-reported diabetes has a considerable impact on employment probabilities of people living in Mexico. The findings are robust to the application of different estimation strategies. No consistent evidence of an adverse effect of diabetes on wages or working hours is found. The findings for Mexico in the second and third paper suggest considerable inequities in the adverse labour market effects of diabetes, as it mainly affects the poor, less protected as well as women. Taking into account those unaware of the disease, the adverse effect of diabetes is reduced because undiagnosed diabetes itself does not show an adverse association with any labour market outcome. This suggests that the undiagnosed population is distinctly different from the diagnosed population, likely due to differences in health status and health information. The results from the fourth paper display strong discrepancies in the effect of diabetes on both employment probabilities and behavioural risk factors. For men, employment appears not to be affected by the diagnosis and they are able to reduce their levels of \ac{BMI}, waist circumference, alcohol and caloric consumption. Women, however, experience important reductions in employment probabilities but do not experience strong post-diagnosis reductions in behavioural risk factors. These sex differences in risk factor reductions may provide an explanation for the more adverse employment effects for women. Importantly, accounting for confounding appears to be of particular importance when estimating the effects on \ac{BMI} and waist circumference.

Overall the thesis identifies a considerable economic burden of diabetes in \acp{MIC} and uncovers several inequities. To reduce this burden the groups most affected by these inequities should be targeted. Further research will be needed to identify the underlying reasons for the found sex differences.

