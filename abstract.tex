This thesis focuses on the economic analysis of type 2 diabetes (T2D) in middle-income countries. Given its rising prevalence, in-depth country specific analysis is key for understanding the economic consequences of T2D in middle-income countries (MICs). I analyse the economic burden of T2D in terms of labour market consequences, taking into account the heterogeneity of the diabetes population, for both Mexico and China. For China I further investigate the effects of a diabetes diagnosis on health behaviours that may help to curb the adverse consequences of diabetes.

The thesis consists of four essays with the unifying theme of improving our understanding of the causal relationship between diabetes and economic outcomes. Essay (1) provides an updated overview, critically assesses and identifies gaps in the current literature on the economic costs of T2D using a systematic review approach; essay (2) studies the effect of self-reported diabetes on employment probabilities in Mexico, using cross-sectional data and making use of a commonly used instrumental variable approach; essay (3) extends the previous essay via the use of panel data and fixed effects and considering a broader range of outcomes, including wages and working hours; it also makes use of cross-sectional biomarker data that allows for the investigation of measurement error in self-reported diabetes; essay (4) investigates the effect of a diabetes diagnosis on employment and income as well as health behaviours in China, using longitudinal data and applying two distinct identification strategies: fixed effects and marginal structural model estimation.

The findings of the first paper document a considerable increase in studies on the economic costs of diabetes in MICs. It also illustrates that most of the evidence is based on cost-of-illness studies and the literature on labour market and potential earning effects of diabetes in MICs is scarce. The thesis fills part of this void and shows that self-reported diabetes has a considerable impact on employment probabilities of people living in Mexico and China. The findings are robust to the application of different estimation strategies. No consistent evidence of an adverse effect of diabetes on wages or working hours is found, suggesting that diabetes mainly affects the extensive margin. The findings for Mexico indicate that particularly people working in the informal or agricultural, hence less protected and often more physically demanding, sectors bear the brunt of the negative effects of diabetes. Taking into account the undiagnosed population, the adverse effect of diabetes is reduced because undiagnosed diabetes itself does not show an adverse association with any labour market outcome. This suggests that the undiagnosed population is distinctly different from the diagnosed population, likely due to differences in health information and health status. Therefore, research using self-reported diabetes information should limit its claims to the diagnosed population as economic effects are likely different for the undiagnosed. With regards to the effect of a diabetes diagnosis on health behaviours, the results from China suggest that a diagnosis leads to moderate reductions in body mass index (BMI), waist circumference, alcohol and caloric consumption. Perhaps surprisingly, especially men appear to be able to lose weight and reduce their caloric consumption. Not accounting for unobserved heterogeneity leads to a change in the coefficient sign for the effect of a diagnosis on BMI and waist circumference, while the differences in estimates are less pronounced for other outcomes.



\comment{Background: This thesis focuses on the economic analyses of type‐2 diabetes complications defined as macro-vascular (myocardial infarction, stroke, ischemic heart disease, heart failure) and micro‐vascular (amputation and eye-related complications leading to blindness in one eye). Diabetes-related complications are a substantial component of the overall economic, physical and psychological burden of the disease. As the efforts in treating diabetes are geared towards reducing the likelihood of complications, understanding the welfare benefits and future savings from reducing diabetes complications is paramount in determining the cost-effectiveness of competing diabetes therapies.
Aims: The thesis is divided into three essays aiming to (1) characterize changes in the health related quality of life of diabetes patients over time and assess the contributions of diabetes complications to these changes; (2) study the drivers of healthcare expenditure for people with diabetes in terms of both inpatient care and non-inpatient resource utilization, and estimate the impacts of diabetes-related complications on health care costs; (3) understand the role played by self-reported quality of life in predicting mortality after controlling for clinical risk factors.
Methods: This thesis uses longitudinal data to answer the questions of interest. A unifying theme across the thesis is the challenge of estimating causal parameters in a context in which there may be substantial observed and unobserved patient heterogeneity.
Findings: Failing to account for patient heterogeneity, and in particular un-measurable variation in patients’ outcomes, is likely to bias the impact of complications on quality of life and on non-inpatient costs, as well as to confound predicted time to death. In the case of QoL, ignoring heterogeneity is likely to overestimate the impact of complications on self reported utility because the patients who will eventually experience diabetes-related complications are already on a lower utility path compared to those who do not. In the case of both inpatient and non-inpatient costs, patients who go on to develop complications have higher cost both pre and post complications. In the case of inpatient costs there is no evidence that unobserved patient heterogeneity matters, while in the case of non-inpatient utilization the hypothesis of a common baseline level of utilization is rejected in the subset of patients that contribute to the FE identification. This subset however is systematically different from the sample as a whole, being predominately more likely to have complications and other causes of hospitalization. Moreover, a trade-off occurs when we are interested in predictions; models that exploit within-patient variation have wider confidence intervals and have thus less precision than population average models. The final substantive chapter finds that HRQoL is significantly associated with survival at the population level and that when patient specific unobserved heterogeneity is taken into account, the power of QoL to predict life expectancy increases. Neglected heterogeneity in frailty causes underestimation of both the extent of positive duration dependence and the impacts of time varying covariates.}
