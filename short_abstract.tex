\documentclass[12pt,english]{article}
\usepackage[affil-it]{authblk}
\usepackage[a4paper,%
            left=4cm,right=2.5cm,top=2cm,bottom=2.5cm,%
            footskip=.25in]{geometry}
\begin{document}
%\begin{acronym}
\acro{2SLS} {two-stage least squares}
\acro{ATE} {average treatment effect}
\acro{ATT} {average treatment effect on the treated}
\acro{AUD} {Australian Dollar}
\acro{BLUE} {best linear unbiased estimator}
\acro{BMI} {body mass index}  
\acro{BP} {bivariate probit}
\acro{CHNS} {China Health and Nutrition Survey}
\acro{CHARLS} {The China Health and Retirement Longitudinal Study}
\acro{COI} {cost-of-illness} 
\acro{DAG} {direct acyclic graph}
\acro{DALYs} {disability-adjusted life years}
\acro{EUR} {Europe}
\acro{ENSA} {La Encuesta Nacional de Salud}
\acro{FE} {fixed effects}  
\acro{ENSANUT}{Encuesta Nacional de Salud y Nutricion}
\acro{FGLS} {Feasible General Least Squares}
\acro{GDP} {gross-domestic-product} 
\acro{HbA1c} {glycated hemoglobin}  
\acro{HIC} {high-income country} 
\acro{ICD}{International Statistical Classification of Diseases and Related Health Problems}
\acro{IDF} {International Diabetes Federation}
\acro{INEGI} {Instituto Nacional de Estadistica y Geografia} 
\acro{IV} {instrumental variable}
\acro{LATE} {local average treatment effect}
\acro{LIC} {low-income country} 
\acro{LMIC} {low- and middle-income country} 
\acro{LPM} {linear probability model}
\acro{MSM} {marginal structural model} 
\acro{MENA} {Middle East and North Africa}
\acro{MIC} {middle-income country}  
\acro{MxFLS} {Mexican Family Life Survey}
\acro{NAC} {North American and Caribbean}
\acro{NCD} {non-communicable disease}
\acro{OLS} {ordinary least squares}
\acro{OOP} {out-of-pocket}   
\acro{PML} {pseudo-maximum-likelihood}
\acro{PPP}{purchasing-power-parity}
\acro{PRISMA} {Preferred Reporting Items for Systematic Reviews and Meta-Analyses}
\acro{RE} {random effects}
\acro{SACA} {South and Central America}
\acro{SEA} {South East Asia}
\acro{SSA} {Sub-Saharan Africa}
\acro{UK} {United Kingdom}
\acro{WHO} {World Health Organization}
\acro{WP} {Western Pacific}
\acro{WTP} {willingness to pay}    

\end{acronym}

\acrodefplural{LMIC}[LMICs]{low- and middle-income countries}  

\acrodefplural{HIC}[HICs]{high-income countries}

\acrodefplural{MIC}[MICs]{middle-income countries}
 
\acrodefplural{LIC}[LICs]{low-income countries} 
% Title Page
%\comment{
\makeatletter
\begin{titlepage}
\centering
\vfill
Till Seuring \\
\par
\vspace*{1cm}
2016
\par
\vspace*{2cm}
\begin{Large}\bfseries
The Economics of Type 2 Diabetes in Middle-Income Countries\par
\end{Large}
\vspace{1in}



\begin{abstract}
This thesis researches the economics of type 2 diabetes in middle-income countries (MICs). Given the high prevalence of type 2 diabetes in MICs, in-depth country specific analysis is key for understanding the economic consequences of type 2 diabetes. The thesis consists of four studies with the unifying theme of improving our understanding of the causal impact of diabetes on  economic outcomes. Study (1) provides an updated overview, critically assesses and identifies gaps in the current literature on the economic costs of type 2 diabetes using a systematic review approach; study (2) investigates the effects of self-reported diabetes on employment probabilities in Mexico, using cross-sectional data and making use of a commonly used instrumental variable approach; study (3) revisits and extends these results via the use of a fixed effects panel data analysis, also considering a broader range of outcomes, including wages and working hours. Further, it makes use of cross-sectional biomarker data that allow for the investigation of undiagnosed diabetes. Study (4) researches the effect of a diabetes diagnosis on employment as well as behavioural risk factors in China, using longitudinal data and applying an alternative identification strategy, marginal structural models estimation, while comparing these results with fixed effects estimation results. The thesis identifies a considerable economic burden of diabetes in middle-income countries and uncovers several inequities affecting women, the poor and the uninsured. Women are also found to achieve fewer positive changes of their behavioural risk factors after a diabetes diagnosis, offering a potential explanation for their more adverse employment outcomes compared to men. To reduce the economic  burden, the groups most affected by the identified inequities should be targeted. Further, the underlying reasons for the found sex differences need to be identified.
\end{abstract}
\end{titlepage}
\makeatother
\end{document}