\documentclass[a4paper,12pt,english,abstract=on,titlepage=false]{scrreprt}
\usepackage{setspace} %linespacing
\onehalfspacing
%\setlength{\parskip}{1em} % paragraph spacing
\usepackage[affil-it]{authblk}
\usepackage{graphicx}
\usepackage[space]{grffile}
\usepackage{latexsym}
\usepackage{textcomp}
\usepackage{longtable}
%causes each table to be on seperate page
\renewcommand\floatpagefraction{0.1}
%causes each figure to be on seperate page
%\makeatletter
%\@fpsep\textheight
%\makeatother
%multirow tables etc.
\usepackage{multirow,booktabs}
\usepackage{amsfonts,amsmath,amssymb}
\usepackage{gensymb}  % to show special symbols like degree °
\usepackage{url}
\usepackage{dcolumn} % needed for tables in chapter 5
%\usepackage[utf8]{inputenc}
\usepackage{hyperref}
\hypersetup{colorlinks=false,pdfborder={0 0 0}}
%\usepackage{latexml}
\newcommand{\truncateit}[1]{\truncate{0.8\textwidth}{#1}}
\newcommand{\scititle}[1]{\title[\truncateit{#1}]{#1}}
\usepackage{lmodern}
\usepackage[T1]{fontenc}
\usepackage[utf8]{inputenc}
\usepackage{geometry}
\geometry{verbose,tmargin=2cm,bmargin=2cm,lmargin=4cm,rmargin=2cm}
\providecommand{\tabularnewline}{\\}
\usepackage[printonlyused]{acronym}
%\usepackage[nolist]{acronym}
\usepackage{ltablex,array} % to scale longtables
\usepackage{lipsum}
\usepackage[flushleft]{threeparttable}  %adds notes to tables
\usepackage{etoolbox}% http://ctan.org/pkg/etoolbox
\appto\TPTnoteSettings{\footnotesize}
\usepackage{caption}
\usepackage{chngcntr} %esnures running table numbers
\counterwithout{table}{chapter}
\counterwithout{figure}{chapter}
%fix problem with narrower table once using caption inside tabularx environment as suggested in last response here http://www.latex-community.org/forum/viewtopic.php?f=45&p=39951

\makeatletter
\patchcmd{\TX@endtabularx}% <cmd>
  {\def\caption}% <search>
  {\def\caption{\caption@withoptargs\TX@caption}%
   \def\TX@caption##1##2}% <replace>
  {}{}% <success><failure>
\makeatother


\usepackage{graphicx}
\usepackage{adjustbox}
%\newcommand{\sym}[1]{\ensuremath{^{#1}}} % for symbols in Table
%landscape pages
\usepackage{pdflscape}
\newcommand{\comment}[1]{}  %allows multiline comments
%\usepackage[english]{babel}% Recommended
%\usepackage{csquotes}
\usepackage[
maxcitenames=2, 
style=authoryear-comp,
firstinits=true,
maxbibnames=99,
backend=biber,
uniquename=false,
url=false,
doi=false,
isbn=false]{biblatex}
\DeclareNameAlias{sortname}{last-first}
\DeclareFieldFormat{pages}{#1}%
\renewcommand*{\nameyeardelim}{\addcomma\space} %adds comma between name and years in citations

\renewbibmacro{in:}{}
\renewbibmacro*{volume+number+eid}{%
  \printfield{volume}%
  \setunit*{\adddot}% DELETED
  \setunit*{\addnbspace}% NEW (optional); there's also \addnbthinspace
 \printfield{number}%
 \setunit{\addcomma\space}%
 \printfield{eid}}
\DeclareFieldFormat[article]{number}{\mkbibparens{#1}}
\DeclareLanguageMapping{english}{english-apa}  %needed for APA style
\AtEveryBibitem{
  \clearfield{labelmonth}
}

%patch fullcite command so that full citation in text is same as in references: http://tex.stackexchange.com/questions/43196/biblatex-fullcite-produces-different-result-from-bibliography-entry
\DeclareCiteCommand{\fullcite}
  {\usebibmacro{prenote}}
  {\usedriver
     {\defcounter{minnames}{6}%
      \defcounter{maxnames}{6}}
     {\thefield{entrytype}}.}
  {\multicitedelim}
  {\usebibmacro{postnote}}
\addbibresource{/home/till/Dokumente/BibTex/Second_Mexico_paper-Article.bib}
\addbibresource{/home/till/Dokumente/BibTex/Thesis.bib}
%\usepackage[authoryear]{natbib}

%\nobibliography*
% paper margins
\usepackage{geometry}
\geometry{
letterpaper,
left=40mm,
right=35mm,
top=20mm,
bottom=25mm,
}   
%limiting tables to only float within section
\usepackage[section]{placeins}
  
%use for commands only working with pdf
  


% *****************************************************************
% siunitx
% *****************************************************************
\usepackage{siunitx} % centering in tables
	\sisetup{
		detect-mode,
		tight-spacing		= true,
		group-digits		= false ,
		input-signs		= ,
		input-symbols		= ( ) [ ] - + *,
		input-open-uncertainty	= ,
		input-close-uncertainty	= ,
		table-align-text-post	= false
        }


\pagenumbering{gobble} %no page numbering
%For table commands to change column sizeand alignment, especially tabularx
\newcolumntype{b}{X}  %large columns http://tex.stackexchange.com/questions/84400/table-layout-with-tabularx-column-widths-502525
\newcolumntype{m}{>{\hsize=.5\hsize}X} % medium columns
\newcommand{\merge}[1]{\multicolumn{2}{>{\hsize=\dimexpr2\hsize+2\tabcolsep+\arrayrulewidth\relax}X}{#1}}  %allows merging of two columns in tabularx http://tex.stackexchange.com/questions/236155/tabularx-and-multicolumn

\newcolumntype{Y}{>{\centering\arraybackslash}X} %new columntype for X columns in tabularx to center them http://tex.stackexchange.com/questions/89166/centering-in-tabularx-and-x-columns
\newcolumntype{Z}{>{\raggedright\let\newline\\\arraybackslash\hspace{0pt}}X} %left aligned X columns http://tex.stackexchange.com/questions/97180/how-to-get-column-alignment-in-tabularx
\newcolumntype{z}{>{\hsize=.5\hsize\\\raggedright\let\newline\\\arraybackslash\hspace{0pt}}X} %left aligned X columns http://tex.stackexchange.com/questions/97180/how-to-get-column-alignment-in-tabularx
\newcolumntype{y}{>{\hsize=.5\hsize}Y} % small columns
