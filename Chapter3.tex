

\begin{abstract}
This study explores the impact of diabetes on employment in Mexico using data from the \acf{MxFLS} (2005), taking into account the possible endogeneity of diabetes via an instrumental variable estimation strategy. We find that diabetes significantly decreases employment probabilities for men by about 10 percentage points (p<0.01) and somewhat less so for women---4.5 percentage points (p<0.1)---without any indication of diabetes being endogenous. Further analysis shows that diabetes mainly affects the employment probabilities of men and women above the age of 44 and also has stronger effects on the poor than on the rich, particularly for men. We also find some indication for more adverse effects of diabetes on those in the large informal labour market compared to those in formal employment. Our results highlight---for the first time---the detrimental employment impact of diabetes in a developing country.
\end{abstract}

\section{\label{sec:Introduction3}Introduction}

Diabetes, similar to other conditions that have been coined ''diseases of affluence'', has traditionally been seen as mostly a problem of the developed, more affluent countries. Only in recent
years the awareness has been growing of the sheer size of the problem
in health terms \parencite{Yach2006,Hu2011}. Mexico is one example of
a middle-income country that has seen diabetes rates increase sharply
over the last years, from about 7.5\% in 2000 \parencite{Barquera2013}
to 12.6\% in 2013 \parencite{InternationalDiabetesFederation2013}.
The high prevalence of diabetes in Mexico reflects an epidemiological
transition from a disease pattern previously characterized by high
mortality and infectious diseases to low-mortality rates and \acp{NCD}
affecting predominantly adults \parencite{Stevens2008}. This transition
has likely been reinforced by nutritional changes away from a traditional
diet towards an energy dense, but nutritionally poor diet with an
increasing amount of processed foods and sugars \parencite{Barquera2008b,Basu2013,Rivera2004},
a reduction in physical activity, as well as what appears to be a
particular genetic predisposition of many Mexicans to develop type
2 diabetes \parencite{Williams2013}. While many of the high-income countries
may be in a position to cope resource-wise with the health care consequences
of diabetes, this will be less so the case for Mexico and other \acp{LMIC}.
The most recent "cost-of-illness" estimates put the costs of diabetes
to the Mexican society at more than US\$778 million in 2010, with
a large part of these costs being paid out-of-pocket \parencite{A.2011z}.
While the above includes some estimate of indirect costs, meant to
capture the cost burden attributable to foregone productivity resulting
from diabetes, there exists no rigorous, econometric assessment of
the effect of diabetes on employment chances for Mexico, as the research
has thus far focused on high-income countries \parencite{Lin2011b,Latif2009,Brown2005,Minor2011,Bastida2002,Vijan2004,Zhang2009}.

There are several reasons to expect a significant adverse
effect of diabetes on employment chances in Mexico and that this effect
might be stronger than in high-income countries. In Mexico type 2
diabetes is increasingly affecting people in their productive age,
raising the possibility that a larger share of people with diabetes
will have to cope with debilitating complications already relatively
early in life \parencite{Barquera2013,Villalpando2010}. Further, only
a minority of Mexicans appears to successfully manage their diabetes
condition, with as much as 70\% of the people with diabetes
having poor control over their disease \parencite{Villalpando2010}. In
addition, many Mexicans are working in the large informal economy\footnote{In 2005 around 58\% of the working population in Mexico were
employed in the informal sector \parencite{Aguila2011}.}, possibly limiting their access to quality health care and hence
to appropriate treatment options. All these factors are likely to
both increase the risk of developing debilitating diabetes complications
as well as to reduce productivity as a result. Against this background,
the aim of this study is to investigate how diabetes affects employment
probabilities in a middle-income country such as Mexico. To the best
of our knowledge this is the first such paper on Mexico and indeed
on any \ac{LMIC}. We also investigate if the impact of diabetes on
employment chances differs across age groups and---again for the
first time in this field---by wealth, as well as between those formally
and informally employed.

The majority of the more recent studies on the labour market
impact of diabetes tried to account for the possible endogeneity of
diabetes using family history of diabetes as an instrument. Endogeneity
might arise due to reverse causality: employment status and its effect
on a person's lifestyle may also influence the odds of developing
diabetes. A job with long office working hours might push a person's
diet or exercise pattern towards a more unhealthy and sedentary lifestyle
due to reduced leisure time, increasing the person's risk for diabetes.
In addition, unobserved factors, such as personal traits, could simultaneously
influence a person's employment as well as his or her diabetes status
and introduce an omitted variable bias. A less ambitious person could
be less productive in a job, increasing the risk of being laid off,
and he or she could simultaneously have only modest, if any, exercise
goals or healthy eating habits, thereby increasing the chances of
developing diabetes.

\textcite{Brown2005} estimated the impact of the disease on
employment in 1996--1997 in an older population of Mexican Americans
in the USA close to the Mexican border, using a recursive bivariate
probit model. They found diabetes to be endogenous for women but not
for men. The results of the \ac{IV} estimation suggested no significant
effect on women which, compared to the adverse effect found in the
probit model, indicated an overestimation of the effect for women
when endogeneity was not accounted for. For men, the probit estimates
showed a significant adverse effect of about 7 percentage points.
\textcite{Latif2009} estimated the effect of the disease on employment
probabilities in Canada in 1998. Contrary to \textcite{Brown2005}, he
found diabetes to be exogenous for females and endogenous for males;
taking this into account he obtained a significant negative impact
on the employment probabilities for women, but not for men. Because
the simple probit model showed a significant negative effect for males,
\textcite{Latif2009} concluded that not accounting for endogeneity resulted
in an overestimation of the effect on male employment chances. \textcite{Minor2011}
investigated the effect of diabetes on female employment, among other
outcomes, in the USA in 2006. This particular study differed
from earlier work in that it not only analysed the effects of diabetes
in general, but also of type 1 and type 2 diabetes separately. The
study found diabetes to be endogenous and underestimated if exogeneity
was assumed. In the \ac{IV} estimates, type 2 diabetes had a significant
negative effect on female employment chances. For Taiwan, \textcite{Lin2011b}
found diabetes to be endogenous, with the \ac{IV} results showing
significant changes in the employment effect of diabetes. The impact
was found to be significantly negative for men in the \ac{IV} model
indicating an underestimation in the standard probit model, where
the diabetes coefficient was also significant but much smaller in
size. For women, no significant effect was found in the \ac{IV} estimation
after the probit model had indicated a significant and negative impact
of diabetes. 

Accordingly, at least in some cases, there seems to be the
risk of biased estimates of the impact of diabetes on employment,
when exogeneity is assumed, with an a priori ambiguous bias. Hence,
our decision in this study to also assess if diabetes is endogenous
and how precisely taking account of endogeneity might affect the estimates.
  In order to account for this possible endogeneity we use data from
the second wave of the \acf{MxFLS} from 2005, which not only provides
information on people\textquoteright s diabetes status and socioeconomic
background, but also on parental diabetes, enabling us to construct
an instrumental variable similar to what has been used in the previous
literature on high-income countries.\footnote{Studies that have used the family history of diabetes as an instrument
for diabetes are \textcite{Brown2005} for a Mexican-American community,
\textcite{Latif2009} for Canada, \textcite{Minor2011} for females in
the USA and \textcite{Lin2011b} for Taiwan.} The data also allows the extension of the analysis to test if the
inclusion of information on parental education as an additional control
variable affects the \ac{IV} parameter estimates.

The remainder of the paper is structured as follows. Section
\ref{sec:Methodology3} provides details about the used dataset and
the econometric specification; and section \ref{sec:RESULTS} presents
and discusses the empirical results. Section \ref{sec:Conclusion}
concludes.



\section{\label{sec:Methodology3}Methodology}


\subsection{\label{sub:Data}Dataset and descriptive statistics}

The dataset used for the empirical analysis is the \acf{MxFLS}.
It is a nationally representative household survey which was conducted
in 2002 and 2005. We use data from the second wave in 2005, which
includes almost 40,000 individuals. Interviews were conducted with
all household members aged 15+, and information on a wide range of
social, demographic, economic and health related topics was collected
\parencite{Rubalcava2008}. While there are more recent datasets available
on Mexico, none of these provide as extensive information on parental
characteristics as does the \ac{MxFLS} which includes information
on parental diabetes and education status, even if parents were not
alive anymore or were living in a non-surveyed household at the time
of the survey. Diabetes is self-reported and 3.7\% of males
and 5.1\% of females report a diagnosis by a doctor.\footnote{
This is well below the estimated prevalence rate for 2013
of almost 12\%. This is likely due to the fact that, according
to the \ac{IDF}, more than half of the people with diabetes in Mexico
are undiagnosed and consequently did not report it \parencite{InternationalDiabetesFederation2013}.
Further, the sample in the survey at hand is restricted to people
between the age of 15 to 64, which does not match exactly with the
population the \ac{IDF} used for the diabetes prevalence estimates
(20 -- 79). Hence, our used sample includes a greater share of young
people with a very low diabetes prevalence and excludes people above
64 years of age, which likely have a higher than average prevalence
rate. Taken together, this---as well as a further increase in prevalence
since 2005---should explain the difference between the diabetes
prevalence in our sample and the one estimated by the \ac{IDF}.
} Unfortunately we cannot---with the data at hand---distinguish
between the different types of diabetes. It can be assumed, however,
that about 90\% of the reported diagnoses are due to type 2
diabetes, which is by far the most common type of diabetes \parencite{Sicree2009}.
The sub-sample used for analysis is limited to the age group of 15
to 64 years, which represents the majority of the working population.
To allow for heterogeneity in the coefficients across gender, the
sample has been split to estimate the male and female groups separately. 


\begin{landscape}
\begin{table}[p]
\protect\caption{\label{tab:Summary-statistics-for}Summary statistics for males and
females with and without diabetes}
\begin{center}
\begin{adjustbox}{max width=\textwidth, center} 
\begin{threeparttable}

{ \def\sym#1{\ifmmode^{#1}\else\(^{#1}\)\fi} \begin{tabular}{l*{2}{ccc}} \toprule             &\multicolumn{3}{c}{Males}             &\multicolumn{3}{c}{Females}           \\             &Mean with diabetes&Mean without diabetes&  p (t-test)&Mean with diabetes&Mean without diabetes&  p (t-test)\\ \midrule Employed       &       0.714&       0.804&       0.000&       0.229&       0.313&       0.000\\ Age         &      50.945&      35.016&       0.000&      48.955&      34.717&       0.000\\ Age 15--24  &       0.008&       0.294&       0.000&       0.036&       0.282&       0.000\\ Age 25--34  &       0.043&       0.232&       0.000&       0.076&       0.250&       0.000\\ Age 35--44  &       0.161&       0.196&       0.162&       0.180&       0.221&       0.042\\ Age 45--54  &       0.392&       0.166&       0.000&       0.366&       0.159&       0.000\\ Age 55--64  &       0.396&       0.111&       0.000&       0.342&       0.089&       0.000\\ Rural       &       0.337&       0.399&       0.047&       0.391&       0.399&       0.723\\ Small city  &       0.082&       0.126&       0.038&       0.144&       0.123&       0.204\\ City        &       0.145&       0.102&       0.028&       0.103&       0.098&       0.737\\ Big city    &       0.435&       0.372&       0.042&       0.362&       0.379&       0.475\\ Southsoutheast&       0.208&       0.203&       0.864&       0.184&       0.206&       0.270\\ Central     &       0.243&       0.184&       0.017&       0.231&       0.195&       0.062\\ Westcentral &       0.173&       0.213&       0.124&       0.191&       0.210&       0.343\\ Northeastcentral&       0.196&       0.177&       0.446&       0.209&       0.186&       0.236\\ Northwestcentral&       0.180&       0.223&       0.112&       0.184&       0.202&       0.355\\ No education&       0.090&       0.062&       0.070&       0.151&       0.081&       0.000\\ Primary     &       0.518&       0.352&       0.000&       0.607&       0.368&       0.000\\ Secondary   &       0.231&       0.308&       0.009&       0.171&       0.314&       0.000\\ Highschool  &       0.059&       0.158&       0.000&       0.043&       0.138&       0.000\\ College or university&       0.102&       0.120&       0.379&       0.029&       0.098&       0.000\\ Indigenous  &       0.137&       0.121&       0.448&       0.133&       0.118&       0.341\\ Married     &       0.812&       0.535&       0.000&       0.663&       0.539&       0.000\\ Children (under 15)&       1.118&       1.510&       0.000&       1.207&       1.600&       0.000\\ Wealth      &       0.179&      -0.010&       0.003&       0.004&      -0.003&       0.885\\ Diabetes    &       1.000&       0.000&           .&       1.000&       0.000&           .\\ Diabetes father&       0.180&       0.071&       0.000&       0.146&       0.079&       0.000\\ Diabetes mother&       0.251&       0.107&       0.000&       0.236&       0.113&       0.000\\ Education parents&       0.596&       0.697&       0.001&       0.528&       0.699&       0.000\\ Formal employment      &       0.286&       0.306&       0.508&       0.083&       0.140&       0.001\\ Informal employment    &       0.529&       0.560&       0.342&       0.191&       0.220&       0.155\\ \midrule N&       255&       6031&       &    445   &       7798&      \\ \bottomrule \end{tabular} 
}
\end{threeparttable}
\end{adjustbox}
\end{center}
\end{table}
\end{landscape}

The descriptive statistics presented in Table \ref{tab:Summary-statistics-for}
suggest that the groups of respondents with and without diabetes differ
significantly in various aspects. Both males and females with diabetes
have a lower employment rate than their counterparts. This would suggest
that diabetes has a negative impact on the employment chances of both
males and females with diabetes. However, since the groups with diabetes
are also significantly older and differ in terms of education, this
may be a spurious relationship. As a result, only a multivariate analysis
will provide more reliable information on how diabetes truly affects
employment probabilities.


\subsection{Econometric specification}

We first estimate a probit model with the following specification 


\begin{equation}
Employed_{i}=\beta_{0}+\beta_{1}Diabetes_{i}+\beta_{2}X_{i}+u_{i}\label{eq:employed-2}
\end{equation}


where diabetes is assumed to be exogenous. $Employed_{i}$
takes the value of $1$ if person $i$ is employed and $0$ if unemployed.
Employment status is defined as having worked or carried out an activity
that helped with the household expenses for at least ten hours over
the last week. This explicitly includes those employed informally,
for instance people working in a family business. 

$Diabetes_{i}$ denotes the main independent variable of interest, taking the value
of $1$ if individual $i$ has reported a diagnosis of diabetes and
$0$ otherwise. 

$X_{i}$ contains various control variables. Because
no information on job history is available in the data to adequately
account for work experience, we need to rely on the combination of
age and education to proxy for work experience \parencite{Aaronson2010}.
The effect of age is captured through dummy variables for age intervals.
Education is taken into account by dummy variables indicating if the
highest level of schooling attained was either primary school, secondary
school, high school, university or some other form of higher education
with no education serving as the reference category, to control for
the impact of education on employment and to account for the relationship
between diabetes and education \parencite{Agardh2011}. 

Since Mexico is a large and diverse country with regional socioeconomic differences
we also include dummies for five different Mexican regions\footnote{The region variables have been constructed after recommendations on
the MxFLS-Homepage. South-southeastern Mexico: Oaxaca, Veracruz, Yucatan;
Central Mexico: Federal District of Mexico, State of Mexico, Morelos,
Puebla; Central northeast Mexico: Coahuila, Durango, Nuevo Leon; Central
western Mexico: Guanajuato, Jalisco, Michoacan; Northwest Mexico:
Baja California Sur, Sinaloa, Sonora.}. Apart from the more obvious effects economic differences between
regions can have on employment chances and diabetes through their
impact on employment opportunities and lifestyles, the dummies should
also account for less obvious effects that macroeconomic problems,
such as a high unemployment rate, could have on employment chances
and diabetes by affecting psychological well-being and sleeping patterns
\parencite{Antillon2014}. Because differences in economic opportunities
and lifestyles should also be expected between rural and urban areas,
three dummy variables are included to capture the effects these factors
might have on employment chances and diabetes, with living in a rural
area being the reference category\footnote{Rural: < 2,500 inhabitants; Small city: 2,500 to 15,000 inhabitants;
City: 15,000 to 100,000 inhabitants; Big city: > 100,000 inhabitants. } \parencite{Villalpando2010}. Further, to control for labour market discrimination
and possible differences in genetic susceptibility to diabetes of
indigenous populations \parencite{Yu2007}, a dummy for being a member
of an indigenous group is included. We also account for for the marital
status to control for the impact of marriage on employment chances
and lifestyle habits. Further a variable capturing the number of children
residing in the household below the age of 15 is inlcuded, to control
for their impact on employment chances and for the effect of childbearing
and related gestational diabetes on the probabilities of women to
develop type 2 diabetes \parencite{Bellamy2009}. 

To account for the effect that household wealth might have on diabetes and employment chances,
we use the well established method of principal component analysis
of multiple indicators of household assets and housing conditions
to create an indicator for household wealth \parencite{Filmer2001}. Our
composite wealth index consists of owning a vehicle, owning a house
or other real estate, owning another house, owning a washing machine,
dryer, stove, refrigerator or furniture, owning any electric appliances,
owning any domestic appliances, owning a bicycle and owning farm animals.
It further accounts for the physical condition of the house, proxied
by the floor material of the house, and the type of water access. 

The error term is denoted as $u_{i}$. We do not control
for the general health status and other diabetes related chronic diseases
as they are likely determined by diabetes itself and, hence, could
bias the estimates and compromise a causal interpretation of the effect
of diabetes on employment \parencite{Angrist2009a}.

As diabetes could be endogenous, the probit model might
deliver biased estimates. Therefore we employ an \ac{IV} strategy,
using a bivariate probit model to estimate the following two equations
simultaneously:


\begin{equation}
Diabetes_{i}=\delta_{0}+\delta_{1}X_{i}+\delta_{2}diabetesmother_{i}+\delta_{3}diabetesfather_{i}+\eta_{i}\label{eq:employed-1}
\end{equation}



\begin{equation}
Employed_{i}=\beta_{0}+\beta_{1}Diabetes_{i}+\beta_{2}X_{i}+u_{i}\label{eq:employed}
\end{equation}
In equation \ref{eq:employed-1}, $Diabetes_{i}$
is a dummy variable and is modelled as a function of the same socioeconomic
and demographic factors $X_{i}$ as in equation \ref{eq:employed-2}
and of the instrumental dummy variables $diabetesmother_{i}$ and
$diabetesfather_{i}$, indicating if the father or the mother had
been diagnosed with diabetes. The error term is denoted as $\eta_{i}$.
Equation \ref{eq:employed} is identical to the probit specification
(equation \ref{eq:employed-2}) and estimates the effect of diabetes
on employment, now taking into account the possible endogeneity of
diabetes. Diabetes is exogenous if the error terms of both equations
are independent of each other ($Cov(u_{i}\eta_{i})=0$). Endogeneity
is tested using a likelihood ratio test based on the idea that if
$Cov(u_{i}\eta_{i})=0$, the log-likelihood for the bivariate probit
will be equal to the sum of the log-likelihoods from the two univariate
probit models \parencite{Knapp1998}. If $u_{i}$ 
 and $\eta_{i}$ are correlated, the estimation of equation \ref{eq:employed-2}
using a probit model will not provide consistent estimates of the
impact of diabetes on employment. In this case the simultaneous estimation
of both equations using the bivariate probit should be preferred.
For the estimation of the bivariate probit model it is assumed that
$u_{i}$ 
 and $\eta_{i}$ are distributed randomly and bivariate normal. To
test the assumption of normality, we use Murphey's goodness-of-fit
score test with the null-hypothesis of bivariate normally distributed
errors, as suggested by \textcite{Chiburis2012}.\footnote{Murphey's score test ''\ldots{}embeds the bivariate normal distribution
within a larger family of distributions by adding more parameters
to the model and checks whether the additional parameters are all
zeros using the score for the additional parameters at the bivariate
probit estimate.'' \parencite[p. 19]{Chiburis2012}.}

We choose the bivariate probit model over the linear \ac{IV}
model to account for endogeneity, as there is evidence that it performs
better if the sample is relatively small (<5,000) and---more important
in our case---when treatment probabilities are low. In such cases
the linear \ac{IV} can produce uninformative estimates while the
bivariate probit model has been shown to provide much more reasonable
results \parencite{Chiburis2012}. Because only 4\% of males and
5.4\% of females report a diagnosis of diabetes, treatment probabilities
are indeed low in the given case, providing good justification for
the use of the bivariate probit model. 

In order to fulfil the conditions of a valid instrument,
parental diabetes needs to impact the diabetes risk of the offspring
while at the same time being unrelated to the offspring's employment
chances. It has been shown that there is a strong hereditary component
of type 2 diabetes which predisposes the offspring of people with
diabetes to develop the condition as well \parencite{Herder2011,TheInteractConsortium2013}.
This is supported by the notion that genes seem to play a crucial
role, besides the recent epidemiological transition and the migration
from rural to urban areas, in explaining Mexico's high diabetes prevalence
according to a recent study by \textcite{Williams2013}. The authors
identified a specific gene particularly prevalent in Mexican and other
Latin American populations with native American ancestry, which is
associated with a 20\% increase in the risk of developing type
2 diabetes. Furthermore, research has shown that parental lifestyle
factors, socioeconomic background as well as parental \ac{BMI} can
explain but a very small fraction of the increased risk of type 2
diabetes in the offspring, which is why we assume that the increased
risk is mainly due to genetic factors unrelated to lifestyle \parencite{Herder2011,TheInteractConsortium2013}.
This is supported by \textcite{Hemminki2010}, who find that adoptees
whose biological parents had type 2 diabetes, had an increased risk
of developing type 2 diabetes even though they were living in a different
household, while if their adopted parents had the disease, they had
no elevated risk. 

Nonetheless, there might still be the chance that parental
diabetes decreases the offspring's employment chances. The additional
financial burden of diabetes or an early death due to diabetes could
have prevented the parents from investing in their children's education
the way they would have liked to or it could have led to the child
dropping out of school in order to support the family. However, controlling
for education should account for these effects if they exist. Therefore
parental diabetes should be a valid instrument which predicts diabetes
while not affecting employment probabilities through other unobserved
pathways. To further improve instrument validity we also account for
the possibility that parental education is simultaneously correlated
with the parental diabetes status as well as their children's employment
chances, by including a dummy variable indicating if any of the parents
had attained more than primary education. 

A possible limitation of using parental diabetes as our
instrument is that it might directly affect the offspring's employment
decision through other pathways than education. Conceivably, diabetes
might deteriorate parental health in such a way that the offspring
has or had to give up its own employment in order to care for its
parents or is forced to take up work to financially provide for the
parents. With the data at hand we are unable to account for this,
but if this effect exists it should be picked up by the overidentification
test. 

We also estimate the linear \ac{IV} model
as it is consistent even under non-normality \parencite{Angrist2009a}.
The linear \ac{IV} model takes the following form of a first (Equation
\ref{eq:employed-1-1}) and a second stage (Equation \ref{eq:employed-3}).


\begin{equation}
Diabetes_{i}=\pi_{0}+\pi_{1}X_{i}+\pi_{2}diabetesmother_{i}+\pi_{3}diabetesfather_{i}+\eta_{i}\label{eq:employed-1-1}
\end{equation}



\begin{equation}
Employed_{i}=\beta_{0}+\beta_{1}Diabetes_{i}+\beta_{2}X_{i}+u_{i}\label{eq:employed-3}
\end{equation}
In the second stage, the potentially endogenous actual diabetes values
are replaced with the predicted values from the first stage. The covariates
are the same as in the bivariate probit case described in equations
\ref{eq:employed-1} and \ref{eq:employed}. In the linear \ac{IV}
model the Hausman test is used to identify endogeneity. Validity of
the instruments is tested using first stage diagnostics of the linear
\ac{IV} model, as similar tests are not available for the bivariate
probit model. Average marginal effects are presented for the probit and bivariate probit models. 


\section{\label{sec:RESULTS}Results}

This section presents the estimation results using 1) a
probit model model that assumes diabetes to be exogenous and 2) \ac{IV}
models with parental diabetes as an instrument for diabetes, to determine
if diabetes is endogenous or if instead the results from the probit
model can be used. 


\subsection{\label{sub:Probit-estimation}Probit results}

\begin{table}[p]
\protect\caption{\label{tab:Impact-of-diabetes-employement}Impact of diabetes on employment
probabilities (probit) }
\begin{center}
\begin{adjustbox}{max width=\textwidth, center} 
\begin{threeparttable}
{ \def\sym#1{\ifmmode^{#1}\else\(^{#1}\)\fi} \begin{tabular}{l*{2}{S S}} \toprule           &\multicolumn{2}{c}{(1)}     &\multicolumn{2}{c}{(2)}     \\           &\multicolumn{2}{c}{Males}   &\multicolumn{2}{c}{Females} \\ \midrule Age 25--34&     .124\sym{***}&   (.011)&     .121\sym{***}&   (.017)\\ Age 35--44&     .133\sym{***}&   (.012)&     .232\sym{***}&   (.018)\\ Age 45--54&     .085\sym{***}&   (.014)&     .170\sym{***}&   (.022)\\ Age 55--64&    -.034  &   (.020)&     .039         &   (.026)\\ Small city&    -.013         &   (.017)&     .043\sym{**} &   (.020)\\ City      &    -.036\sym{*} &   (.019)&     .042\sym{**} &   (.021)\\ Big city  &     .029\sym{**} &   (.013)&     .101\sym{***}&   (.014)\\ Central   &     .027  &   (.015)&    -.032\sym{*}  &   (.018)\\ Westcentral&     .020         &   (.015)&    -.008         &   (.018)\\ Northeastcentral&     .003         &   (.016)&    -.053\sym{***}&   (.017)\\ Northwestcentral&    -.037\sym{**} &   (.016)&    -.100\sym{***}&   (.016)\\ Primary   &     .056\sym{***}&   (.020)&    -.006         &   (.022)\\ Secondary &     .051\sym{**} &   (.021)&     .058\sym{**} &   (.025)\\ Highschool&     .040\sym{*}  &   (.023)&     .126\sym{***}&   (.029)\\ College or university&     .047\sym{**}  &   (.023)&     .297\sym{***}&   (.033)\\ Indigenous&     .005         &   (.016)&    -.005         &   (.020)\\ Married   &     .092\sym{***}&   (.012)&    -.231\sym{***}&   (.012)\\ Children (under 15)&     .010\sym{**}&   (.004)&    -.018\sym{***}&   (.004)\\ Wealth    &     .002         &   (.006)&     .037\sym{***}&   (.007)\\ Education parents&    -.007         &   (.013)&     .000         &   (.013)\\ Diabetes  &    -.100\sym{***}&   (.029)&    -.045\sym{*}  &   (.023)\\ \midrule Log likelihood&-2897.807         &         &-4508.573         &         \\ N         &     6286         &         &     8243         &         \\ \bottomrule \multicolumn{5}{l}{\footnotesize Average marginal effects; robust standard errors in parentheses.} \\ \multicolumn{5}{l}{\footnotesize * p < 0.1, ** p < 0.05, *** p < 0.01}\\ \end{tabular}
}
\end{threeparttable}
\end{adjustbox}
\end{center}
\end{table}

Table \ref{tab:Impact-of-diabetes-employement} indicates
that the effect of diabetes is negative for both sexes. For males,
it reduces the probability of being employed by 10 percentage points
(p<0.01).


For females, the effect is also negative but smaller, and
shows a reduction in employment probabilities of about 4.5 percentage
points (p<0.1).




The other covariates largely show the expected relationships.
Employability increases with age and is highest for the 35--44 years
age group. Especially for women, living in a more urban environment
increases employment chances compared to women living in rural areas.
Also, women seem to benefit substantially from higher education in
terms of employment chances. For men the effects of education are
also positive, though, not as marked as for women. Perhaps surprisingly,
being part of an indigenous population does not affect employment
probabilities, neither for males or females. 

The probit results suggest a significant negative effect
of diabetes on the employment probabilities of males and likely also
females in Mexico. In light of the concern that diabetes could be
endogenous the following section presents the results of the \ac{IV}
estimations. 
\FloatBarrier

\subsection{\label{sub:Bivariate-probit}IV results}


\begin{table}[p]
\caption{\label{tab:Bivariate-probit-model}Impact of diabetes on employment
probabilities (bivariate probit)}
\begin{center}
\begin{adjustbox}{max width=\textwidth, center}
\begin{threeparttable}
{ \def\sym#1{\ifmmode^{#1}\else\(^{#1}\)\fi} \begin{tabular}{l*{2}{S S}} \toprule           &\multicolumn{2}{c}{(1)}     &\multicolumn{2}{c}{(2)}     \\           &\multicolumn{2}{c}{Males}   &\multicolumn{2}{c}{Females} \\ \midrule Age 25--34&     .125\sym{***}&   (.012)&     .109\sym{***}&   (.015)\\ Age 35--44&     .134\sym{***}&   (.012)&     .207\sym{***}&   (.016)\\ Age 45--54&     .089\sym{***}&   (.016)&     .149\sym{***}&   (.021)\\ Age 55--64&    -.025         &   (.025)&     .032         &   (.029)\\ Small city&    -.014         &   (.017)&     .039\sym{**} &   (.018)\\ City      &    -.035\sym{**} &   (.018)&     .038\sym{**} &   (.019)\\ Big city  &     .030\sym{**} &   (.013)&     .093\sym{***}&   (.013)\\ Central   &     .027         &   (.018)&    -.030\sym{*}  &   (.015)\\ Westcentral&     .019         &   (.018)&    -.007         &   (.016)\\ Northeastcentral&     .002         &   (.018)&    -.049\sym{***}&   (.017)\\ Northwestcentral&    -.038\sym{**} &   (.017)&    -.091\sym{***}&   (.015)\\ Primary   &     .057\sym{***}&   (.020)&    -.006         &   (.021)\\ Secondary &     .052\sym{**} &   (.023)&     .052\sym{**} &   (.022)\\ Highschool&     .040         &   (.025)&     .113\sym{***}&   (.027)\\ College or university&     .046\sym{*}  &   (.025)&     .273\sym{***}&   (.032)\\ Indigenous&     .006         &   (.017)&    -.005         &   (.016)\\ Married   &     .093\sym{***}&   (.012)&    -.215\sym{***}&   (.011)\\ Children (under 15)&     .010\sym{**} &   (.004)&    -.016\sym{***}&   (.004)\\ Wealth    &     .002         &   (.006)&     .033\sym{***}&   (.007)\\ Parental education&    -.006         &   (.013)&     .000         &   (.012)\\ Diabetes  &    -.185         &   (.143)&    -.021         &   (.108)\\ \midrule Instruments  & & & & \\ \hspace{10 mm} Diabetes father  & .048\sym{***} &(.011) & .041\sym{***} & (.010) \\ \hspace{10 mm} Diabetes mother & .037\sym{***} & (.008) & .054\sym{***} & (.008) \\ \midrule Log likelihood&-3737.766         &         &-5939.588         &         \\ Score goodness-of-fit (H0=normality of errors) &   12.32       &         &    8.85    &     \\
\hspace{10 mm}p value & .196      &         & .451        & \\  Endogeneity (H0: Diabetes exogeneous) &         .443       &         &.039            &    \\  \hspace{10 mm}p value&     .506         &              &  .844            & \\ N         &     6286         &         &     8243         &         \\ \bottomrule  \end{tabular} 
\begin{tablenotes}
\item \textit{Notes}  Average marginal effects; robust standard errors in parentheses. The presented coefficients and standard errors for the instruments result from the estimation of the model specified in Equation II, indicating the effect of parental diabetes on a person's diabetes risk.
\item \sym{*} \(p<0.10\), \sym{**} \(p<0.05\), \sym{***} \(p<0.01\))
\end{tablenotes}
}
\end{threeparttable} 
\end{adjustbox}
\end{center}
\end{table}
Using the bivariate probit model, the diabetes coefficient for males
increases in size and remains negative whereas for females it decreases
but also remains negative. However, standard errors increase in both
models and the results turn insignificant, suggesting considerable
loss of efficiency (see Table \ref{tab:Bivariate-probit-model}).
The likelihood-ratio test does not reject the null hypothesis of no
correlation between the disturbance terms of equations \ref{eq:employed-1}
and \ref{eq:employed} for males and females, suggesting exogeneity
of diabetes. The test for normality of the error term does not reject
the null hypothesis of normality for the male and the female model,
increasing our confidence in the estimates. Nonetheless we also consider
the results of the linear \ac{IV} model: the test statistics indicate
sufficiently strong and valid instruments, as shown by the Kleibergen-Paap
Wald F statistic for weak instruments of 20.48 for men and 27.71 for
women, being above the critical value of 19.93 for ten \%\ac{IV}
size and well above the rule of thumb of 10 for weak identification
not to be considered a problem \parencite{Staiger1997,Baum2007}. The
Sargan test does not reject the null hypothesis of instruments uncorrelated
with the error term and instruments correctly excluded from the estimated
equation. The coefficients of the linear \ac{IV} model are very different
from the bivariate probit model, turning positive for males and females,
but also very imprecise as indicated by the large standard errors
(see Table \ref{tab:Linear-IV-and} displaying the main results and
Table \ref{tab:Linear-IV-estimates-1st-2nd-stage}
presenting the complete first and second stage estimates). As mentioned
before, \textcite{Chiburis2012} show that the estimates of the linear
\ac{IV} model are likely to be imprecise when low treatment probabilities
exist and can differ substantially from the bivariate probit model,
which seems to be the case here.\footnote{It could also be the case that the difference in estimates is due
to the fact that while the bivariate probit model estimates the \ac{ATE}
of the variable of interest for the whole sample, the linear \ac{IV}
model estimates the \ac{LATE}, which estimates the effect of diabetes
on employment only for those that have diabetes and whose parents
have or have had diabetes as well. Therefore, the estimates of both
models can be different \parencite{Angrist2009a,Chiburis2012}.} Since the linear \ac{IV} models fail to reject exogeneity of diabetes
as well, we are confident that the standard probit model provides
unbiased and efficient estimates of the effect of diabetes on employment
chances in Mexico and should therefore be used for inference.


\begin{table}[p]
\protect\caption{\label{tab:Linear-IV-and}Impact of diabetes on employment probabilities
(linear IV)}


\begin{center}
\begin{adjustbox}{max width=\textwidth, center} 
\begin{threeparttable}

{ \def\sym#1{\ifmmode^{#1}\else\(^{#1}\)\fi} \begin{tabular}{l*{2}{S S}} \toprule           &\multicolumn{2}{c}{(1)}     &\multicolumn{2}{c}{(2)}     \\           &\multicolumn{2}{c}{Males}   &\multicolumn{2}{c}{Females} \\ \midrule Diabetes  &     .098         &   (.215)&     .239         &   (.214)\\ \midrule R2        &     .067         &         &     .120         &         \\ F stat (H0: weak instruments)&   20.483         &         &   27.706         &         \\ Sargan test (H0: valid instruments)&     .862         &         &     .295         &         \\ \hspace{10 mm}p value&     .353         &         &     .587         &         \\ Endogeneity (H0: Diabetes exogenous)&     .864         &         &    1.796         &         \\ \hspace{10 mm}p value&     .353         &         &     .180         &         \\ N         &     6286         &         &     8243         &         \\ \bottomrule 
\end{tabular} 
\begin{tablenotes}
\item \textit{Notes} Robust standard errors in parentheses. Instruments: diabetes of mother, diabetes of father. Other control variables: age, region, urban, education, indigenous, marital status, children, wealth, parental education. Critical values for weak identification test F statistic: 10\% maximal IV size 19.93, 15\% maximal IV size 11.59, 20\% maximal IV size 8.75, 25\% maximal IV size 7.25.
\item \sym{*} \(p<0.10\), \sym{**} \(p<0.05\), \sym{***} \(p<0.01\))
\end{tablenotes}
}
\end{threeparttable}
\end{adjustbox}
\end{center}
\end{table}


The next section investigates the effects of diabetes for
two different age groups, 15--44 and 45--64, to explore whether, and
if so, how the effect of diabetes on employment chances differs between
older and younger people. There might be reason to believe that diabetes
has a more adverse effect in older age groups, when those suffering
from diabetes are likely to have accumulated more years lived with
diabetes, and hence are more likely to develop complications. 

\FloatBarrier
\subsection{Differences by age groups }

When divided into an older and younger age group using the
cut-off point of 45 years, the negative effect of diabetes is mainly
found in the older age group, for males and females alike (see Table
\ref{tab:age groups probit}), where 12.5\% report having diabetes,
compared to only 1.7\% in the younger age group. The probability
of being employed is reduced by about 10 percentage points for men
between 45 and 64 years at the 1\% significance level, while
there is no significant effect on younger men. For women, the employment
probability is reduced by about 6 percentage points, with the effect
being significant at the 5\% level. Similar to men, there
is no effect of diabetes on younger women. To investigate in more
detail for which age group the effect is strongest, we run separate
regressions for both age groups above 44 years. The results (Table
\ref{tab:Impact-of-diabetes-age-groups-1}) show that
for men the strongest effect appears in the oldest age group (i.e.
55--64 years), where employment chances are reduced by almost 13 percentage
points. For females, a significant effect is found solely for those
between 45 and 54 years, where employment chances are reduced by 7.6
percentage points. Hence, there appear to be relevant differences
between males and females in the age at which the biggest adverse
effect of diabetes on employment chances occurs. 


\begin{table}[p]
\protect\caption{\label{tab:age groups probit}Impact of diabetes on employment probabilities
by age group (probit)}
\begin{center}
\begin{adjustbox}{max width=\textwidth, center} \begin{threeparttable}
{ \def\sym#1{\ifmmode^{#1}\else\(^{#1}\)\fi} \begin{tabular}{l*{4}{S S}} \toprule           &\multicolumn{2}{c}{15-44}            &\multicolumn{2}{c}{45-64}            \\\cmidrule(lr){2-3}\cmidrule(lr){4-5}           &\multicolumn{1}{c}{(1)}&\multicolumn{1}{c}{(2)}&\multicolumn{1}{c}{(3)}&\multicolumn{1}{c}{(4)}\\           &\multicolumn{1}{c}{Males}&\multicolumn{1}{c}{Females}&\multicolumn{1}{c}{Males}&\multicolumn{1}{c}{Females}\\ \midrule Diabetes  &    -.009         &    -.004         &    -.110\sym{***}&    -.057\sym{**} \\           &   (.062)         &   (.042)         &   (.034)         &   (.025)         \\ \midrule Log likelihood&-1987.285         &-3354.003         &-925.409         &-1167.491         \\ N         &     4415         &     5997         &     1871         &     2246         \\ \bottomrule 
\end{tabular} 
\begin{tablenotes}
\item \textit{Notes}  Average marginal effects; robust standard errors in parentheses. For the younger age group, the model contains the age categories 25--34 and 35--44 with 15--24 as the reference category. For the older age group, the model contains the age category 55--64 with 45--54 as the reference category. Other control variables: region, urban, education, indigenous, marital status, children, wealth, parental education.
\item \sym{*} \(p<0.10\), \sym{**} \(p<0.05\), \sym{***} \(p<0.01\))
\end{tablenotes}
}
\end{threeparttable} 
\end{adjustbox}
\end{center}
\end{table}

The use of \ac{IV} methods in the age stratified samples
is compromised due to a reduction in instrument power, sample size
and particularly treatment probabilities. Especially for the younger
age group, where treatment probabilities are close to zero, a meaningful
interpretation of the \ac{IV} results is difficult. Further, because
no endogeneity was found in the pooled samples for males and females
presented in section \ref{sub:Bivariate-probit}, we would not expect
endogeneity of diabetes in the age stratified samples. We nonetheless
test for the possibility of diabetes being endogenous using the bivariate
probit model and an approach suggested by \textcite{Lewbel2012}, to
improve instrument strength (see Table
\ref{tab:IV-estimates-forYOUNG} and Table \ref{tab:IV-estimates-forOLDAGE}).


\FloatBarrier

\subsection{Differences by wealth}

To explore the heterogeneity of the effect of diabetes on employment
across different levels of wealth, we divide the sample into two wealth
groups at the 50\textsuperscript{th} percentile of our constructed
wealth index.



We run separate regressions for both groups stratified by gender,
finding the strongest negative effect for less wealthy males, where
employment chances are reduced by 15 percentage points, and a smaller
and less significant effect for less wealthy females (see Table \ref{tab:Effect-of-diabetes-wealth}).
Whereas the coefficients for wealthier males and females have a negative
sign, they are not significant at the 10\% significance level.
This indicates that mainly the less wealthy experience an adverse
effect from diabetes. To further explore this, we stratified the sample
into wealth quartiles (see Table \ref{tab:Impact-of-diabetes-wealth-quartile}), finding that significant adverse effects for males
appear in the first and second wealth quartile, where employment chances
are reduced by about 14 percentage points. For females a highly significant
and strong effect is only found in the poorest quartile, were employment
chances are reduced by 10 percentage points. Together these results
indicate that the impact of diabetes on employment chances varies
with wealth, with men and women being more affected when being in
the lower wealth quartiles.

\begin{table}[p]
\protect\caption{\label{tab:Effect-of-diabetes-wealth}Impact of diabetes on employment
probabilities by wealth group (probit)}
\begin{adjustbox}{max width=\textwidth, center} 
\begin{threeparttable}

{ \def\sym#1{\ifmmode^{#1}\else\(^{#1}\)\fi} \begin{tabular}{l*{4}{S S}} \toprule           &\multicolumn{2}{c}{Poor}             &\multicolumn{2}{c}{Rich}             \\\cmidrule(lr){2-3}\cmidrule(lr){4-5}           &\multicolumn{1}{c}{(1)}&\multicolumn{1}{c}{(2)}&\multicolumn{1}{c}{(3)}&\multicolumn{1}{c}{(4)}\\           &\multicolumn{1}{c}{Males}&\multicolumn{1}{c}{Females}&\multicolumn{1}{c}{Males}&\multicolumn{1}{c}{Females}\\ \midrule Diabetes  &    -.150\sym{***}&    -.047\sym{*}  &    -.060  &    -.038         \\           &   (.047)         &   (.027)         &   (.038)         &   (.035)         \\ \midrule Log likelihood&-1459.235         &-2040.517         &-1408.746         &-2421.910         \\ N         &     3140         &     4091         &     3106         &     4117         \\ \bottomrule 
\end{tabular} 
\begin{tablenotes}
\item \textit{Notes}  Average marginal effects; robust standard errors in parentheses.Other control variables: region, urban, education, indigenous, marital status, children, wealth, parental education.
\item \sym{*} \(p<0.10\), \sym{**} \(p<0.05\), \sym{***} \(p<0.01\))
\end{tablenotes}
}
\end{threeparttable} 
\end{adjustbox}
\end{table}
To consider the possible endogeneity of diabetes in the upper and
lower wealth half, we again present the results of the \ac{IV} models.
The stratification into wealth groups significantly reduces instrument
power as well as sample size. For none of the wealth groups the bivariate
probit model indicates endogeneity (see Table \ref{tab:Impact-of-diabetes-wealth-IV}). This does not change even when
using the Lewbel approach to increase instrument strength and we therefore
rely on the probit results for inference. 

\FloatBarrier

\subsection{Differences by employment type}



To investigate the effect of diabetes on the employment chances in
the formal and informal labour market, respectively, we estimate separate
models with being employed in the formal and informal sector as the
respective dependent variables. We define formal employment on the
basis of having a written labour contract. Informal employment is
defined as working without a written contract or being self-employed. 

\begin{table}[p]
\protect\caption{\label{tab:Effect-of-diabetes-formal-informal-probit}Impact of diabetes
on employment probabilities by employment status (probit)}
\begin{center}
\begin{adjustbox}{max width=\textwidth, center} 
\begin{threeparttable}

{ \def\sym#1{\ifmmode^{#1}\else\(^{#1}\)\fi} \begin{tabular}{l*{4}{S S}} \toprule           &\multicolumn{2}{c}{Males}            &\multicolumn{2}{c}{Females}          \\\cmidrule(lr){2-3}\cmidrule(lr){4-5}           &\multicolumn{1}{c}{(1)}&\multicolumn{1}{c}{(2)}&\multicolumn{1}{c}{(3)}&\multicolumn{1}{c}{(4)}\\           &\multicolumn{1}{c}{Informal}&\multicolumn{1}{c}{Formal}&\multicolumn{1}{c}{Informal}&\multicolumn{1}{c}{Formal}\\ \midrule Diabetes  &    -.063\sym{**} &    -.041         &    -.051\sym{**} &     .019         \\           &   (.031)         &   (.043)         &   (.022)         &   (.022)         \\ \midrule Log likelihood&-1780.023         &-1021.771         &-3818.588         &-1859.048         \\ N         &     4604         &     2204         &     6983         &     5652         \\ \bottomrule
\end{tabular} 
\begin{tablenotes}
\item \textit{Notes}  Average marginal effects; robust standard errors in parentheses.Other control variables: region, urban, education, indigenous, marital status, children, wealth, parental education.
\item \sym{*} \(p<0.10\), \sym{**} \(p<0.05\), \sym{***} \(p<0.01\))
\end{tablenotes}
}
\end{threeparttable} 
\end{adjustbox}
\end{center}
\end{table}


For this investigation we use two restricted samples: for the estimation
of the effect of diabetes on informal employment we exclude those
currently in formal employment and for the effect of diabetes on formal
employment we exclude those in informal employment from our sample.
We further assume that those who have worked previously and are currently
unemployed are looking for employment in the same sector, i.e. if
they were previously employed in the informal (formal) labour market
they are again looking for an informal (formal) employment. We therefore
exclude those previously working in the informal (formal) labour market
from our estimation of the effect of diabetes on employment in the
formal (informal) labour market. The respective sample thus only contains
those currently working in the informal (formal) labour market, those
previously employed in the informal (formal) labour market and those
that have never worked before. Using this assumption allows the use
of a normal probit model and the investigation of a possible endogeneity
bias using \ac{IV} techniques. 


Admittedly, the assumption that the currently unemployed look for
work in the same labour market they had previously worked in is quite
strong and is likely not true for everybody. We therefore additionally
estimate a multinomial logit model which is most useful if the decision
to work is not binary but there are more than two choices, such as
the choice of being either unemployed, employed in the informal or
employed in the formal labour market \parencite{Wooldridge2002}. Being
unemployed is used as the reference category.

All estimated models (see Tables \ref{tab:Effect-of-diabetes-formal-informal-probit}
and \ref{tab:Effect-of-diabetes-formal-informal-Mlogit}), regardless
of the estimation approach,  indicate that diabetes significantly
reduces the chances of being in informal employment, while it has
no effect on formal employment.\footnote{Please note, however, that the coefficients of the multinomial logit
and the probit model cannot be directly compared as they are based
on different assumptions. The former takes into account that a person
can choose from more than two employment outcomes (i.e. being unemployed,
being formally employed or being informally employed), while the latter
only allows for a binary outcome without considering any other options
(e.g. being unemployed or informally employed without considering
the possibility of formal employment).} This applies to both males and females. This indicates that people
with diabetes are less likely to be working in the informal labour
market relative to being unemployed, while there is no difference
for those working in the formal labour market. We further find no
indication of endogeneity (see Tables \ref{tab:Impact-of-diabetes-informal-IV}
and \ref{tab:Impact-of-diabetes-formal-IV}). Overall,
there seem to be strong differences in terms of the impact of diabetes
on people in formal and informal employment, with diabetes having
a stronger negative effect for those without a written contract.


\FloatBarrier
\section{\label{sec:Conclusion}Conclusion}

The contribution of this paper has been to analyse---for
the first time for a \ac{LMIC}---the impact of diabetes on employment
in Mexico, taking into account the potential endogeneity in the relationship
between diabetes and employment chances. The presented results add
to the growing literature on the adverse economic effects of diabetes.
They indicate that having diabetes substantially reduces the chances
to work for men and likely also for women. Hence, diabetes may contribute
to a reduction in the pool of the productive workforce available to
the Mexican economy. 

We have also shown that diabetes reduces employment chances
particularly in older people, likely because in this age group people
are more common to already have developed diabetes-related complications
which reduce their productivity and eventually force them into unemployment.
Further, particularly for men the effects of diabetes on employment
chances seem to be particularly strong when they belong to the poorer
half of the population. While there might be some self-selection into
the poorer group by those who lost their job due to diabetes and as
a result descended into the lower wealth group, this finding is indicative
of potentially substantial adverse equity impacts. This is also in
line with our finding that diabetes reduces employment chances particularly
for the informally employed, whereas those in formal employment seem
to be less affected. Nonetheless, in order to establish causality
more research in this area will be needed. 

While in parts of the earlier literature diabetes was found
to be exogenous only for either males or females \parencite{Brown2005,Latif2009},
our study found diabetes to be exogenous using the samples stratified
into males and females, allowing the use of the more efficient probit
model to arrive at a consistent estimate of the effect of diabetes
on employment chances. Further, we found no endogeneity of diabetes
for the sample comprised of the age group above the age of 44,  for
the samples stratified into an upper and lower wealth half and for
the samples stratified by employment type. For the younger age group
the bivariate probit model only indicated exogeneity of diabetes for
males, while for females diabetes was shown to be endogenous and showing
a significant positive effect of diabetes on employment. This result
is rather counterintuitive because there is no obvious reason why
diabetes should increase employment chances. Because all samples stratified
into age, wealth and employment groups suffered from reduced instrument
strength which could cause biased \ac{IV} estimates, we used a method
proposed by \textcite{Lewbel2012} to create additional instruments and
increase instrument power. Using this method we no longer found a
significant positive effect of diabetes on female employment chances
in the younger age group and could not reject the assumption of exogeneity
of diabetes in this sample. Also, for all other wealth, age and employment
samples, the Lewbel \ac{IV} method did not reject the assumption
of exogeneity. We are therefore confident that we can rely on the
probit estimates for inference.

Why was diabetes found to be exclusively exogenous in the Mexican
case? We can only speculate on the potential reasons. Diabetes being
exogenous seems to indicate that a person's employment status might
not have such a strong effect on his or her diabetes risk through
the potential pathways such as lifestyle changes. Rather, the rapid
epidemiological transition experienced in Mexico over the last decades
\parencite{Barquera2006,Barquera2008b,Rivera2002a} together with the
heightened genetic susceptibility of Mexicans to diabetes \parencite{Williams2013},
seem to have increased the risk of developing diabetes in both employed
and unemployed Mexicans.

Taking our results for the older age group and comparing
them to those of \textcite{Brown2005} for the USA, whose sample
of Mexican Americans 45 years and older might be the best suited for
a meaningful comparison, our findings indicate a stronger negative
impact of diabetes on males and particularly females residing in Mexico.\footnote{This is based on comparing our estimates to the appropriate
models in \textcite{Brown2005} based on their test for endogeneity,
which indicates the use of the bivariate probit results for women
and the probit results for men. } This finding lends some support to our hypothesis that the adverse
impact of diabetes on employment could be larger in \acp{LMIC} than
in high-income countries. Comparing the study to \textcite{Lin2011b}
for Taiwan, who also used a sample of people between 45 and 64 years
of age, our results are similar in that a larger absolute effect is found for
males than for females. However, when compared
to other studies in more developed countries, with more advanced health
systems and very different populations, such as \textcite{Latif2009}
for Canada and \textcite{Minor2011} for women in the US, our results
differ in that they do find effects for men and potentially also women. 

While the results for women in the main analysis do not reach the levels of statistical significance that those for men do, the negative impact on women is supported by the subgroup analysis. When we take into account the lower overall female employment rates (31\%) compared to men (80\%), the absolute reduction in employment chances in women translates into a an even larger decrease in absolute levels of over 16\% compared to 12.5\% for men. This suggests that diabetes affects employment chances of both sexes were considerable.

A limitation of this study is the use of cross-sectional
data, which does not allow for the use of fixed effects and hence
for the control of unobserved time-invariant heterogeneity. Data spanning
a longer time period would be required to be able
to observe changes in the diabetes and employment status which would
allow the use of fixed effects. A further limitation is the somewhat
old data from 2005, which precedes the main implementation period
of the public health insurance scheme called Seguro Popular. This
should be taken into account when interpreting our results as the
effects might be different today, where most Mexicans have access
to some sort of health insurance \parencite{Knaul2012}. The presented
results rather show the effects of diabetes on employment chances
in 2005 in an environment were insufficient healthcare coverage was
common for parts of the Mexican population. We nonetheless deliberately chose
this particular data as it provided us with a sensible instrument
in parental diabetes as well as an array of other socioeconomic information
which---as far as we have been able to ascertain---is not provided
by any other dataset in \acp{LMIC}. Finally, due to data limitations,
we were not able to investigate the relationship between diabetes
duration and employment chances and how long it takes for an employment
penalty to develop. Recent research by \textcite{Minor2013} on the US
has shown that the effect of diabetes on employment chances changes
with the duration of diabetes and is strongest in the first five years
after diagnosis for males, whereas for females a strong effect appears
only about 11--15 years after diagnosis.

Looking ahead, it would evidently be worthwhile to investigate
the effects of diabetes on employment in Mexico using more recent
data. In light of the recently completed implementation of  Seguro
Popular---which increased its coverage from about 10 million people
in 2005 to over 50 million in 2012 and now provides almost all previously
uninsured Mexicans with access to healthcare \parencite{Knaul2012}---the results of this paper might be used as a baseline to judge the
success of Seguro Popular in reducing the adverse effects of diabetes
on employment. In addition, the reasons for the differences between
males and females in the estimated effects remain a matter of speculation
and more research is needed to explore the underlying pathways. This
information would be valuable in the design of more effective measures
to reduce the negative effects of diabetes for both males and females.

In conclusion, this paper shows that diabetes represents
a large burden for people in Mexico and likely in other \acp{LMIC},
not only due to the associated disease and medical cost burden but
also because of its effect on employment chances. This is particularly
a problem for the poor who are more adversely affected by diabetes
than the more affluent. To alleviate some of the negative effects
of diabetes, Seguro Popular may provide an opportunity to further improve
the prevention and treatment of diabetes in the poor, especially if
the health system adapts to the challenges presented by chronic diseases
\parencite{Samb2010}. Evidence of possible cost-effective interventions
for secondary prevention in the context of Seguro Popular already
exists \parencite{Salomon2012}. There remains, however, an evidence gap
on cost-effective strategies for the primary prevention of diabetes.

\clearpage

\subsection*{\label{sec:Lewbel-and-linear}Linear IV estimates (1st and 2nd stage)}


\begin{landscape}
\begin{table}[p]
\protect\caption{\label{tab:Linear-IV-estimates-1st-2nd-stage}Impact of diabetes on
employment probabilities (linear IV, 1st and 2nd stage)}
\begin{center}
\begin{adjustbox}{max width=\linewidth}  
\begin{threeparttable}


{ \def\sym#1{\ifmmode^{#1}\else\(^{#1}\)\fi} \begin{tabular}{l*{4}{S S}} \toprule           &\multicolumn{4}{c}{linear IV male}                       &\multicolumn{4}{c}{linear IV female}                     \\\cmidrule(lr){2-5}\cmidrule(lr){6-9}           &\multicolumn{2}{c}{(1)}     &\multicolumn{2}{c}{(2)}     &\multicolumn{2}{c}{(3)}     &\multicolumn{2}{c}{(4)}     \\           &\multicolumn{2}{c}{Diabetes}&\multicolumn{2}{c}{Employed}&\multicolumn{2}{c}{Diabetes}&\multicolumn{2}{c}{Employed}\\ \midrule Age 25--34&    -.001         &   (.005)&     .151\sym{***}&   (.015)&     .003         &   (.005)&     .111\sym{***}&   (.015)\\ Age 35--44&     .016\sym{*}  &   (.009)&     .154\sym{***}&   (.019)&     .032\sym{***}&   (.008)&     .198\sym{***}&   (.017)\\ Age 45--54&     .081\sym{***}&   (.014)&     .098\sym{***}&   (.028)&     .108\sym{***}&   (.014)&     .122\sym{***}&   (.028)\\ Age 55--64&     .101\sym{***}&   (.016)&    -.052         &   (.039)&     .198\sym{***}&   (.021)&     .001         &   (.040)\\ Small city&     .001         &   (.010)&    -.010         &   (.019)&    -.005         &   (.011)&     .034\sym{**} &   (.017)\\ City      &     .014         &   (.014)&    -.041\sym{**} &   (.020)&    -.009         &   (.013)&     .032\sym{*}  &   (.019)\\ Big city  &     .008         &   (.008)&     .027\sym{*}  &   (.014)&    -.004         &   (.009)&     .093\sym{***}&   (.013)\\ Central   &     .011         &   (.011)&     .024         &   (.017)&     .015         &   (.011)&    -.035\sym{**} &   (.017)\\ Westcentral&    -.002         &   (.010)&     .021         &   (.017)&    -.002         &   (.010)&    -.006         &   (.018)\\ Northeastcentral&     .007         &   (.012)&     .005         &   (.017)&     .009         &   (.012)&    -.051\sym{***}&   (.017)\\ Northwestcentral&    -.006         &   (.009)&    -.033\sym{**} &   (.017)&     .007         &   (.011)&    -.095\sym{***}&   (.017)\\ Primary   &    -.009         &   (.020)&     .060\sym{**} &   (.027)&     .017         &   (.018)&    -.011         &   (.019)\\ Secondary &    -.003         &   (.020)&     .056\sym{*}  &   (.030)&    -.005         &   (.018)&     .052\sym{**} &   (.021)\\ Highschool&    -.027         &   (.020)&     .045         &   (.031)&    -.008         &   (.020)&     .117\sym{***}&   (.026)\\ College or university&    -.018         &   (.023)&     .057\sym{*}  &   (.032)&    -.028         &   (.020)&     .291\sym{***}&   (.025)\\ Indigenous&     .009         &   (.010)&     .005         &   (.017)&     .012         &   (.013)&    -.006         &   (.018)\\ Married   &     .015\sym{**} &   (.007)&     .086\sym{***}&   (.012)&    -.002         &   (.007)&    -.216\sym{***}&   (.011)\\ Children (under 15)&    -.005\sym{**} &   (.002)&     .010\sym{**} &   (.004)&     .003         &   (.002)&    -.016\sym{***}&   (.004)\\ Wealth    &     .003         &   (.004)&    -.001         &   (.007)&     .003         &   (.004)&     .030\sym{***}&   (.006)\\ Parental education&     .019\sym{**} &   (.009)&    -.010         &   (.013)&     .014         &   (.009)&    -.001         &   (.011)\\ Diabetes father&     .068\sym{***}&   (.020)&                  &         &     .035\sym{**} &   (.014)&                  &         \\ Diabetes mother&     .043\sym{***}&   (.016)&                  &         &     .055\sym{***}&   (.013)&                  &         \\ Diabetes  &                  &         &     .098         &   (.215)&                  &         &     .239         &   (.214)\\ Constant  &    -.015         &   (.022)&     .607\sym{***}&   (.036)&    -.020         &   (.021)&     .289\sym{***}&   (.027)\\ \midrule R2        &     .075         &         &     .067         &         &     .090         &         &     .120         &         \\ F stat (H0: weak instruements)&                  &         &   20.483         &         &                  &         &   27.706         &         \\ Sargan test (H0: valid instruments)&              &         &     .862         &         &            &         &     .295         &         \\ \hspace{10 mm}p value&                  &         &     .353         &         &                  &         &     .587         &         \\ Endogeneity (H0: Diabetes exogenous)&                  &         &     .864         &         &                  &         &    1.796         &         \\ \hspace{10 mm}p value&                  &         &     .353         &         &                  &         &     .180         &         \\ N         &     6228         &         &     6286         &         &     8186         &         &     8243         &         \\ \bottomrule
\end{tabular} 
\begin{tablenotes}
\item \textit{Notes} Robust standard errors in parentheses. Instruments: diabetes of mother, diabetes of father. Other control variables: age, region, urban, education, indigenous marital status, children, wealth, parental education.
\item \sym{*} \(p<0.10\), \sym{**} \(p<0.05\), \sym{***} \(p<0.01\))
\end{tablenotes}
}
\end{threeparttable} \end{adjustbox}
\end{center}
\end{table}
\end{landscape}
\clearpage


\subsection*{Results for older age groups}


\begin{table}[p]
\protect\caption{\label{tab:Impact-of-diabetes-age-groups-1}Impact of diabetes on
employment probabilities by age groups older than 44 (probit)}


\begin{center}
\begin{adjustbox}{max width=\textwidth, center} \begin{threeparttable}

{ \def\sym#1{\ifmmode^{#1}\else\(^{#1}\)\fi} \begin{tabular}{l*{4}{S S}} \toprule           &\multicolumn{2}{c}{45-54}            &\multicolumn{2}{c}{55-64}            \\\cmidrule(lr){2-3}\cmidrule(lr){4-5}           &\multicolumn{1}{c}{(1)}&\multicolumn{1}{c}{(2)}&\multicolumn{1}{c}{(3)}&\multicolumn{1}{c}{(4)}\\           &\multicolumn{1}{c}{Males}&\multicolumn{1}{c}{Females}&\multicolumn{1}{c}{Males}&\multicolumn{1}{c}{Females}\\ \midrule Diabetes  &    -.083\sym{*}  &    -.076\sym{**} &    -.128\sym{**} &    -.033         \\           &   (.048)         &   (.034)         &   (.056)         &   (.039)         \\ \midrule Log likelihood& -451.544         & -764.722         & -458.632         & -392.174         \\ N         &     1101         &     1399         &      770         &      847         \\ \bottomrule \end{tabular} 
\begin{tablenotes}
\item \textit{Notes}  Average marginal effects; robust standard errors in parentheses. Other control variables: region, urban, education, indigenous, marital status, children, wealth, parental education.
\item \sym{*} \(p<0.10\), \sym{**} \(p<0.05\), \sym{***} \(p<0.01\))
\end{tablenotes}
}
\end{threeparttable} \end{adjustbox}
\end{center}
\end{table}


\clearpage


\subsection*{Results for wealth quartiles}
\begin{landscape}
\begin{table}[p]
\protect\caption{\label{tab:Impact-of-diabetes-wealth-quartile}Impact of diabetes
on employment probabilities by wealth quartile (probit)}
\begin{center}
\begin{adjustbox}{max width=\linewidth, center}
\begin{threeparttable}

{ \def\sym#1{\ifmmode^{#1}\else\(^{#1}\)\fi} \begin{tabular}{l*{8}{S S}} \toprule           &\multicolumn{2}{c}{1st}              &\multicolumn{2}{c}{2nd}              &\multicolumn{2}{c}{3rd}              &\multicolumn{2}{c}{4th}              \\\cmidrule(lr){2-3}\cmidrule(lr){4-5}\cmidrule(lr){6-7}\cmidrule(lr){8-9}           &\multicolumn{1}{c}{(1)}&\multicolumn{1}{c}{(2)}&\multicolumn{1}{c}{(3)}&\multicolumn{1}{c}{(4)}&\multicolumn{1}{c}{(5)}&\multicolumn{1}{c}{(6)}&\multicolumn{1}{c}{(7)}&\multicolumn{1}{c}{(8)}\\           &\multicolumn{1}{c}{Males}&\multicolumn{1}{c}{Females}&\multicolumn{1}{c}{Males}&\multicolumn{1}{c}{Females}&\multicolumn{1}{c}{Males}&\multicolumn{1}{c}{Females}&\multicolumn{1}{c}{Males}&\multicolumn{1}{c}{Females}\\ \midrule Diabetes  &    -.142\sym{*} &    -.101\sym{***}&    -.144\sym{**}&     .028         &    -.082  &    -.026         &    -.040         &    -.053         \\           &   (.077)         &   (.029)         &   (.060)         &   (.048)         &   (.053)         &   (.044)         &   (.046)         &   (.048)         \\ \midrule Log likelihood& -776.619         & -937.144         & -672.633         &-1092.280         & -689.910         &-1266.304         & -703.495         &-1144.588         \\ N         &     1577         &     2039         &     1563         &     2052         &     1516         &     2143         &     1590         &     1974         \\ \bottomrule \end{tabular} 
\begin{tablenotes}
\item \textit{Notes}  Average marginal effects; robust standard errors in parentheses.Other control variables: region, urban, education, indigenous, marital status, children, wealth, parental education.
\item \sym{*} \(p<0.10\), \sym{**} \(p<0.05\), \sym{***} \(p<0.01\))
\end{tablenotes}
}
\end{threeparttable} 
\end{adjustbox}
\end{center}
\end{table}
\end{landscape}

\clearpage


\subsection*{Instrumental variable analysis for age groups}


The results of the bivariate probit models do not indicate
endogeneity for the older age group and for males in the younger age
group (see Tables \ref{tab:IV-estimates-forYOUNG} and \ref{tab:IV-estimates-forOLDAGE}),
suggesting that particularly for males the results of the more efficient
probit model (Table \ref{tab:age groups probit}) show the true effect
of diabetes on employment chances. Only for females in the younger
age group the test for endogeneity rejects the assumption of exogeneity
and the diabetes coefficient---surprisingly---shows a strong positive
effect of diabetes on female employment chances. Instrument strength,
however, is reduced significantly, which together with the very low
treatment probabilities questions the validity of the \ac{IV} results
for the sample of the younger age group, as weak instruments possibly
introduce a bias similar to or stronger than the potential bias in
the probit estimates \parencite{Staiger1997}. We therefore additionally
apply a method proposed by \textcite{Lewbel2012}, which uses heteroscedasticity
in the estimated models to construct additional instruments. Instruments
are generated by multiplying the heteroscedastic residuals from the
first-stage regressions with a subset of the included exogenous variables.
\textcite{Lewbel2012} recommends the use of this method when traditional
instruments are not available or if it is suspected that the traditional
instrument is too weak for identification, which is the issue at hand.
The approach has been widely used over the last years both in health
economics \parencite{Drichoutis2011,Kelly2012,Schroeter2012,Brown2014}
and in other economic disciplines \parencite{Huang2009a,Emran2012,Denny2013}.
Using this method to construct additional instruments by using our
age group dummies, we are able to increase instrument strength significantly
in the younger age group and the overidentification test indicates
validity of the instruments. The results of the linear \ac{IV} model
with the additional instruments show exogeneity of diabetes for males
and females and do not indicate a significant positive effect of diabetes
on employment chances.

Apart from the results of the \citeauthor{Lewbel2012} approach, we
also think that there are theoretical reasons why diabetes is likely
exogenous in the younger age group. While we cannot distinguish between
the types of diabetes with the data at hand, it is likely that a relatively
large proportion of the people reporting diabetes in this age group
have type 1 diabetes, which people tend to get at a younger age \parencite{Maahs2010}.
The disease has a strong genetic component and it is very unlikely
that there are unobserved factors that affect the chances to develop
type 1 diabetes and being employed at the same time, nor that employment
status would affect the development of type 1 diabetes. Therefore,
for a large part of the people reporting diabetes in the younger age
group, endogeneity should not present a problem because they have
type 1 diabetes. Furthermore, it is also less likely that reverse
causality is a problem for those having type 2 diabetes in this age
group, because any effects of being employed on developing type 2
diabetes take time to develop. It would be reasonable to expect that
if being employed affected a person's weight or any other diabetes
risk factor, this would happen by changing the person's lifestyle
due to changes in income or available leisure time, or by reducing
or increasing a person's activity levels at work. Until these changes
are expressed in changes in weight or any other risk factor for diabetes
and finally cause a development of type 2 diabetes, a considerable
time period of various years has likely passed and people have reached
an advanced age. We therefore believe, that the risk of diabetes being
affected by employment is much lower in the younger age group based
on the nature of the disease, compared to the older age group. Hence
we think that the assumption of exogeneity of diabetes in the younger
age group is valid---which is also supported by the Lewbel estimates---and that the endogeneity indicated for younger females in the
bivariate probit model is likely the result of the low prevalence
rates, and consequently the very low treatment probabilities, together
with weak instruments, making a meaningful \ac{IV} analysis difficult
\parencite{Chiburis2012}. We are therefore confident that we can rely
on our probit estimates for inference.

\begin{table}[p]
\protect\caption{\label{tab:IV-estimates-forYOUNG}IV estimates for the age group 15--44}


\begin{center}
\begin{adjustbox}{max width=\textwidth, center} 
\begin{threeparttable}

{ \def\sym#1{\ifmmode^{#1}\else\(^{#1}\)\fi} \begin{tabular}{l*{4}{S S}} \toprule           &\multicolumn{2}{c}{BP}                       &\multicolumn{2}{c}{Lewbel IV}        \\\cmidrule(lr){2-3}\cmidrule(lr){4-5}           &\multicolumn{1}{c}{(1)}&\multicolumn{1}{c}{(2)}&\multicolumn{1}{c}{(3)}&\multicolumn{1}{c}{(4)}\\           &\multicolumn{1}{c}{Males}&\multicolumn{1}{c}{Females}&\multicolumn{1}{c}{Males}&\multicolumn{1}{c}{Females}\\ \midrule Diabetes  &      .171\sym{***}            &     .496\sym{***}   &             .007         &     .051         \\           &     (.046)             &   (.080)               &   (.053)         &   (.071)         \\ \midrule R2        &                  &                  &     .093         &     .143          \\ Score goodness-of-fit (H0=normality of errors)&   9.56            &   14.25           &             &         \\ \hspace{10 mm}p value&   0.387               &   0.114           &              &           \\F stat (H0: weak instruments)&         4.288$^a$         &   10.835$^a$         &  366.480         &   65.872         \\ Sargan test (H0: valid instruments)&  .008$^a$         &     .044$^a$         &    1.817         &    3.487         \\ \hspace{10 mm}p value&    .930$^a$         &     .834$^a$         &     .611         &     .322         \\ Endogeneity (H0: Diabetes exogenous)&      1.422            &      12.948                 &    1.065         &    1.429         \\ \hspace{10 mm}p value&    .233          &      .000      &     .302         &     .232         \\ N         &     4415         &     5997        &     4415         &     5997         \\ \bottomrule 
 \end{tabular}
\begin{tablenotes}
\item \textit{Notes}  Average marginal effects for bivariate probit (BP); robust standard errors in parentheses. Instruments: diabetes of mother, diabetes of father; for Lewbel additionally created age groups instruments. The models contain the age categories 25--34 and 35--44 with 15--24 as the reference category. Other control variables: region, urban, education, indigenous, marital status, children, wealth, parental education. $^a$ The test statistics are taken from the linear IV model not presented here.
\item \sym{*} \(p<0.10\), \sym{**} \(p<0.05\), \sym{***} \(p<0.01\))
\end{tablenotes}
}
\end{threeparttable} 
\end{adjustbox}
\end{center}
\end{table}


\clearpage

\begin{table}[p]
\protect\caption{\label{tab:IV-estimates-forOLDAGE}IV estimates for the age group
45--64}


\begin{center}
\begin{adjustbox}{max width=\textwidth, center} 
\begin{threeparttable}

{ \def\sym#1{\ifmmode^{#1}\else\(^{#1}\)\fi} \begin{tabular}{l*{4}{S S}} \toprule           &\multicolumn{2}{c}{BP}                     &\multicolumn{2}{c}{Lewbel IV}        \\\cmidrule(lr){2-3}\cmidrule(lr){4-5}           &\multicolumn{1}{c}{(1)}&\multicolumn{1}{c}{(2)}&\multicolumn{1}{c}{(3)}&\multicolumn{1}{c}{(4)}\\           &\multicolumn{1}{c}{Males}&\multicolumn{1}{c}{Females}&\multicolumn{1}{c}{Males}&\multicolumn{1}{c}{Females}\\ \midrule Diabetes  &    -.022              &    -.112        &    -.178         &    -.042         \\           &      (.138)            &     (.111)                &   (.160)         &   (.104)         \\ \midrule R2        &                  &                       &     .058         &     .118      \\ Score goodness-of-fit (H0=normality of errors)&     7.00          &  11.10            &             &         \\ \hspace{10 mm}p value&  0.637                &   0.269           &              &           \\ F stat. (H0: weak instruments)&  15.408$^a$         &   18.305$^a$         &   12.534         &   18.897         \\ Sargan test (H0: valid instruments)&    2.717$^a$         &     .482$^a$         &    4.397         &    1.688         \\ \hspace{10 mm}p value&    .067$^a$         &     .487$^a$         &     .111         &     .430         \\ Endogeneity (H0: Diabetes exogenous)& .688                &      .574       &     .082         &     .024         \\ \hspace{10 mm}p value&         .407             &      0.449     &     .774         &     .876         \\ N         &     1871         &     2246              &     1871         &     2246         \\ \bottomrule 
\end{tabular}
\begin{tablenotes}
\item \textit{Notes}  Average marginal effects for bivariate probit (BP); robust standard errors in parentheses. Instruments: diabetes of mother, diabetes of father; for Lewbel additionally created age groups instruments. The models contain the age categories 55--64 with 45--54 as the reference category. Other control variables: region, urban, education, indigenous, marital status, children, wealth, parental education. $^a$ The test statistics are taken from the linear IV model not presented here.
\item \sym{*} \(p<0.10\), \sym{**} \(p<0.05\), \sym{***} \(p<0.01\))
\end{tablenotes}
}
\end{threeparttable} 
\end{adjustbox}
\end{center}
\end{table}


\clearpage


\subsection*{Instrumental variable analysis for wealth groups}


To consider the possible endogeneity of diabetes in the upper and
lower wealth half, we again present the results of the bivariate probit
and the Lewbel model. The stratification into wealth groups significantly
reduces instrument power as well as sample size. For none of the wealth
groups the bivariate probit model indicates endogeneity (see Table
\ref{tab:Impact-of-diabetes-wealth-IV} and Table \ref{tab:Impact-of-diabetes_wealth_rich}\emph{)}.
This does not change even when using the Lewbel approach to increase
instrument strength. Accordingly, we do not find any indication of
endogeneity of diabetes in the wealth groups and rely on our probit
estimates for inference.

\begin{table}[p]
\protect\caption{\label{tab:Impact-of-diabetes-wealth-IV}IV results for lower wealth
half}


\begin{center}
\begin{adjustbox}{max width=\textwidth, center} 
\begin{threeparttable}

{ \def\sym#1{\ifmmode^{#1}\else\(^{#1}\)\fi} \begin{tabular}{l*{4}{S S}} \toprule           &\multicolumn{2}{c}{BP}                    &\multicolumn{2}{c}{Lewbel IV}        \\\cmidrule(lr){2-3}\cmidrule(lr){4-5}           &\multicolumn{1}{c}{(1)}&\multicolumn{1}{c}{(2)}&\multicolumn{1}{c}{(3)}&\multicolumn{1}{c}{(4)}\\           &\multicolumn{1}{c}{Males}&\multicolumn{1}{c}{Females}&\multicolumn{1}{c}{Males}&\multicolumn{1}{c}{Females}\\ \midrule Diabetes  &   -.354               &     -.064            &    -.142\sym{***}&    -.054\sym{*}  \\          &    (.241)              &       (.139)                   &   (.050)         &   (.032)         \\ \midrule R2        &                  &                      &     .071         &     .099         \\  Score goodness-of-fit (H0=normality of errors)&   NA$^a$            &  7.41            &             &         \\ \hspace{10 mm}p value&      NA$^a$           &  0.594            &              &      \\  F stat (H0: weak instruments)&    6.322$^b$         &   15.420$^b$         & 2589.091         & 1311.647         \\ Sargan test (H0: valid instruments)&                 .342$^b$         &    1.106$^b$         &    4.169         &    2.804         \\ \hspace{10 mm}p value& .558$^b$         &     .293$^b$         &     .525         &     .730         \\ Endogeneity (H0: Diabetes exogenous)&   1.190               &     .016        &     .005         &     .156         \\ \hspace{10 mm}p value&     0.275             &    0.901              &     .941         &     .693         \\ N         &     3169         &     4111        &     3169         &     4111         \\ \bottomrule
\end{tabular}
\begin{tablenotes}
\item \textit{Notes}  Average marginal effects for bivariate probit (BP); robust standard errors in parentheses. Instruments: diabetes of mother, diabetes of father; for Lewbel additionally created age groups instruments. Other control variables: region, urban, education, indigenous, marital status, children, wealth, parental education. $^a$ The test statistics are taken from the linear IV model not presented here. The command SCOREGOF failed to produce the test statistic for this subsample.
\item \sym{*} \(p<0.10\), \sym{**} \(p<0.05\), \sym{***} \(p<0.01\))
\end{tablenotes}
}
\end{threeparttable}
 \end{adjustbox}
\end{center}
\end{table}


\clearpage

\begin{table}[p]
\protect\caption{\label{tab:Impact-of-diabetes_wealth_rich}IV results for upper wealth
half}


\begin{center}
\begin{adjustbox}{max width=\textwidth, center} 
\begin{threeparttable}

{ \def\sym#1{\ifmmode^{#1}\else\(^{#1}\)\fi} \begin{tabular}{l*{4}{S S}} \toprule &\multicolumn{2}{c}{BP}          &\multicolumn{2}{c}{Lewbel IV}        \\\cmidrule(lr){2-3}\cmidrule(lr){4-5}          &\multicolumn{1}{c}{(1)}&\multicolumn{1}{c}{(2)}&\multicolumn{1}{c}{(3)}&\multicolumn{1}{c}{(4)}\\           &\multicolumn{1}{c}{Males}&\multicolumn{1}{c}{Females}&\multicolumn{1}{c}{Males}&\multicolumn{1}{c}{Females}\\ \midrule Diabetes  &  -.142           &      .103    &    -.057         &    -.000         \\           &  (.199)            &       (.203)          &   (.037)         &   (.039)         \\ \midrule R2        &                  &                  &     .089         &     .142        \\ Score goodness-of-fit (H0=normality of errors)&  11.40             &  12.92            &             &         \\ \hspace{10 mm}p value&      0.249            &      0.166        &              &         \\  F stat (H0: weak instruments)&        14.003$^a$         &   13.215$^a$         &28673.088         & 1225.456         \\ Sargan test (H0: valid instruments)&   .848$^a$         &     .019$^a$         &   10.180         &    5.787         \\ \hspace{10 mm}p value&        .357$^a$         &     .889$^a$         &     .070         &     .327         \\ Endogeneity (H0: Diabetes exogenous)&   .238               &    .730                    &     .955         &    1.807         \\ \hspace{10 mm}p value&   0.626               &      0.393            &     .329         &     .179         \\ N         &     3117         &     4132          &     3117         &     4132         \\ \bottomrule
\end{tabular}
\begin{tablenotes}
\item \textit{Notes}  Average marginal effects for bivariate probit (BP); robust standard errors in parentheses. Instruments: diabetes of mother, diabetes of father; for Lewbel additionally created age groups instruments. Other control variables: region, urban, education, indigenous, marital status, children, wealth, parental education. $^a$ The test statistics are taken from the linear IV model not presented here.
\item \sym{*} \(p<0.10\), \sym{**} \(p<0.05\), \sym{***} \(p<0.01\))
\end{tablenotes}
}
\end{threeparttable} 
\end{adjustbox}
\end{center}
\end{table}


\clearpage


\subsection*{Multinomial logit and IV results for formal and informal employment}


\begin{table}[p]
\protect\caption{\label{tab:Effect-of-diabetes-formal-informal-Mlogit}Impact of diabetes
on employment probabilities by employment status (multinomial logit)}


\begin{center}
\begin{adjustbox}{max width=\textwidth, center} 
\begin{threeparttable}

{ \def\sym#1{\ifmmode^{#1}\else\(^{#1}\)\fi} \begin{tabular}{l*{4}{S S}} \toprule           &\multicolumn{2}{c}{Males}            &\multicolumn{2}{c}{Females}          \\\cmidrule(lr){2-3}\cmidrule(lr){4-5}           &\multicolumn{1}{c}{(1)}&\multicolumn{1}{c}{(2)}&\multicolumn{1}{c}{(3)}&\multicolumn{1}{c}{(4)}\\           &\multicolumn{1}{c}{Informal}&\multicolumn{1}{c}{Formal}&\multicolumn{1}{c}{Informal}&\multicolumn{1}{c}{Formal}\\ \midrule Diabetes  &    -.073\sym{**}&    0.031&    -.044\sym{**} &     .008  \\           &   (.031)         &   (.026)         &   (.019)         &   (.018)         \\ \midrule Log likelihood&-4997.064         &-4997.064         &-6267.941         &-6267.941         \\ N         &     6286         &     6286         &     8243         &     8243         \\ \bottomrule 
\end{tabular} 
\begin{tablenotes}
\item \textit{Notes} Average marginal effects. Base category is being unemployed. Other control variables: region, urban, education, indigenous, marital status, children, wealth, parental education.
\item \sym{*} \(p<0.10\), \sym{**} \(p<0.05\), \sym{***} \(p<0.01\))
\end{tablenotes}
}
\end{threeparttable} 
\end{adjustbox}
\end{center}
\end{table}


\clearpage

To consider the possible endogeneity of diabetes when estimating its
effect on formal and informal employment, we again present the results
of the bivariate probit and the Lewbel model. The stratification into
formal and informal employment groups significantly reduces instrument
power as well as sample size. For none of the employment groups the
bivariate probit model indicates endogeneity (see Table \ref{tab:Impact-of-diabetes-informal-IV}
and Table \ref{tab:Impact-of-diabetes-formal-IV}). This does not
change even when using the Lewbel approach to increase instrument
strength. Accordingly, we do not find any indication of endogeneity
of diabetes for the stratification into formal and informal employment
and rely on our probit estimates for inference.

\begin{table}[p]
\protect\caption{\label{tab:Impact-of-diabetes-informal-IV}IV results for informal
employment}

\begin{center}
\begin{adjustbox}{max width=\textwidth, center}
\begin{threeparttable}

{ \def\sym#1{\ifmmode^{#1}\else\(^{#1}\)\fi} \begin{tabular}{l*{4}{S S}} \toprule           &\multicolumn{2}{c}{BP}               &\multicolumn{2}{c}{Lewbel IV}        \\\cmidrule(lr){2-3}\cmidrule(lr){4-5}           &\multicolumn{1}{c}{(1)}&\multicolumn{1}{c}{(2)}&\multicolumn{1}{c}{(3)}&\multicolumn{1}{c}{(4)}\\           &\multicolumn{1}{c}{Male}&\multicolumn{1}{c}{Female}&\multicolumn{1}{c}{Male}&\multicolumn{1}{c}{Female}\\ \midrule Diabetes  &    -.046              &      .069            &    -.048         &    -.037         \\           &      (.123)            &     (.130)             &   (.030)         &   (.025)         \\ \midrule R2        &                  &                  &     .103         &     .088        \\ Score goodness-of-fit (H0=normality of errors)& 13.84 & 17.37 & &  \\ \hspace{10 mm}p value& 0.128  & 0.043 & &   \\ F stat (H0: weak instruments)&   13.565$^a$         &   25.123$^a$         & 5349.118         & 2536.362     \\ Sargan test (H0: valid instruments)&      .551$^a$         &    1.684$^a$        &    4.067         &    4.063         \\ \hspace{10 mm}p value&   .458$^a$         &     .194$^a$             &     .540         &     .540         \\ Endogeneity (H0: Diabetes exogenous)&    .025              &   1.152       &    1.128         &     .722         \\ \hspace{10 mm}p value&    0.873              &     0.283      &     .288         &     .395         \\ N         &     4604         &     6983         &     4604         &     6983         \\     \\ \bottomrule  \end{tabular} 
\begin{tablenotes}
\item \textit{Notes}  Average marginal effects for bivariate probit (BP); robust standard errors in parentheses. Instruments: diabetes of mother, diabetes of father; for Lewbel additionally created age groups instruments. The models contain the age categories 55--64 with 45--54 as the reference category. Other control variables: region, urban, education, indigenous, marital status, children, wealth, parental education. $^a$ The test statistics are taken from the linear IV model not presented here. Base category is being unemployed.
\item \sym{*} \(p<0.10\), \sym{**} \(p<0.05\), \sym{***} \(p<0.01\))
\end{tablenotes}
}
\end{threeparttable}
\end{adjustbox}
\end{center}
\end{table}


\clearpage

\begin{table}[p]
\protect\caption{\label{tab:Impact-of-diabetes-formal-IV}IV results for formal employment}


\begin{center}
\begin{adjustbox}{max width=\textwidth, center} 
\begin{threeparttable}

{ \def\sym#1{\ifmmode^{#1}\else\(^{#1}\)\fi} \begin{tabular}{l*{4}{S S}} \toprule           &\multicolumn{2}{c}{BP}               &\multicolumn{2}{c}{Lewbel IV}        \\\cmidrule(lr){2-3}\cmidrule(lr){4-5}           &\multicolumn{1}{c}{(1)}&\multicolumn{1}{c}{(2)}&\multicolumn{1}{c}{(3)}&\multicolumn{1}{c}{(4)}\\           &\multicolumn{1}{c}{Male}&\multicolumn{1}{c}{Female}&\multicolumn{1}{c}{Male}&\multicolumn{1}{c}{Female}\\ \midrule Diabetes  & .098                 &     -.103             &    -.022         &     .003         \\           &   (.195)               &      (.069)            &   (.049)         &   (.021)         \\ \midrule R2        &                  &                  &     .256         &     .262        \\ Score goodness-of-fit (H0=normality of errors)& 12.95 & 8.03 & &  \\ \hspace{10 mm}p value& 0.165 & 0.531  & &  \\ F stat (H0: weak instruments)&     8.518$^a$             &      19.996$^a$            & 2764.273         & 1647.887         \\ Sargan test (H0: valid instruments)&   1.111$^a$               &        1.075$^a$          &    9.286         &    6.741         \\ \hspace{10 mm}p value&    .292$^a$         &     .300$^a$      &     .098         &     .241         \\ Endogeneity (H0: Diabetes exogenous)&      .516            &       1.833           &    1.602         &     .318         \\ \hspace{10 mm}p value&   0.473               &       0.176           &     .206         &     .573         \\ N         &     2204         &     5652         &     2204         &     5652         \\ \bottomrule 
\end{tabular} 
\begin{tablenotes}
\item \textit{Notes}  Average marginal effects for bivariate probit (BP); robust standard errors in parentheses. Instruments: diabetes of mother, diabetes of father; for Lewbel additionally created age groups instruments. The models contain the age categories 55--64 with 45--54 as the reference category. Other control variables: region, urban, education, indigenous, marital status, children, wealth, parental education. $^a$ The test statistics are taken from the linear IV model not presented here. Base category is being unemployed.
\item \sym{*} \(p<0.10\), \sym{**} \(p<0.05\), \sym{***} \(p<0.01\))
\end{tablenotes}
}
\end{threeparttable} 
\end{adjustbox}
\end{center}
\end{table}



